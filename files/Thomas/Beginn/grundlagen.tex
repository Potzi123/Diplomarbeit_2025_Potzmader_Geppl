\begin{inhalt}
\chapter{Grundlagen \& Methoden}
\renewcommand*\chapterpagestyle{scrheadings}

Verwendete Technologien 
\section{Web Entwicklung}

Web-Entwicklung \cite{WebEntwicklungWiki} bezeichnet die Erstellung und Gestaltung von Websites und Webanwendungen, die über das Internet zugänglich sind. Dabei werden verschiedene Technologien und Programmiersprachen genutzt, um Inhalte darzustellen, interaktive Funktionen bereitzustellen und Daten zu verarbeiten. Eine Website besteht typischerweise aus dem Frontend, die dem Nutzer präsentiert wird, sowie einen Backend, die im Hintergrund abläuft und beispielsweise Daten verarbeitet oder speichert. Dabei kommen moderne Frameworks und Entwicklungsumgebungen zum Einsatz, um eine effiziente und ansprechende Umsetzung zu ermöglichen. 

\subsection{Frontend} 

Das Frontend \cite{WebEntwicklungFrontendWiki} umfasst alle Komponenten einer Website oder Webanwendung, die direkt mit dem Nutzer interagieren. Es stellt die visuelle und funktionale Oberfläche bereit, über die Inhalte dargestellt und Aktionen durchgeführt werden können. Typische Technologien im Frontend-Bereich sind HTML, CSS und JavaScript. Moderne Frameworks wie React, Angular oder Vue.js ermöglichen dabei die Erstellung dynamischer und reaktionsschneller Benutzeroberflächen. Das Ziel des Frontends ist es, ein ansprechendes, intuitives und barrierefreies Nutzererlebnis zu gewährleisten. Dabei spielen Aspekte wie responsives Design, Performance und Zugänglichkeit eine wesentliche Rolle. Das Frontend, wird in den Kapiteln bis 2.1.6 beschrieben. 

\subsection{Backend}

Das Backend \cite{WebEntwicklungFrontendBackendWiki} bildet das Rückgrat einer Webanwendung und arbeitet im Hintergrund, um Daten zu verarbeiten, Geschäftslogik umzusetzen und Verbindungen zu Datenbanken herzustellen. Es ist nicht direkt für den Endnutzer sichtbar, sorgt jedoch dafür, dass alle Funktionen einer Website reibungslos ablaufen. Gängige Programmiersprachen und Frameworks im Backend-Bereich sind unter anderem PHP, Python, Ruby, Java, Node.js oder .NET. Die Aufgaben des Backends umfassen unter anderem Authentifizierung, Autorisierung, API-Entwicklung, Datenverwaltung und Sicherheitsaspekte. Durch die enge Zusammenarbeit mit dem Frontend wird eine nahtlose Integration und effiziente Datenübertragung zwischen Client und Server gewährleistet.

\subsection{Datenbank}

Eine Datenbank \cite{DatenBankWiki} bildet das zentrale Element zur Speicherung, Verwaltung und Abfrage von Daten innerhalb einer Webanwendung. Sie ermöglicht es, Informationen strukturiert abzulegen und bei Bedarf effizient abzurufen oder zu aktualisieren. Dabei wird oft zwischen relationalen Datenbanken, wie MySQL, PostgreSQL oder Oracle, und NoSQL-Datenbanken, wie MongoDB oder Cassandra, unterschieden. Relationale Datenbanken nutzen Tabellen und vordefinierte Beziehungen, um Daten zu organisieren, während NoSQL-Datenbanken flexiblere Datenmodelle bieten, die insbesondere bei großen, unstrukturierten Datenmengen Vorteile bieten können. Durch die enge Integration der Datenbank mit dem Backend wird sichergestellt, dass die Webanwendung zuverlässig und performant auf die benötigten Daten zugreifen kann.

\subsection{Next.js}
Next.js \cite{NextJSWiki} ist ein modernes Framework für die Entwicklung von React-Anwendungen, das sowohl serverseitiges Rendering als auch statische Seitengenerierung unterstützt. Durch diese Funktionen können Entwickler performante und SEO-freundliche Webanwendungen erstellen. Next.js vereinfacht die Handhabung von Routing, API-Routen und anderen komplexen Aufgaben, indem es eine klare und strukturierte Entwicklungsumgebung bietet. Dies führt zu einer verbesserten Entwicklererfahrung und ermöglicht die effiziente Erstellung skalierbarer Webprojekte.

\subsection{ReactWiki}
React \cite{ReactWiki} ist eine JavaScript-Bibliothek zur Erstellung von Benutzeroberflächen, die es Entwicklern ermöglicht, wiederverwendbare UI-Komponenten zu erstellen. Es bildet die Grundlage für Next.js und konzentriert sich darauf, den Zustand und die Darstellung von Komponenten effizient zu verwalten. Durch das deklarative Programmiermodell wird die Entwicklung von interaktiven Anwendungen vereinfacht, während gleichzeitig eine hohe Performance und Skalierbarkeit gewährleistet werden kann.

\subsection{TypescriptWiki}
Typescript \cite{TypescriptWiki} erweitert JavaScript um statische Typisierung und andere Features, die die Codequalität und Wartbarkeit von Anwendungen verbessern. Durch die frühzeitige Erkennung von Fehlern im Code und die Unterstützung moderner JavaScript-Funktionalitäten bietet Typescript eine solide Basis für die Entwicklung von robusten und fehlerarmen Anwendungen. Viele moderne Frameworks, darunter auch Next.js, profitieren von den zusätzlichen Sicherheitsmechanismen und der besseren Entwicklerunterstützung, die Typescript bietet.

\subsection{TailwindWiki}
Tailwind \cite{TailwindWiki} ist ein Utility-first CSS-Framework, das es ermöglicht, direkt im Markup stilisierte Komponenten zu erstellen. Anstatt vordefinierte Komponenten zu nutzen, bietet Tailwind eine umfangreiche Sammlung an Klassen, die individuell kombiniert werden können, um maßgeschneiderte Designs zu realisieren. Dies führt zu einem flexiblen und effizienten Styling-Prozess, bei dem Entwickler schnell und ohne umfangreiche CSS-Dateien arbeiten können. Tailwind unterstützt dabei die Erstellung von responsiven und modernen Benutzeroberflächen, die sich leicht an unterschiedliche Designanforderungen anpassen lassen.

\subsection{ShadCN}
ShadCN \cite{ShadCN} ist eine moderne UI-Komponentenbibliothek, die speziell für die Erstellung ansprechender und konsistenter Benutzeroberflächen entwickelt wurde. Sie integriert sich nahtlos in moderne Frontend-Frameworks und bietet eine Vielzahl von wiederverwendbaren Komponenten, die den Entwicklungsprozess beschleunigen und die Wartbarkeit der Anwendungen verbessern.


\subsection{Zustand} \label{subsec:Zustand} 
Zustand \cite{Zustand} ist ein leichtgewichtiges und flexibles State-Management-Tool für React-Anwendungen, das sich auf die Nutzung von Hooks stützt. Es ermöglicht Entwicklern, globale Zustände einfach zu definieren und zu verwalten, ohne dabei auf komplexe Provider- oder Context-Modelle zurückgreifen zu müssen. Durch den Verzicht auf übermäßigen Boilerplate-Code und die intuitive API bietet Zustand eine effiziente Alternative zu anderen State-Management-Lösungen. Die Bibliothek zeichnet sich durch ihre hohe Performance und einfache Skalierbarkeit aus, was sie besonders attraktiv für moderne, dynamische Webanwendungen macht.

\subsection{Zod}
\label{subsec:Zod}
Zod \cite{Zod} ist eine TypeScript-orientierte Validierungsbibliothek, die es ermöglicht, Datenstrukturen präzise zu definieren und zur Laufzeit zu überprüfen. Durch die Nutzung von Zod können Entwickler sicherstellen, dass ihre Anwendungen robust gegenüber fehlerhaften oder unerwarteten Daten sind, was die Zuverlässigkeit und Wartbarkeit des Codes erheblich verbessert.

\subsection{Supabase}
Supabase \cite{Supabase} ist eine umfassende Backend-as-a-Service-Plattform, die Entwicklern eine Alternative zu traditionellen Backend-Lösungen bietet. Sie kombiniert mehrere Funktionen, die eine schnelle und sichere Entwicklung von Webanwendungen ermöglichen.

\subsubsection{Supabase DB}
Die Supabase DB \cite{SupabaseDB} basiert auf PostgreSQL und bietet eine skalierbare und leistungsfähige relationale Datenbanklösung. Sie ermöglicht die Speicherung, Abfrage und Verwaltung von Daten in Echtzeit und ist dabei vollständig in die Supabase-Plattform integriert, um eine reibungslose Datenverwaltung zu gewährleisten.

\subsubsection{Supabase Auth}
Supabase Auth \cite{SupabaseAuth} stellt eine vollständige Authentifizierungs- und Autorisierungslösung bereit. Es unterstützt verschiedene Authentifizierungsmethoden, wie E-Mail/Passwort, OAuth und weitere, und ermöglicht so eine einfache Integration sicherer Login-Mechanismen in Webanwendungen.

\subsubsection{Supabase Storage}
Supabase Storage \cite{SupabaseStorage} bietet eine skalierbare Lösung zur Speicherung und Verwaltung von Dateien. Es ermöglicht Entwicklern, Medien und Dokumente effizient zu speichern, abzurufen und zu verwalten, wobei der Zugriff auf diese Dateien über ein sicheres und gut integriertes API erfolgt.

\subsubsection{Supabase Realtime}
Supabase Realtime \cite{SupabaseRealtime} ermöglicht die Echtzeitübertragung von Datenänderungen innerhalb einer Supabase-Datenbank. Es basiert auf PostgreSQL-Listen/Notify und WebSockets und erlaubt es Entwicklern, Anwendungen mit Live-Updates zu erstellen – etwa für Chats, Dashboards oder kollaborative Tools. Durch die enge Integration mit der bestehenden Datenbankinfrastruktur und Authentifizierung bietet Supabase Realtime eine sichere und effiziente Lösung für synchronisierte Datenübertragungen ohne zusätzliche Backend-Logik.

\subsection{Vercel}
Vercel \cite{VercelWiki} ist eine Cloud-Plattform, die sich auf die Bereitstellung und das Hosting moderner Webanwendungen spezialisiert hat. Sie bietet optimierte Deployment-Prozesse und eine hervorragende Performance, insbesondere für Frameworks wie Next.js, und stellt sicher, dass Anwendungen weltweit schnell und zuverlässig zugänglich sind.

\section{Gehäuse}  
Im Kontext der Produktentwicklung bezeichnet "Gehäuse" ein physisches Gehäuse oder eine Hülle, in der elektronische oder mechanische Komponenten untergebracht werden. Solche Gehäuse dienen nicht nur dem Schutz der internen Bauteile vor äußeren Einflüssen, sondern sind auch essenziell für das Design und die Ergonomie des Endprodukts. Die Konstruktion eines Gehäuses umfasst dabei Aspekte wie Materialwahl, Wärmeableitung, Montagepunkte und Benutzerzugänglichkeit, um eine funktionale und ästhetisch ansprechende Lösung zu realisieren.

\subsection{Fusion 360}
\label{ref:fusion360_grundlagen}
Fusion 360 \cite{Fusion360Wiki} ist eine integrierte CAD-, CAM- und CAE-Plattform von Autodesk, die speziell für die Produktentwicklung und mechanische Konstruktion entwickelt wurde. Mit Fusion 360 können Designer und Ingenieure 3D-Modelle erstellen, simulieren und optimieren, um präzise Gehäusedesigns zu entwerfen. Die Software unterstützt den gesamten Entwicklungsprozess – von der Konzeptskizze über die detaillierte Modellierung bis hin zur Fertigungsplanung – und ermöglicht so eine effiziente Umsetzung komplexer Projekte.

\subsection{Slicer}

Ein Slicer \cite{SlicerWiki} ist eine Software, die 3D-Modelle (meist im STL-Format) in druckbare Anweisungen für 3D-Drucker umwandelt. Diese Anweisungen, in Form von G-Code, enthalten Informationen über die Druckpfade, Schichthöhe, Druckgeschwindigkeit und andere Parameter. Der Slicer ermöglicht eine präzise Kontrolle über den Druckprozess und beeinflusst maßgeblich die Qualität und Effizienz des Endprodukts. Bekannte Slicer-Programme sind beispielsweise Cura, PrusaSlicer oder Simplify3D.

\subsection{3D-Drucker}

Ein 3D-Drucker \cite{3DDruckerWiki} ist ein Gerät, das digitale Modelle schichtweise in physische Objekte umwandelt. In dieser Arbeit kommt das sogenannte FDM-Verfahren (Fused Deposition Modeling) zum Einsatz, bei dem ein Kunststofffilament erhitzt und schichtweise aufgetragen wird. 3D-Drucker ermöglichen die schnelle und kostengünstige Herstellung von Prototypen, Gehäusen oder technischen Bauteilen und spielen daher eine zentrale Rolle in der Umsetzung individueller Hardwarelösungen.


\end{inhalt}