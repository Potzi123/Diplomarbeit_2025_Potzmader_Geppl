\begin{inhalt}
\renewcommand*\chapterpagestyle{scrheadings}
\chapter{Backend}

Einführung in Tobias Teil Kurze Erklärung

\section{Komponentenwahl}

\subsection{Sensoren}
Im ersten Schritt werden grundlegende Datenbankeinstellungen vorgenommen, die als „Sensoren“ fungieren und die Umgebung überwachen. Beispiele hierfür sind Befehle, die Zeitüberschreitungen verhindern und die richtige Zeichencodierung sicherstellen:
\begin{lstlisting}[language=SQL, caption=Grundlegende Datenbankeinstellungen]
SET statement_timeout = 0;
SET client_encoding = 'UTF8';
SET standard_conforming_strings = on;
\end{lstlisting}
Diese Einstellungen legen den Grundstein für einen stabilen Betrieb der Datenbank.

\subsection{Mikrocontroller}
Die „Mikrocontroller“ im Backend entsprechen den installierten Erweiterungen, die verschiedene Funktionen steuern und die Performance der Datenbank verbessern. Beispiele hierfür sind:
\begin{lstlisting}[language=SQL, caption=Installation von Erweiterungen]
CREATE EXTENSION IF NOT EXISTS "pgsodium" WITH SCHEMA "pgsodium";
CREATE EXTENSION IF NOT EXISTS "pg_graphql" WITH SCHEMA "graphql";
CREATE EXTENSION IF NOT EXISTS "pg_stat_statements" WITH SCHEMA "extensions";
CREATE EXTENSION IF NOT EXISTS "pgcrypto" WITH SCHEMA "extensions";
CREATE EXTENSION IF NOT EXISTS "pgjwt" WITH SCHEMA "extensions";
CREATE EXTENSION IF NOT EXISTS "supabase_vault" WITH SCHEMA "vault";
CREATE EXTENSION IF NOT EXISTS "uuid-ossp" WITH SCHEMA "extensions";
\end{lstlisting}
Diese Erweiterungen erweitern die Funktionalität, etwa in der Kryptographie oder bei der Implementierung von GraphQL-Schnittstellen.

\subsection{Bedienelemente}
Zu den „Bedienelementen“ zählt die Erstellung der Datenstrukturen, also Tabellen und zugehörige Constraints, die das Rückgrat der Datenbank bilden. Beispielsweise wird die Tabelle \texttt{classes} folgendermaßen erstellt:
\begin{lstlisting}[language=SQL, caption=Erstellung der Tabelle "classes"]
CREATE TABLE IF NOT EXISTS "public"."classes" (
    "id" uuid DEFAULT gen_random_uuid() NOT NULL,
    "created_at" timestamp with time zone DEFAULT now() NOT NULL,
    "name" text,
    "department_id" uuid DEFAULT gen_random_uuid()
);
\end{lstlisting}
Zusätzlich werden eindeutige Schlüssel (Primary Keys) und Fremdschlüsselbeziehungen definiert, um die Datenintegrität zu gewährleisten.

\section{Versorgungen \& Ansteuerungen}

\subsection{Bussysteme}
Die „Bussysteme“ im Backend spiegeln die Kommunikationswege zwischen den einzelnen Tabellen wider – insbesondere die Beziehungen, die über Fremdschlüssel hergestellt werden. Ein Beispiel:
\begin{lstlisting}[language=SQL, caption=Definition eines Fremdschlüssels]
ALTER TABLE ONLY "public"."classes"
    ADD CONSTRAINT "classes_department_id_fkey" FOREIGN KEY ("department_id")
    REFERENCES "public"."departments"("id") ON UPDATE CASCADE ON DELETE CASCADE;
\end{lstlisting}
Dieser Befehl sorgt dafür, dass Änderungen in den referenzierten Tabellen automatisch übernommen werden.

\subsection{Spannungsversorgungen}
Die „Spannungsversorgungen“ symbolisieren die Sicherheits- und Automatisierungselemente des Backends, beispielsweise Triggerfunktionen und die Vergabe von Zugriffsrechten. Ein Beispiel für eine Triggerfunktion, die beim Anlegen eines neuen Benutzers automatisch ein Profil erstellt, lautet:
\begin{lstlisting}[language=SQL, caption=Triggerfunktion handle_new_user()]
CREATE OR REPLACE FUNCTION public.handle_new_user()
 RETURNS trigger
 LANGUAGE plpgsql
 SECURITY DEFINER
 SET search_path TO ''
AS $function$
begin
  insert into public.profiles (id)
  values (new.id);
  return new;
end;
$function$
;
\end{lstlisting}
Zusätzlich werden mittels GRANT-Befehlen Zugriffsrechte für unterschiedliche Rollen definiert, um einen kontrollierten Zugriff auf die Daten zu gewährleisten.

\section{Pin-Layout \& Schematik}
Das „Pin-Layout \& Schematik“ entspricht der Gesamtstruktur der Datenbank. Hier wird das Layout der Tabellen, deren Beziehungen sowie Sicherheitsmechanismen wie Row Level Security dargestellt:
\begin{lstlisting}[language=SQL, caption=Aktivierung von Row Level Security]
ALTER TABLE "public"."classes" ENABLE ROW LEVEL SECURITY;
ALTER TABLE "public"."departments" ENABLE ROW LEVEL SECURITY;
ALTER TABLE "public"."schools" ENABLE ROW LEVEL SECURITY;
\end{lstlisting}
Diese Maßnahmen stellen sicher, dass nur autorisierte Benutzer auf bestimmte Daten zugreifen können.

\section{PCB-Design}
Im „PCB-Design“ wird das finale Zusammenspiel aller Komponenten des Backends zusammengefasst. Dies umfasst abschließend die Konfiguration der Tabellen, Triggerfunktionen und Zugriffsrechte, welche zusammen ein robustes und sicheres System bilden:
\begin{lstlisting}[language=SQL, caption=Vergabe von Zugriffsrechten]
grant select on table "public"."profiles" to "anon";
grant insert on table "public"."profiles" to "authenticated";
grant update on table "public"."profiles" to "service_role";
\end{lstlisting}
Das Endresultat ist ein Backend, das durch präzise Konfiguration und sorgfältige Planung den Anforderungen moderner Anwendungen gerecht wird.

\end{inhalt}
