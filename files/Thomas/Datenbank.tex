\begin{inhalt}
\renewcommand*\chapterpagestyle{scrheadings}
\chapter{Supabase}

\section{Datenbank}

In diesem Abschnitt werden die Struktur, verwendete Tabellen sowie die wichtigen SQL-Befehle, Trigger und Funktionen der Datenbank detailliert erläutert.

\subsection{Tabellen}

Die Datenbank besteht aus den folgenden Tabellen:

\subsubsection{Schule (Schools)}
Die Tabelle speichert Informationen zu den Schulen, einschließlich Standort und Webseite. Der Primärschlüssel (\texttt{id}) wird automatisch generiert.

\begin{lstlisting}[style=mysql]
CREATE TABLE schools (
    id UUID PRIMARY KEY DEFAULT gen_random_uuid(),
    created_at TIMESTAMPTZ DEFAULT now(),
    name TEXT,
    location JSONB,
    website_url TEXT
);
\end{lstlisting}

\subsubsection{Abteilungen (Departments)}
Die Tabelle verwaltet die verschiedenen Abteilungen innerhalb einer Schule. Jede Abteilung ist eindeutig einer Schule zugeordnet.

\begin{lstlisting}[style=mysql]
CREATE TABLE departments (
    id UUID PRIMARY KEY DEFAULT gen_random_uuid(),
    created_at TIMESTAMPTZ DEFAULT now(),
    name TEXT,
    school_id UUID REFERENCES schools(id)
);
\end{lstlisting}

\subsubsection{Klassen (Classes)}
Diese Tabelle speichert Informationen zu den Klassen, die spezifischen Abteilungen angehören.

\begin{lstlisting}[style=mysql]
CREATE TABLE classes (
    id UUID PRIMARY KEY DEFAULT gen_random_uuid(),
    created_at TIMESTAMPTZ DEFAULT now(),
    name TEXT,
    department_id UUID REFERENCES departments(id)
);
\end{lstlisting}

\subsubsection{Geräte (Sensors)}
Die Tabelle enthält Informationen zu den einzelnen Sensor-Geräten, welche jeweils einer Klasse und einem Benutzer zugeordnet sind.

\begin{lstlisting}[style=mysql]
CREATE TABLE sensors (
    id UUID PRIMARY KEY DEFAULT gen_random_uuid(),
    name TEXT,
    class_id UUID REFERENCES classes(id),
    user_id UUID REFERENCES profiles(id),
    created_at TIMESTAMPTZ DEFAULT now()
);
\end{lstlisting}

\subsubsection{Benutzerprofile (Profiles)}
In dieser Tabelle werden Benutzerdaten, Rollen und deren Zugehörigkeit zu Klassen verwaltet. Zudem sind hier Statusinformationen wie \texttt{approved} und \texttt{email\_confirmed\_at} gespeichert.

\begin{lstlisting}[style=mysql]
CREATE TABLE profiles (
    id UUID PRIMARY KEY DEFAULT gen_random_uuid(),
    created_at TIMESTAMPTZ DEFAULT now(),
    first_name TEXT,
    surname TEXT,
    username TEXT,
    role INTEGER,
    class_id UUID REFERENCES classes(id),
    email_confirmed_at TIMESTAMPTZ,
    approved BOOLEAN
);
\end{lstlisting}

\subsection{Trigger und Funktionen}

\subsubsection{handle\_new\_user}
Diese Funktion erstellt automatisch ein Benutzerprofil, sobald ein neuer Benutzer angelegt wird. Dies erleichtert die Verwaltung von Nutzerdaten.

\begin{lstlisting}[style=mysql]
CREATE OR REPLACE FUNCTION public.handle_new_user()
 RETURNS trigger
 LANGUAGE plpgsql
 SECURITY DEFINER
 SET search_path TO ''
AS $function$
begin
  insert into public.profiles (id)
  values (new.id);
  return new;
end;
$function$;
\end{lstlisting}

Der Einsatz dieser Funktion erfolgt typischerweise über einen Trigger auf der Tabelle \texttt{auth.users}, um neue Benutzer automatisiert zu integrieren.

\subsubsection{handle\_updated\_email\_confirmed\_at}
Diese Funktion aktualisiert das Feld \texttt{email\_confirmed\_at} im Profil eines Benutzers, wenn die E-Mail-Adresse bestätigt wird. Dies gewährleistet die Synchronität zwischen Authentifizierungs- und Profiltabellen.

\begin{lstlisting}[style=mysql]
CREATE OR REPLACE FUNCTION public.handle_updated_email_confirmed_at()
 RETURNS trigger
 LANGUAGE plpgsql
 SECURITY DEFINER
 SET search_path TO ''
AS $function$
BEGIN
  IF NEW.email_confirmed_at IS DISTINCT FROM OLD.email_confirmed_at THEN
    UPDATE public.profiles
    SET email_confirmed_at = NEW.email_confirmed_at
    WHERE id = NEW.id;
  END IF;
  RETURN NEW;
END;
$function$;
\end{lstlisting}

Die Implementierung dieser Funktion erfolgt ebenfalls durch einen Trigger, welcher Änderungen in der Tabelle \texttt{auth.users} überwacht.

Durch diese Trigger und Funktionen wird eine automatische, zuverlässige und konsistente Pflege der Benutzerdaten in der Datenbank gewährleistet.


\section{Storage}

\section{Auth}









\end{inhalt}