\begin{inhalt}
\renewcommand*\chapterpagestyle{scrheadings}
\chapter{Website}
\label{ref:Website}

Die Webseite wurde in \texttt{TypeScript} programmiert und verwendet \texttt{Next.js} als Framework.

\vspace{0.15cm}

Zum Starten des Projekts musste folgender Setup-Befehl ausgeführt werden:

\begin{lstlisting}[style=myconsole]
npx create-next-app@latest
\end{lstlisting}

Anschließend wurden die gewünschten Projekt-Einstellungen ausgewählt.  
Für dieses Projekt wurden folgende Optionen festgelegt:

\begin{lstlisting}[style=myconsole]
√ What is your project named? ... classscout
√ Would you like to use TypeScript? ... Yes
√ Would you like to use ESLint? ... Yes
√ Would you like to use Tailwind CSS? ... Yes
√ Would you like your code inside a `src/` directory? ... No
√ Would you like to use App Router? (recommended) ... Yes
√ Would you like to use Turbopack for `next dev`? ... Yes
√ Would you like to customize the import alias (`@/*` by default)? ... Yes
√ What import alias would you like configured? ... @/*
\end{lstlisting}



\newpage

\section{Signup}
\label{ref:Signup}
Für die Zuordnung der jeweiligen Benutzer ist es notwendig, dass sich die Nutzer einloggen können.  
Hierfür wurde eine Registrierkomponente erstellt, die den Registrier-Prozess ermöglicht.


\begin{figure}[!htb]
\centering
\includegraphics[width=0.5\textwidth]{files/Thomas/pics/Website/Signup/sign-up.png}
\caption[Registrierkomponente]{Registrierkomponente}
\label{fig:gehaeuse_internet_bild}
\end{figure}

Zur Validierung der Eingaben in den Feldern für \texttt{E-Mail} und \texttt{Passwort} wird die Bibliothek \textbf{Zod} verwendet. Dabei überprüft \textbf{Zod}, ob das E-Mail-Feld eine syntaktisch gültige E-Mail-Adresse enthält und ob das Passwortfeld einen String mit einer Mindestlänge von sechs Zeichen beinhaltet.


\begin{lstlisting}[style=mytsx]
const formSchema = z.object({
  email: z.string().email({
    message: "Please enter a valid email.",
  }),
  password: z.string().min(6, {
    message: "Password must be at least 6 characters.",
  }),
})
\end{lstlisting}
\index{Code Snippet 1@\hyperref[code:snippet1]{Snippet 1}}

\newpage

Für den Registrierungsvorgang wurde eine Funktion implementiert, die die eingegebene E-Mail-Adresse sowie das Passwort entgegennimmt und diese über eine Funktion des Supabase-Auth-Services im System registriert.


\begin{lstlisting}[style=mytsx]
export async function insert_user(email: string, password: string) {
  const supabase = await createClient();
  const { error } = await supabase.auth.signUp({
    email,
    password,
  });
  return error;
}
\end{lstlisting}

Nachdem der Nutzer erfolgreich registriert wurde, erfolgt eine Weiterleitung auf die Seite \texttt{/checkyouremails}, um ihn darüber zu informieren, dass er seine E-Mails überprüfen soll.

\begin{lstlisting}[style=mytsx]
      router.push("checkyouremails");
\end{lstlisting}

\subsection{Check Your Emails}

Auf dieser Seite wird überprüft, ob im Table \texttt{profiles} (siehe Referenz \ref{ref:profile-table-design}) das Feld \texttt{email\_confirmed\_at} gesetzt wurde und somit nicht mehr den Wert \texttt{null} enthält.  



\begin{figure}[H] 
  \centering
  \begin{subfigure}[b]{0.48\textwidth}
    \centering
    \includegraphics[width=1\textwidth]{files/Thomas/pics/Website/emailconfirmed/checkyouremails_waiting_verification.png}
    \caption{Email - Warte auf Verifikation}
    \label{fig:email_waiting_verification}
  \end{subfigure}
  \hfill
  \begin{subfigure}[b]{0.48\textwidth}
    \centering
    \includegraphics[width=1\textwidth]{files/Thomas/pics/Website/emailconfirmed/checkyouremails_verified.png}
    \caption{Email - Verifiziert}
    \label{fig:email_verified}
  \end{subfigure}
  \caption{Email Verifikation Status}
  \label{fig:email_verification_states}
\end{figure}


Dies geschieht über die Echtzeitfunktionalität von Supabase, welche es ermöglicht, Daten in Echtzeit zu aktualisieren und Änderungen in der Datenbank unmittelbar zu erkennen.

Sobald erkannt wird, dass die E-Mail-Adresse verifiziert wurde, erfolgt eine automatische Weiterleitung auf die \texttt{/login}-Seite.

\begin{lstlisting}[style=mytsx]
      router.push("/login");
\end{lstlisting}     


\section{Login}
\label{ref:Login}
Anschließend muss sich der Benutzer einloggen. Dieser Vorgang ist nahezu identisch mit dem Registrierungsprozess auf der \texttt{Sign-up}-Seite. 

\begin{figure}[!htb]
\centering
\includegraphics[width=0.5\textwidth]{files/Thomas/pics/Website/Login/login.png}
\caption[Loginkomponete]{Loginkomponete}
\label{fig:gehaeuse_internet_bild}
\end{figure}


Allerdings wird anstelle der \texttt{insert\_user}-Funktion zur Registrierung nun die \texttt{login\_user}-Funktion verwendet.
Da durch die E-Mail-Verifizierung keine automatische Benutzer-Session gestartet wird, muss der Benutzer sich manuell einloggen, um eine Session zu starten.

\begin{lstlisting}[style=mytsx]
export async function login_user(email: string, password: string) {
  const supabase = await createClient();
  const { error } = await supabase.auth.signInWithPassword({
    email,
    password,
  });
  return error;
}
\end{lstlisting}

\clearpage

\newpage

\section{Setup Profile}
\label{ref:profile-setup}

Wenn ein Benutzer versucht, auf die \texttt{/setup-profile}-Seite zuzugreifen, und bereits angemeldet ist, wird überprüft, ob im Profil entweder der Benutzername, der Vorname oder der Nachname fehlt.  
Falls einer dieser Werte nicht gesetzt ist, wird der Benutzer automatisch durch die Middleware (siehe \ref{ref:middelware}) auf die entsprechende Seite weitergeleitet.

\vspace{0.25cm}

Die \texttt{/setup-profile}-Seite dient dazu, sicherzustellen, dass jeder Benutzer einen Benutzernamen, einen Vornamen und einen Nachnamen angegeben hat und diese Informationen vollständig vorliegen.

\begin{figure}[!htb]
\centering
\includegraphics[width=0.55\textwidth]{files/Thomas/pics/Website/setup-profile/setup-profile.png}
\caption[Profil Erstellung und Vervollständigung Komponente]{Profil Erstellung und Vervollständigung Komponente}
\label{fig:gehaeuse_internet_bild}
\end{figure}

\clearpage

Diese Seite besteht ebenfalls aus einer Komponente, die ähnlich zur \texttt{Login}- und \texttt{Signup}-Komponente aufgebaut ist (siehe \ref{ref:Signup}, \ref{ref:Login}).

Der wesentliche Unterschied zwischen dem \texttt{Login}-Formular und der \texttt{Setup-Profile}-Seite besteht darin, dass auf der \texttt{Setup-Profile}-Seite zusätzlich ein Avatar hinzugefügt oder gelöscht werden kann.

\begin{lstlisting}[style=mytsx]
<AvatarUpload
    userName={form.getValues("first_name") || "User"}
    userId={userId || undefined}
    onImageChange={handleImageChange}
/>
\end{lstlisting}

\newpage

\subsection{AvatarUpload-Komponente}

Die \texttt{AvatarUpload}-Komponente stellt ein zentrales Element der Benutzerprofilseite (\texttt{/setup-profile}) dar. Sie ermöglicht eine intuitive Verwaltung von Profilbildern. Die Komponente implementiert vier Kernfunktionalitäten:

\vspace{0.75cm}

\textbf{Hochladen eines neuen Avatars mit Live-Vorschau}

\vspace{0.15cm}

Die Komponente ermöglicht das Hochladen von Profilbildern mit sofortiger Vorschau. Der Prozess umfasst:
\begin{itemize}
    \item Dateiauswahl aus dem lokalen Dateisystem
    \item Automatische Validierung (max. 5MB, nur Bildformate)
    \item Sofortige Vorschau ohne direkten Server-Upload
\end{itemize}


\begin{lstlisting}[style=mytsx, label={lst:avatar_upload}]
const handleFileSelect = async (event: React.ChangeEvent<HTMLInputElement>) => {
  const file = event.target.files?.[0];
  if (!file || !validateFile(file)) return;

  try {
    const fileExtension = file.type === "image/jpeg" ? "jpg" : "png";
    const standardizedFile = new File(
      [file], 
      `avatar.${fileExtension}`, 
      { type: file.type }
    );
    const previewUrl = URL.createObjectURL(file);

    setState(prev => ({
      ...prev,
      selectedFile: standardizedFile,
      previewUrl,
      avatarAction: "upload"
    }));
  } catch (error) {
    toast({
      title: "Fehler",
      description: "Datei konnte nicht verarbeitet werden."
    });
  }
};
\end{lstlisting}

\textbf{Entfernen bestehender Avatare}
\vspace{0.15cm}
Ein vorhandenes Profilbild kann entfernt werden:
\begin{itemize}
    \item Visuelles Feedback durch \texttt{X}-Symbol
    \item Sofortige Aktualisierung der Anzeige
    \item Markierung zur Löschung beim nächsten Speichervorgang
\end{itemize}

\begin{lstlisting}[style=mytsx, label={lst:avatar_delete}]
const handleRemoveAvatar = async () => {
  try {
    setState(prev => ({
      ...prev,
      selectedFile: null,
      previewUrl: null,
      avatarAction: "delete"
    }));
    
    window.dispatchEvent(new CustomEvent("avatarChanged", {
      detail: { userId, action: "delete" }
    }));
  } catch (error) {
    toast({
      title: "Fehler",
      description: "Avatar konnte nicht entfernt werden."
    });
  }
};
\end{lstlisting}

\textbf{Intelligente Fallback-Darstellung}

\vspace{0.15cm}

Bei fehlendem Profilbild generiert die Komponente automatisch einen visuellen Platzhalter:
\begin{itemize}
    \item Extraktion der Benutzerinitialen aus dem Namen
    \item Konsistente Darstellung im UI-Design
    \item Automatische Anpassung an verschiedene Namensformate
\end{itemize}

\begin{lstlisting}[style=mytsx, label={lst:avatar_initials}]
const getInitials = (name: string): string => {
  if (!name) return "??";
  const parts = name.split(" ");
  return parts.length === 1
    ? parts[0].substring(0, 2).toUpperCase()
    : (parts[0].charAt(0) + parts[parts.length - 1].charAt(0)).toUpperCase();
};
\end{lstlisting}

\textbf{Echtzeit-Synchronisation}+

\vspace{0.15cm}

Die Komponente nutzt Supabase Realtime für die Live-Synchronisation:
\begin{itemize}
    \item Automatische Aktualisierung bei externen Änderungen
    \item Effiziente Subscription-Verwaltung
    \item Robuste Fehlerbehandlung
\end{itemize}

\begin{lstlisting}[style=mytsx, label={lst:avatar_subscription}]
useEffect(() => {
  if (!userId) return;
  
  let mounted = true;
  const setupSubscription = async () => {
    try {
      const channel = await subscribeToAvatarChanges(userId, () => {
        if (mounted) refreshAvatar();
      });
      if (mounted) channelRef.current = channel;
    } catch (error) {
      console.error("Subscription fehlgeschlagen:", error);
    }
  };

  setupSubscription();
  return () => {
    mounted = false;
    if (channelRef.current) unsubscribe(channelRef.current);
  };
}, [userId]);
\end{lstlisting}

Die \texttt{AvatarUpload}-Komponente vereint somit moderne Webtechnologien mit benutzerfreundlicher Funktionalität und robuster Fehlerbehandlung. Durch die Verwendung von TypeScript und React-Hooks wird eine wartbare und typsichere Implementierung gewährleistet.







\newpage

\section{Middleware}
\label{ref:middelware}

Die Datei \texttt{middleware} ist ein zentraler Bestandteil der Anwendung und wird bei jedem Seitenaufruf ausgeführt.  
\vspace{0.15cm}

Sie übernimmt wichtige Aufgaben im Bereich Sicherheit, Weiterleitung und Zustandskontrolle der Anwendung.  

\vspace{2cm}

\begin{figure}[!htb]
\centering
\includegraphics[width=1\textwidth]{files/Thomas/pics/Website/middelware/middelware.png}
\caption[Ablaufdiagramm von der Middelware]{Ablaufdiagramm von der Middelware}
\label{fig:middleware}
\end{figure}

\newpage

Die Middleware wird automatisch bei jedem Aufruf einer Seite ausgeführt, sofern der aufgerufene Pfad nicht explizit ausgeschlossen wurde.  
Dadurch lassen sich serverseitige Logiken zentral abbilden, wie z.\,B. Authentifizierungsprüfungen, Weiterleitungen oder Cookie-Handling.

\subsection*{Wichtige Funktionen der Middleware}

\begin{enumerate}[label=\textbf{\arabic*.}]
  \item \textbf{Authentifizierung und Autorisierung}
  \begin{itemize}
    \item Überprüfung, ob ein Benutzer eingeloggt ist, mittels \texttt{supabase.auth.getUser()}
    \item Nicht eingeloggte Benutzer werden zur Login-Seite weitergeleitet
    \item Eingeloggte Benutzer werden automatisch zum Dashboard umgeleitet
  \end{itemize}

  \item \textbf{Profilüberprüfung}
  \begin{itemize}
    \item Die Middleware prüft, ob das Benutzerprofil vollständig ist (Vorname, Nachname, Benutzername)
    \item Bei unvollständigem Profil erfolgt eine Weiterleitung zur \texttt{/setup-profile}-Seite
  \end{itemize}

  \item \textbf{Admin-Berechtigungen}
  \begin{itemize}
    \item Für Admin-Routen (\texttt{/admin/*}) wird geprüft, ob der Benutzer die Rolle \texttt{admin} besitzt
    \item Andernfalls erfolgt eine automatische Weiterleitung zum Dashboard
  \end{itemize}

  \item \textbf{Cookie-Management}
  \begin{itemize}
    \item Die Authentifizierung erfolgt mithilfe des Supabase-Cookie-Managements
  \end{itemize}

\newpage

  \item \textbf{Routing-Logik}
  Die Routing-Logik prüft anhand des Authentifizierungsstatus, ob ein Benutzer weitergeleitet werden muss:


\begin{lstlisting}[style=mytsx]
  if (!user) {
    // If user is not logged in
    const isAuthRoute = path === '/login' || 
                        path === '/signup' || 
                        path === '/checkyouremails'
    
    if (!isAuthRoute) {
      // Redirect unauthenticated users to login page
      const redirectUrl = new URL('/login', request.url)
      return NextResponse.redirect(redirectUrl)
    }
  } else {
    // If user is logged in
    const isAuthRoute = path === '/login' || 
                        path === '/signup' || 
                        path === '/'
    
    if (isAuthRoute) {
      // Redirect authenticated users to dashboard
      const redirectUrl = new URL('/dashboard', request.url)
      return NextResponse.redirect(redirectUrl)
    }

  \end{lstlisting}

  \item \textbf{Matcher-Konfiguration}
  Die \texttt{matcher}-Konfiguration legt fest, auf welchen Routen die Middleware ausgeführt werden soll in diesen Falle auf fast alle:

\begin{lstlisting}[language=TypeScript]
matcher: [
  '/((?!_next/static|_next/image|favicon.ico|public/|api/|.
  *\\.(?:svg|png|jpg|jpeg|gif|webp|js|css)$).*)',
]


\end{lstlisting}



\clearpage

\newpage

\section{Seitenleiste}

Die Seitenleiste ist für die Navigation zwischen den verschiedenen Klassen, Abteilungen und Schulen verantwortlich.  

Der Aufbau der Seitenleiste besteht aus den folgenden Komponenten:

\begin{itemize}
    \item school-swticher (SidebarHeader)
    \item nav-departments-classes-devices (SidebarContent)
    \item nav-user (SidebarFooter)
\end{itemize}

\begin{figure}[!htb]
\centering
\includegraphics[width=1\textwidth]{files/Thomas/pics/Website/Sidebar/sidebar-aufbau.png}
\caption[Aufbau der Seitenleiste von SchadCN \cite{SchadCNSeitenleiste}]{Aufbau der Seitenleiste von SchadCN \cite{SchadCNSeitenleiste}}
\label{fig:gehaeuse_internet_bild}
\end{figure}

\newpage

\subsection{School-Switcher}

Die \texttt{school-switcher}-Komponente dient dazu, die aktuelle Schule auszuwählen bzw. zu wechseln.

\begin{figure}[!htb]
\centering
\includegraphics[width=1\textwidth]{files/Thomas/pics/Website/Sidebar/school-switcher/school-switcher.png}
\caption[School-Switcher Komponente geschlossen]{School-Switcher Komponente im geschlossenen Zustand}
\label{fig:school_switcher_closed}
\end{figure}

Administratoren können über einen Klick auf die Komponente ein Auswahlfenster öffnen, in dem sie zwischen den verfügbaren Schulen wählen können.  
Schüler, Lehrer und Direktoren besitzen hingegen keine Berechtigung, die Schule zu ändern.

\begin{figure}[!htb]
\centering
\includegraphics[width=0.75\textwidth]{files/Thomas/pics/Website/Sidebar/school-switcher/school-switcher-open.png}
\caption[School-Switcher geöffnet]{School-Switcher Komponente im geöffneten Zustand}
\label{fig:school_switcher_open}
\end{figure}

\newpage

Die ausgewählte Schule wird anschließend in einem Cookie gespeichert, um beim nächsten Aufruf der Webseite automatisch wiederhergestellt zu werden.  
Im folgenden Codebeispiel wird das Cookie clientseitig unter dem Namen \texttt{activeSchool} gesetzt und speichert die \texttt{UUID} der Schul-ID.

\begin{lstlisting}[language=mytsx]
Cookies.set('activeSchool', school.id, {
    secure: process.env.NODE_ENV === 'production',
    sameSite: 'lax',
    path: '/',
});
\end{lstlisting}

Das Attribut \texttt{secure} sorgt dafür, dass das Cookie nur über HTTPS gesendet wird.  
Da dies jedoch nur in der Produktionsumgebung erforderlich ist, wird die Einstellung über \texttt{process.env.NODE\_ENV} gesteuert.  
Durch die Einstellung \texttt{sameSite = lax} wird das Cookie nur bei normaler Navigation und GET-Anfragen mitgesendet.  
Der Parameter \texttt{path} legt fest, dass das Cookie auf der gesamten Webseite verfügbar ist.


%%% ADD School maby







\newpage

\subsection{nav-departments-classes-devices}


\begin{figure}[!htb]
\centering
\includegraphics[width=0.5\textwidth]{files/Thomas/pics/Website/Sidebar/nav-departments-classes-devices/nav-screen.png}
\caption[Naviagtion in der Seitenleiste für Klassen und Abteilungen und Sensoren]{Naviagtion in der Seitenleiste für Klassen und Abteilungen und Sensoren}
\label{fig:gehaeuse_internet_bild}
\end{figure}

\vspace{1cm}


Die Aufgabe der nav-departments-classes-devices Komponente ist das Man zwischen den Abteilungen Klassen und Sensoren umher Navigiren kann. Damit erkannt wird in welcher Schule man sich Befindet wurde eine Speziele URL Addresse entworfen:

\vspace{1cm}

\begin{lstlisting}[language=myurl, breaklines=true, breakatwhitespace=false, basicstyle=\ttfamily\small, columns=flexible]
domainname/uuid-der-schule/dashboard?department=uuid-in-dem-Department-man-gerade-ist&class=uuid-in-welcher-Klasse-man-sich-gerade-befindet&device=device-uuid
\end{lstlisting}

Beispiel:

\begin{lstlisting}[language=myurl, breaklines=true, breakatwhitespace=false, basicstyle=\ttfamily\small, columns=flexible]
domainname/0d3e37cf-8350-46f2-823b-26dd90c01266/dashboard?department=e1ed2bd8-4c9c-40f8-bec7-4edcebe59fc1&class=eb4355fd-1e98-4961-be62-36fa9fc5cec0&device=a52eb8dc-5df5-475a-8a9f-f5f5a24ac2d3
\end{lstlisting}




%%% gehört vlt noch was hin 


\vspace{-1.75cm}

\subsection{nav-user}

Die \texttt{nav-user}-Komponente befindet sich im \texttt{SidebarFooter} (am unteren Ende der Seitenleiste).  
Sie ist dafür verantwortlich, den Benutzernamen, die E-Mail-Adresse sowie den Avatar des aktuellen Benutzers anzuzeigen.

\begin{figure}[!htb]
\centering
\includegraphics[width=0.5\textwidth]{files/Thomas/pics/Website/Sidebar/nav-user/user-component.png}
\caption[Benutzer-Komponente in der Sidebar]{Benutzer-Komponente in der Seitenleiste geschlossen}
\label{fig:nav_user_component}
\end{figure}

Zusätzlich ermöglicht die Komponente den Zugriff auf den Profil-Editor.  
Für Administratoren werden darüber hinaus Links zu den Administrationsseiten der Webseite angezeigt.  
Außerdem stehen eine Einstellungsseite(\ref{ref:settings}) sowie ein Logout-Button zum Abmelden zur Verfügung.


\begin{figure}[!htb]
\centering
\includegraphics[width=0.5\textwidth]{files/Thomas/pics/Website/Sidebar/nav-user/user-component-open.png}
\caption[Benutzer-Komponente in der Seitenleiste geöffnet]{Benutzer-Komponente in der Seitenleiste geöffnet}
\label{fig:gehaeuse_internet_bild}
\end{figure}



Beim Abmelden von der Seite wird als erstes  gewarted bis die signOut funktion ein success zurück bekommt danach wird der user auf die Login seite weitergeleitet. 

\begin{lstlisting}[language=mytsx]
  const handleSignOut = async () => {
    const { success, error } = await signOut();
    if (success) {
      router.push("/login");
    } else {
      console.error("Error signing out:", error);
    }
  };
\end{lstlisting}


\newpage

\section{Einstellungs Seite}
\label{ref:settings}

\begin{figure}[!htb]
\centering
\includegraphics[width=1\textwidth]{files/Thomas/pics/Website/Settings/settings-component.png}
\caption[Bild von der Einstellungs Seite]{Bild von der Einstellungs Seite}
\label{fig:gehaeuse_internet_bild}
\end{figure}


Die \texttt{Settings}-Seite (Einstellungsseite) hat ausschließlich die Aufgabe, das Theme der Webseite zu wechseln.  
Das ausgewählte Theme wird anschließend in einem Cookie mit dem Namen \texttt{theme} gespeichert, um sicherzustellen, dass beim Aktualisieren der Seite (Refresh) das Theme erhalten bleibt.

\vspace{0.5cm}

Im folgenden Codebeispiel wird ein Schalter (\texttt{Switch}) eingesetzt, um zwischen dem Dark Mode und dem Light Mode zu wechseln.  
Der Schalter besitzt die ID \texttt{dark-mode} und eine Variable, die den aktuellen Zustand (aktiviert/deaktiviert) verwaltet.  
Jedes Mal, wenn der Schalter betätigt wird, wird die Funktion \texttt{toggleTheme()} aufgerufen, welche den Cookie aktualisiert und das Theme der Seite entsprechend anpasst.


\begin{lstlisting}[language=mytsx]
<Switch
    id="dark-mode"
    checked={isDarkMode}
    onCheckedChange={toggleTheme}
/>
\end{lstlisting}

\newpage

\section{Profile Seite}
\label{ref:profil-site}

Die Profil-Seite ist im Grunde die \texttt{Profile-Setup}-Seite \ref{ref:profile-setup} und bietet dem Benutzer die Möglichkeit, das Profilbild, den Vornamen, Nachnamen sowie den Benutzernamen zu aktualisieren.



\begin{figure}[!htb]
\centering
\includegraphics[width=1\textwidth]{files/Thomas/pics/Website/Profile/profile-screen.png}
\caption[Bild von der Profil Seite]{Bild von der Profil Seite}
\label{fig:gehaeuse_internet_bild}
\end{figure}

\newpage

\section{Admin Dashboard}

Das Admin Dashboard ist dafür da um alle User, Geräte, Klassen, Abteilungen und Schulen zu Verwalten. 

\begin{figure}[!htb]
\centering
\includegraphics[width=1\textwidth]{files/Thomas/pics/Website/admin/devices/sensors-admin-screen.png}
\caption[Bild von der Administration Seite]{Bild von der Administration Seite}
\label{fig:gehaeuse_internet_bild}
\end{figure}


In dem Bild kann man sehen wie das Dashboard der Sensoren ausschaut.

\newpage

\subsection{Table}

Jeder Table hat außer die User Table die ist speziel hat am anfag die uuid von den Strukturelementen. Diese Last sich immer ganz einfach kopiern durch einen Button. Danach wird der Name angezeigt und nachdem wird der Zeitpunkt der Erstellung des Strukturelementes angeziegt und am schluss eine Möglichkeit diese Strukturelemente zu löschen. 

\begin{figure}[!htb]
\centering
\includegraphics[width=1\textwidth]{files/Thomas/pics/Website/admin/sensors/sensor-table.png}
\caption[Bild von der Tabelle der Sensoren]{Bild von der Tabelle der Sensoren}
\label{fig:gehaeuse_internet_bild}
\end{figure}

Zum Kopieren des Textes wird der Inhalt zunächst in die Zwischenablage (\texttt{clipboard}) gespeichert mit der funktion \texttt{navigator.clipboard.writeText(value)}. 
Anschließend wird die Variable \texttt{isCopied} auf \texttt{true} gesetzt, wodurch das anzuzeigende Icon entsprechend geändert wird.  
Nach zwei Sekunden wird die Variable wieder auf \texttt{false} zurückgesetzt, um das Icon erneut zu aktualisieren.


\begin{lstlisting}[language=mytsx]
const copy = async () => {
    try {
      await navigator.clipboard.writeText(value);
      setIsCopied(true);
      setTimeout(() => setIsCopied(false), 2000);
    } catch (err) {
      console.error("Failed to copy text: ", err);
    }
  };
\end{lstlisting}
%%% code hinzufügen 

\clearpage

\subsection{Erstellen}

Beim Erstellen von den Verschieden Strukturelementen wurde eine immer die Selbe Struktur genommen. Oben Wird immer der Name eingegeben danch kommen die abhängigen slektoren.

\begin{figure}[!htb]
\centering
\includegraphics[width=0.5\textwidth]{files/Thomas/pics/Website/admin/sensors/sensor-create.png}
\caption[Bild von der Erstell Komponente der Sensoren]{Bild von der Erstell Komponente der Sensoren}
\label{fig:sensor-create}
\end{figure}

Bei den Beispiel für die Sensor-Erstellungs Komponente \ref{fig:sensor-create} kann man sehen das diese die Schule, Abeteilung und die Klasse benötigt da sie von diesen Parametern abhängig ist.

\newpage

\textbf{Schule automatische Vervollständigung}

Für das Erstellen von Schulen wurde eine Auto Vervollständigung von Google bentutz um leichter den Standort der Schulen einzugeben.

\textbf{Validierung der Autocomplete-Eingabe}

In diesem Code-Snippet wird geprüft, ob der eingegebene Text mindestens 3 Zeichen lang ist und der \texttt{autocompleteService} verfügbar ist. Wenn dies der Fall ist, werden über die Google Places API passende Vorschläge abgefragt.

\begin{itemize}
    \item Falls Ergebnisse gefunden wurden, werden diese in den State \texttt{predictions} geladen.
    \item Falls keine Ergebnisse vorhanden sind (\texttt{ZERO\_RESULTS}), wird eine leere Liste angezeigt und eine Fehlermeldung gesetzt.
    \item Bei allen anderen Fehlern wird ebenfalls eine leere Liste angezeigt und eine allgemeine Fehlermeldung angezeigt.
    \item Nach der Verarbeitung wird der Ladezustand auf \texttt{false} gesetzt.
\end{itemize}

\vspace{0.5cm}

\begin{lstlisting}[language=mytsx]
if (inputValue.length > 2 && autocompleteService.current) {
    autocompleteService.current.getPlacePredictions(
        { input: inputValue },
        (predictions, status) => {
            if (status === window.google.maps.places.PlacesServiceStatus.OK && predictions) {
                setPredictions(predictions);
            } else if (status === window.google.maps.places.PlacesServiceStatus.ZERO_RESULTS) {
                setPredictions([]);
                setError("No results found");
            } else {
                setPredictions([]);
                setError("Error fetching predictions");
            }
            setLoading(false);
        },
    );
} else {
    setPredictions([]);
    setLoading(false);
}
\end{lstlisting}

\clearpage

\subsection{Schule Map}


\begin{figure}[!htb]
\centering
\includegraphics[width=0.5\textwidth]{files/Thomas/pics/Website/admin/school/school-map.png}
\caption[Google Maps Karte auf der Webseite]{Google Maps Karte auf der Webseite}
\label{fig:gehaeuse_internet_bild}
\end{figure}

Die \texttt{Schule-Map} ist eine Google Maps Karte, die den Standort der Schulen anzeigen kann.

\vspace{0.5cm}

Für diese Komponente wurde die Bibliothek \texttt{react-google-maps/api} verwendet.  
Sie bietet zahlreiche Einstellungsmöglichkeiten und ist gleichzeitig einfach zu implementieren.

\vspace{0.15cm}

Im folgenden Codeabschnitt wird die Karte konfiguriert.  
Der Standardausschnitt ist so gewählt, dass die Karte auf St. Pölten zentriert und hereingezoomt ist.  
Zudem wurden viele Steuerungselemente (\texttt{Controls}) deaktiviert, da die Benutzer lediglich die Karte zum Zoomen und zur Standortsuche der Schulen nutzen sollen.



\begin{lstlisting}[language=mytsx]
<GoogleMap
        mapContainerStyle={containerStyle}
        center={defaultCenter}
        zoom={defaultZoom}
        options={{
          disableDefaultUI: false,
          zoomControl: true,
          mapTypeControl: false,
          streetViewControl: false,
          fullscreenControl: false,
        }}
      >
\end{lstlisting}





\clearpage

\section{Dashbaord}

Das Dashboard stellt den zentralen Bestandteil dieser Diplomarbeit dar, da es die vom Sensor gemessenen Werte visuell darstellt.


\begin{figure}[!htb]
\centering
\includegraphics[width=1\textwidth]{files/Thomas/pics/Website/dashbord/dashbaord-screen.png}
\vspace{-1.5cm}
\caption[Bild vom fertigen Dashbaord]{Bild vom Fertigen Dashbaord}
\label{fig:gehaeuse_internet_bild}
\end{figure}

\vspace{-0.75cm}

\subsection{DatePicke Komponente}

\begin{figure}[!htb]
\centering
\begin{minipage}[c]{0.4\textwidth}
    \centering
    \vspace*{\fill}
    Für das Auswählen eines bestimmten Zeitraums, in dem die Daten betrachtet werden sollen, wurde eine \texttt{DatePicker} Komponente erstellt.  
    Diese ermöglicht es, flexibel einen beliebigen Zeitraum für die Anzeige der Messdaten einzustellen.
    \vspace*{\fill}
\end{minipage}
\hfill
\begin{minipage}[c]{0.55\textwidth}
    \centering
    \raisebox{0.5cm}{ % <--- shifts the image down to center with the text
        \includegraphics[width=0.5\textwidth]{files/Thomas/pics/Website/dashbord/dashbaord-datepicker.png}
    }
    \caption[Geöffneter DatePicker]{Geöffneter-DatePicker}
    \label{fig:datepicker_geoeffnet}
\end{minipage}
\end{figure}





\begin{figure}[!htb]
\centering
\includegraphics[width=1\textwidth]{files/Thomas/pics/Website/dashbord/dashbaord-datepicker-flowchart.png}
\caption[Ablaufdiagramm von DatePicker Komponente]{Ablaufdiagramm von DatePicker Komponente}
\label{fig:gehaeuse_internet_bild}
\end{figure}


\clearpage

\subsection{Diagramm}

\begin{figure}[!htb]
\centering
\includegraphics[width=1\textwidth]{files/Thomas/pics/Website/dashbord/chart.png}
\caption[Diagramm Dashboard]{Diagramm Dashboard}
\label{fig:Flowchart_Backend}
\end{figure}

Das Diagramm wurde mit den Diagrammen von Shadcn gemacht. Die Werte die angezeigt werden sollten sind.

\begin{itemize}
    \item Temperature

    \item Humidity

    \item CO2

    \item Gas Resistance
\end{itemize}

Da nicht jederzeit alle Werte gleichzeitig angezeigt werden sollen, wurde eine Lösung mit Schaltflächen implementiert, um einzelne Linien flexibel ein- oder auszublenden.
Mithilfe von \texttt{zustand} wird im folgenden Code-Snippet im State gespeichert, welche Linien aktuell sichtbar sind.

\newpage
Zunächst wird geprüft, ob die Schaltfläche \texttt{''all''} gedrückt wurde. In diesem Fall werden entweder alle vier Linien gleichzeitig ein- oder ausgeblendet, je nachdem, ob aktuell bereits alle sichtbar sind.
Wird hingegen eine einzelne Linie ein- oder ausgeblendet, so wird ihr Name entweder aus dem \texttt{activeLines}-Array entfernt oder wieder hinzugefügt.

\begin{lstlisting}[language=mytsx]
toggleLine: (line) => {
    const { activeLines } = get();
    
    if (line === "all") {
      // Alle Linien ein-/ausblenden
      if (activeLines.length === 4) {
        set({ activeLines: [] });
      } else {
        set({ activeLines: ["temperature", "humidity", "co2", "gas_resistance"] });
      }
    } else {
      // Einzelne Linie ein-/ausblenden
      if (activeLines.includes(line)) {
        set({ activeLines: activeLines.filter(l => l !== line) });
      } else {
        set({ activeLines: [...activeLines, line] });
      }
    }
  },
\end{lstlisting}




\newpage


\subsection{Realtime}
\label{ref:realtime-sensor}

Im folgenden Code-Snippet wird eine Realtime-Subscription für einen Sensor eingerichtet.  
Dabei wird ein Channel auf der Tabelle \texttt{sensor\_readings} erstellt, der jedes Mal einen Callback auslöst, sobald sich über \texttt{postgres\_changes} eine Änderung für genau diesen Sensor ereignet.

\begin{lstlisting}[language=mytsx]
export async function subscribe_to_sensor_readings(
  callback: (payload: RealtimePostgresChangesPayload<SensorReadingRow>) => void,
  sensorId: string
): Promise<RealtimeChannel> {
  const supabase = await createClient();
  const channel = supabase
    .channel(`sensor-readings-${sensorId}`)
    .on(
      'postgres_changes',
      {
        event: '*',
        schema: 'public',
        table: 'sensor_readings',
        filter: `sensor_id=eq.${sensorId}`,
      },
      (payload) => callback(payload as RealtimePostgresChangesPayload<SensorReadingRow>)
    )
    .subscribe();
  
  return channel;
}
\end{lstlisting}

\newpage

\subsection{Cards Komponente}

Auf dem Dashboard werden 4 Karten angezeit wo jede den aktulesten wert anzeigt der von dem Sensor gemessen wurde. Dies wurde auch mittels der gelichen Reatlime \ref{ref:realtime-sensor} implementiert.


\begin{figure}[!htb]
\centering
\includegraphics[width=1\textwidth]{files/Thomas/pics/Website/dashbord/dashboard-card.png}
\caption[Bild von den Cards Komponenten wie sie den letzten sensor Wert anzeigen]{Bild von den Cards Komponenten wie sie den letzten sensor Wert anzeigen}
\label{fig:gehaeuse_internet_bild}
\end{figure}


Im zweiten Code-Snippet wird der Channel mithilfe von \texttt{zustand} im State gespeichert.  
Zusätzlich wird hier festgelegt, dass der Callback nur bei einem \texttt{INSERT} oder \texttt{UPDATE} ausgelöst wird.  
Bei einer solchen Änderung wird ein neues Sensor-Reading zur State-Variable hinzugefügt.

\begin{lstlisting}[language=mytsx]
subscribeToReadings: async (sensorId) => {
    try {
      const channel = await subscribe_to_sensor_readings(
        (payload) => {
          console.log('Real-time sensor reading received:', payload);
          
          if (payload.eventType === 'INSERT' || payload.eventType === 'UPDATE') {
            const newReading = payload.new as SensorReadingRow;
            get().addReading(newReading);
          }
        },
        sensorId
      );
      
      set({ realtimeSubscriptions: { ...get().realtimeSubscriptions, readings: channel } });
    } catch (error) {
      console.error('Error setting up real-time subscription:', error);
    }
  },
\end{lstlisting}






\newpage

\section{Backend}

Das Backend kann auf /api/sensor-data mittels einen Post request daten geschickt werden wie in \ref{ref:backend_design} beschreiben. Wie diese Daten verarbeitet werden kann man im Ablaufdiagramm \ref{fig:Flowchart_Backend} sehen.

\begin{figure}[!htb]
\centering
\includegraphics[width=1\textwidth]{files/Thomas/pics/Website/backend/image.png}
\caption[Ablaufdiagramm vom Backend]{Ablaufdiagramm vom Backend}
\label{fig:Flowchart_Backend}
\end{figure}

Durch diesen Abluaf wird sicher gestellt das die Daten Valide sind.

\newpage

In diesen Code Snipet wird geprüft ob der token(uuid des Sensors) Valide ist und wenn nicht wird ein error zurückgeschickt mit dem Statuts 401 und einen Text Unauthorized

\begin{lstlisting}[language=mytsx]
const isValidToken = await validate_sensor_token(token);
        
        if (!isValidToken) {
            return NextResponse.json({ error: 'Unauthorized' }, { status: 401 });
        }
\end{lstlisting}

\vspace{1cm}

In diesen Code am schluss der route wird geprüft ob die sensor daten in die Datenbank gespeichert werden konnten. Wenn dies nicht der Fall ist wird ein Error zürckgeschickt mit dem Status 500 und der Nachricht Failed to save data


\begin{lstlisting}[language=mytsx]
const { success, error } = await insert_sensor_readings(validReadings);

            if (!success) {
                console.error('Error inserting data:', error);
                return NextResponse.json({ error: 'Failed to save data' }, { status: 500 });
            }
\end{lstlisting}




\clearpage





\end{inhalt}