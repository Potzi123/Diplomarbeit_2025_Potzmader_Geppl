\begin{inhalt}
\chapter{Grundlagen \& Methoden}
\renewcommand*\chapterpagestyle{scrheadings}
 \section{Hardwaredesign \& Programmierung}
 
\subsection{Altium Designer}

Altium Designer ist eine kostenpflichtige Software von Autodesk für die Entwicklung und das Design von PCBs (Printed Circuit Boards). Die Software bietet eine komplette Entwicklungsumgebung für die Erstellung von elektronischen Schaltungen. Sie bietet unter anderem Schaltungsdesign, Schaltkreissimulationen und physisches PCB-Design. 
 \cite{AltiumDesignerWiki}

\subsection{I2C}

I2C (Inter-Integrated-Circuit) ist ein synchrones Bussystem basierend auf dem Master-Slave-Prinzip zur Datenübertragung zwischen Peripheriegeräten. Es benutzt eine Daten- sowie eine Taktleitung und wird bei Geräten mit geringem Abstand zueinander eingesetzt. 
 \cite{I2CKommunikation}

\subsection{SPI}

SPI (Serial Peripheral Interface) ist ein synchrones Bussystem basierend auf dem Master-Slave-Prinzip. Es benutzt 4 Leitungen, davon eine Taktleitung, eine Slave-Select-Leitung und 2 Datenleitungen. Durch die 2 Datenleitungen (Master In Slave Out, Master Out Slave In) ist dieses Bussystem vollduplexfähig. \cite{SPI_Kommunikation}

\subsection{HTTPS} \label{sec:HTTPS-Grundlagen}

HTTPS (Hypertext Transfer Protocol Secure) ist ein Protokoll zur Übertragung von Daten zwischen Webservern und Webbrowsern. HTTPS unterscheidet sich von HTTP durch zusätzliche Verschlüsselung mittels SSL/TLS. \cite{HTTPS_Kommunikation}

\subsection{Visual Studio Code} \label{sec:VS-Code}

Visual Studio Code ist ein kostenloser Text-Editor der Firma Microsoft. Er ist sehr umfangreich durch viele Erweiterungsmöglichkeiten und wird für viele verschiedene Anwendungen eingesetzt. Visual Studio Code bietet eine moderne und individuell anpassbare Benutzeroberfläche. \cite{VisualStudioCode}

\subsection{Raspberry PI Pico Erweiterung} \label{sec:PicoExtension}

Die Raspberry Pi Pico Erweiterung ist eine Erweiterung in Visual Studio Code, die das Programmieren von RP2040-Mikrocontroller-Boards in C/C++ vereinfacht. Sie bietet Projektverwaltung sowie Automatisierung typischer Schritte wie das Einrichten des Pico SDK, die Compiler-Einstellungen oder das Laden der Firmware auf das Board. \cite{Raspberry_Pi_Pico_Erweiterung}


\end{inhalt}