%% Dokumentklasse KOMA-Script Report
\documentclass[paper=a4, 12pt]{scrreprt}
%% Encoding UTF8
\usepackage[utf8]{inputenc}
%%Use Source Sans Pro Textstyle
\usepackage[default]{sourcesanspro}
%% 8 Bit Aufloesung der Buchstaben
\usepackage[T1]{fontenc}
%% Seitenraender
%\usepackage[scale=0.72]{geometry}
\usepackage[scale=0.72, twoside, bindingoffset=2mm]{geometry}
%% Spracheinstellungen
\usepackage[english, naustrian]{babel} % your native language must be the last one!!
%% erweiterte Farbenpalette
\usepackage[dvipsnames]{xcolor}
%% Abbildungen
\usepackage{graphicx}
%%Tabelen mit Farbe (cellcolor)
\usepackage{tabulary}
\usepackage{colortbl}
\usepackage{colortbl}
\usepackage{subcaption}
\PassOptionsToPackage{dvipsnames,svgnames,table}{xcolor}
%% Tabellen (erweitert)
\usepackage{tabularx}
%% TikZ + Circuit-TikZ (fuer Schaltungen)
\usepackage[europeanresistors, europeaninductors]{circuitikz}
%% Nuetzliche TikZ Libraries
\usetikzlibrary{arrows, automata, positioning}
%% mathematik
\usepackage{amsmath, amssymb}
%%Formelbeschreibung
\newenvironment{conditions}
  {\par\vspace{\abovedisplayskip}\noindent\begin{tabular}{>{$}l<{$} @{${}...{}$} l}}
  {\end{tabular}\par\vspace{\belowdisplayskip}}
%\usepackage{mathtools}	
%% pdf-einbindung
\usepackage{pdfpages}
%% scource-code einbindung
\usepackage{listings, scrhack} %scrhack vermeidet Umschaltung auf KOMA
% Floats..
\usepackage{courier}
%% euro-symbol
\usepackage{eurosym}
%% landcsape-seiten ermöglichen
\usepackage{lscape}

%% Diplomarbeits-Format
\usepackage{srdpdipa}

%% Abkuerzungsverzeichnis
\usepackage[]{acronym}

%% Todos
\usepackage[]{todonotes}

%% Ganttdiagramme
\usepackage{pgfgantt}

%% Subfigures
\usepackage[lofdepth]{subfig}

%% Set the counter for the 2.1.1 latex how to get 2.1.1.1
\setcounter{secnumdepth}{4}

%% Für Quellenangaben unter Bildern
\newcommand*{\quelle}[1]{\par\raggedleft\footnotesize Quelle:~#1}

% add the rotating package
\usepackage{rotating}

%

\usepackage{subcaption}



% ???
\usepackage{wrapfig}

%% Listings (Code)
\usepackage{listings}
\usepackage{color}




%% TypeScript

\usepackage[utf8]{inputenc}
\usepackage{xcolor}
\usepackage{listings}

% ---------------------------
% TSX Language and Style Setup
% ---------------------------
\lstdefinelanguage{TSX}{
  sensitive=true,
  keywords={
    import, from, class, extends, function, return, if, else, for, while,
    const, let, var, type, interface, implements, private, public, protected,
    static, any, never, boolean, string, number, unknown, readonly, this, new,
    React, JSX
  },
  morecomment=[l]{//},         % single-line comments
  morecomment=[s]{/*}{*/},     % multi-line comments
  morestring=[b]",             % "strings"
  morestring=[b]',             % 'strings'
  alsoletter={:,<,>,/},        % treat these characters as part of words
  % Color angle brackets (and closing tags) in a different color
  moredelim=[s][\color{blue}]{<}{>},
  moredelim=[s][\color{blue}]{</}{>},
}

\lstdefinestyle{mytsx}{
  language=TSX,
  backgroundcolor=\color{white},      % White background
  basicstyle=\ttfamily\small,         % Monospaced, small font
  keywordstyle=\color{blue}\bfseries,
  stringstyle=\color{orange},
  commentstyle=\color{gray}\itshape,
  frame=single,                       % Frame around the code
  numbers=left,                       % Line numbers on the left
  breaklines=true,
  columns=fullflexible
}

% ---------------------------
% JSON Language and Style Setup
% ---------------------------
\lstdefinelanguage{json}{
  basicstyle=\ttfamily\small,
  numbers=none,
  breaklines=true,
  frame=single,
  backgroundcolor=\color{white},
  % Color numbers and punctuation using literate programming:
  literate=
    {0}{{{\color{orange}0}}}{1}
    {1}{{{\color{orange}1}}}{1}
    {2}{{{\color{orange}2}}}{1}
    {3}{{{\color{orange}3}}}{1}
    {4}{{{\color{orange}4}}}{1}
    {5}{{{\color{orange}5}}}{1}
    {6}{{{\color{orange}6}}}{1}
    {7}{{{\color{orange}7}}}{1}
    {8}{{{\color{orange}8}}}{1}
    {9}{{{\color{orange}9}}}{1}
    {:}{{{\color{blue}{:}}}}{1}
    {,}{{{\color{blue}{,}}}}{1}
    {\{}{{{\color{red}{\{}}}}{1}
    {\}}{{{\color{red}{\}}}}}{1}
    {[}{{{\color{red}{[}}}}{1}
    {]}{{{\color{red}{]}}}}{1}
}

\lstdefinestyle{myjson}{
  language=json,
  backgroundcolor=\color{white},
  basicstyle=\ttfamily\small,
  frame=single,
  numbers=left,         % Optional: add line numbers if you want
  breaklines=true,
  columns=fullflexible
}




\lstdefinestyle{mysql}{
  language=SQL,                      % SQL als Sprache auswählen
  backgroundcolor=\color{white},     % Weißer Hintergrund
  basicstyle=\ttfamily\small,        % Monospace-Schrift in kleiner Größe
  keywordstyle=\color{blue}\bfseries, % Schlüsselwörter in Blau und fett
  stringstyle=\color{orange},         % Zeichenketten in Orange
  commentstyle=\color{gray}\itshape,  % Kommentare in Grau und kursiv
  frame=single,                      % Rahmen um den Code
  numbers=left,                      % Zeilennummern links
  breaklines=true,                   % Zeilenumbrüche aktivieren
  columns=fullflexible
}













\definecolor{dkgreen}{rgb}{0,0.6,0}
\definecolor{gray}{rgb}{0.5,0.5,0.5}
\definecolor{mauve}{rgb}{0.58,0,0.82}

\lstset{frame=none,
  language=Python,
  aboveskip=3mm,
  belowskip=3mm,
  showstringspaces=false,
  columns=flexible,
  basicstyle={\small\ttfamily},
  numbers=left,
  numberstyle=\tiny\color{gray},
  keywordstyle=\color{blue},
  commentstyle=\color{dkgreen},
  stringstyle=\color{mauve},
  breaklines=true,
  breakatwhitespace=true,
  tabsize=3
}

\lstset{literate=%
  {Ö}{{\"O}}1
  {Ä}{{\"A}}1
  {Ü}{{\"U}}1
  {ß}{{\ss}}1
  {ü}{{\"u}}1
  {ä}{{\"a}}1
  {ö}{{\"o}}1
}

%% TIKZ Flowcharts
\usepackage{tikz}
\usetikzlibrary{shapes.geometric, arrows}

\tikzstyle{startstop} = [rectangle, rounded corners, minimum width=3cm, minimum height=1cm,text centered, draw=black, fill=green!32]
\tikzstyle{io} = [trapezium, trapezium left angle=70, trapezium right angle=110, minimum width=3cm, minimum height=1cm, text centered, draw=black, fill=red!35]
\tikzstyle{process} = [rectangle, minimum width=3cm, minimum height=1cm, text centered, draw=black, fill=cyan!32]
\tikzstyle{decision} = [diamond, minimum width=3cm, minimum height=1cm, text centered, draw=black, fill=orange!42]

\tikzstyle{arrow} = [thick,->,>=stealth]


%% definitionen =====================================%%
\dataSchool{HTBLuVA St. Pölten}
\dataDepartment{Höhere Lehranstalt für Elektronik und Technische Informatik}
\dataSubdepartment{Ausbildungsschwerpunkte Embedded- \& Wireless Systems}
\dataSession{2024/25}


\title{AirSens Classroom Data Capturing}
\author{Thomas Potzmader \and Tobias Geppl}
\date{4. April 2025}
\place{St. P\"olten}
\professor{Dipl.-Ing. Wolfgang Uriel Kuran \and DI (FH) Johannes Tomitsch}



%%====================================================%%

% Hyperlinks im Dokument
\usepackage[colorlinks=true,
    linkcolor=black,
    citecolor=black,
    bookmarks=true,
    urlcolor=black,
    bookmarksopen=true]{hyperref}

\begin{document}

\frontmatter

%% titelseite ==========================================%%
\maketitle
%%======================================================%%

%% komplett leere seite ================================%%
\newpage\null\thispagestyle{empty}%\newpage
%%======================================================%%

%% eidesstattliche erklärung ===========================%%
\begin{affidavit}
    \unterschrift{Thomas Potzmader}
    \unterschrift{Tobias Geppl}
\end{affidavit}
%%======================================================%%

%% dokumentation (deutsch/englisch) ====================%%
%%\includepdf[pages=-]{doc/pdfs/dokumentation-de.pdf}                 %Dokumentation(PDF) einfügen
%%\includepdf[pages=-]{doc/pdfs/dokumentation-en.pdf}                 %Dokumentation(PDF) einfügen
%%======================================================%%

\cleardoublepage                                              %Seite freilassen


%% diplomantenvorstellung ==============================%%
%%\includepdf[pages=-]{doc/pdfs/Tobias.pdf}                          %Vorstellung(PDF) einfügen
%%\cleardoublepage                                              %Seite freilassen
%%\includepdf[pages=-]{doc/pdfs/Thomas.pdf}                          %Vorstellung(PDF) einfügen
%%======================================================%%

%%\cleardoublepage                                              %Seite freilassen

%% danksagung ==========================================%%
\begin{acknowledgements}
\subsubsection*{Name 1}
Danksagung Inhalt
\\
Danke Danke Danke Danke Danke Danke Danke Danke Danke Danke Danke Danke Danke Danke Danke Danke 
\\                              %Absatz
\subsubsection*{Name 2}
Danksagung Inhalt
\\
Danke Danke Danke Danke Danke Danke Danke Danke Danke Danke Danke Danke Danke Danke 
\end{acknowledgements}
                             
%%======================================================%%

%%\cleardoublepage                                              %Seite freilassen

%% inhaltsverzeichnis ==================================%%
\renewcommand*\chapterpagestyle{scrheadings}
\tableofcontents
%%======================================================%%

%%\cleardoublepage                                              %Seite freilassen

%% HAUPTTEIL ===========================================%%
\responsible{Name 1, Name 2}
\mainmatter

%Chapter 1 - Einführung/Einleitung
\responsible{Thomas Potzmader, Tobias Geppl} 
\begin{inhalt}
\renewcommand*\chapterpagestyle{scrheadings}
\chapter{Einleitung}

Individuelle Zielsetzung und Aufgabenstellung

\section{Ausgangslage}
\section{Zielsetzung}


\end{inhalt}                                     %Einführung(LaTeX File) einfügen

%Chapter 2 - (Grundlagen & Methoden)
\responsible{Thomas Potzmader, Tobias Geppl} 
\begin{inhalt}
\chapter{Grundlagen \& Methoden}
\renewcommand*\chapterpagestyle{scrheadings}

Verwendete Technologien 
\section{Web Entwicklung}

Web-Entwicklung bezeichnet die Erstellung und Gestaltung von Websites und Webanwendungen, die über das Internet zugänglich sind. Dabei werden verschiedene Technologien und Programmiersprachen genutzt, um Inhalte darzustellen, interaktive Funktionen bereitzustellen und Daten zu verarbeiten. Eine Website besteht typischerweise aus dem Frontend, die dem Nutzer präsentiert wird, sowie einen Backend, die im Hintergrund abläuft und beispielsweise Daten verarbeitet oder speichert. Dabei kommen moderne Frameworks und Entwicklungsumgebungen zum Einsatz, um eine effiziente und ansprechende Umsetzung zu ermöglichen. 

\subsection{Frontend} 

Das Frontend umfasst alle Komponenten einer Website oder Webanwendung, die direkt mit dem Nutzer interagieren. Es stellt die visuelle und funktionale Oberfläche bereit, über die Inhalte dargestellt und Aktionen durchgeführt werden können. Typische Technologien im Frontend-Bereich sind HTML, CSS und JavaScript. Moderne Frameworks wie React, Angular oder Vue.js ermöglichen dabei die Erstellung dynamischer und reaktionsschneller Benutzeroberflächen. Das Ziel des Frontends ist es, ein ansprechendes, intuitives und barrierefreies Nutzererlebnis zu gewährleisten. Dabei spielen Aspekte wie responsives Design, Performance und Zugänglichkeit eine wesentliche Rolle.

\subsection{Backend}

Das Backend bildet das Rückgrat einer Webanwendung und arbeitet im Hintergrund, um Daten zu verarbeiten, Geschäftslogik umzusetzen und Verbindungen zu Datenbanken herzustellen. Es ist nicht direkt für den Endnutzer sichtbar, sorgt jedoch dafür, dass alle Funktionen einer Website reibungslos ablaufen. Gängige Programmiersprachen und Frameworks im Backend-Bereich sind unter anderem PHP, Python, Ruby, Java, Node.js oder .NET. Die Aufgaben des Backends umfassen unter anderem Authentifizierung, Autorisierung, API-Entwicklung, Datenverwaltung und Sicherheitsaspekte. Durch die enge Zusammenarbeit mit dem Frontend wird eine nahtlose Integration und effiziente Datenübertragung zwischen Client und Server gewährleistet.

\subsection{Datenbank}

Eine Datenbank bildet das zentrale Element zur Speicherung, Verwaltung und Abfrage von Daten innerhalb einer Webanwendung. Sie ermöglicht es, Informationen strukturiert abzulegen und bei Bedarf effizient abzurufen oder zu aktualisieren. Dabei wird oft zwischen relationalen Datenbanken, wie MySQL, PostgreSQL oder Oracle, und NoSQL-Datenbanken, wie MongoDB oder Cassandra, unterschieden. Relationale Datenbanken nutzen Tabellen und vordefinierte Beziehungen, um Daten zu organisieren, während NoSQL-Datenbanken flexiblere Datenmodelle bieten, die insbesondere bei großen, unstrukturierten Datenmengen Vorteile bieten können. Durch die enge Integration der Datenbank mit dem Backend wird sichergestellt, dass die Webanwendung zuverlässig und performant auf die benötigten Daten zugreifen kann.

\subsection{Next.js}
Next.js ist ein modernes Framework für die Entwicklung von React-Anwendungen, das sowohl serverseitiges Rendering als auch statische Seitengenerierung unterstützt. Durch diese Funktionen können Entwickler performante und SEO-freundliche Webanwendungen erstellen. Next.js vereinfacht die Handhabung von Routing, API-Routen und anderen komplexen Aufgaben, indem es eine klare und strukturierte Entwicklungsumgebung bietet. Dies führt zu einer verbesserten Entwicklererfahrung und ermöglicht die effiziente Erstellung skalierbarer Webprojekte.

\subsubsection{React}
React ist eine JavaScript-Bibliothek zur Erstellung von Benutzeroberflächen, die es Entwicklern ermöglicht, wiederverwendbare UI-Komponenten zu erstellen. Es bildet die Grundlage für Next.js und konzentriert sich darauf, den Zustand und die Darstellung von Komponenten effizient zu verwalten. Durch das deklarative Programmiermodell wird die Entwicklung von interaktiven Anwendungen vereinfacht, während gleichzeitig eine hohe Performance und Skalierbarkeit gewährleistet werden kann.

\subsection{Typescript}
Typescript erweitert JavaScript um statische Typisierung und andere Features, die die Codequalität und Wartbarkeit von Anwendungen verbessern. Durch die frühzeitige Erkennung von Fehlern im Code und die Unterstützung moderner JavaScript-Funktionalitäten bietet Typescript eine solide Basis für die Entwicklung von robusten und fehlerarmen Anwendungen. Viele moderne Frameworks, darunter auch Next.js, profitieren von den zusätzlichen Sicherheitsmechanismen und der besseren Entwicklerunterstützung, die Typescript bietet.

\subsection{Tailwind}
Tailwind ist ein Utility-first CSS-Framework, das es ermöglicht, direkt im Markup stilisierte Komponenten zu erstellen. Anstatt vordefinierte Komponenten zu nutzen, bietet Tailwind eine umfangreiche Sammlung an Klassen, die individuell kombiniert werden können, um maßgeschneiderte Designs zu realisieren. Dies führt zu einem flexiblen und effizienten Styling-Prozess, bei dem Entwickler schnell und ohne umfangreiche CSS-Dateien arbeiten können. Tailwind unterstützt dabei die Erstellung von responsiven und modernen Benutzeroberflächen, die sich leicht an unterschiedliche Designanforderungen anpassen lassen.

\subsection{ShadCN}
ShadCN ist eine moderne UI-Komponentenbibliothek, die speziell für die Erstellung ansprechender und konsistenter Benutzeroberflächen entwickelt wurde. Sie integriert sich nahtlos in moderne Frontend-Frameworks und bietet eine Vielzahl von wiederverwendbaren Komponenten, die den Entwicklungsprozess beschleunigen und die Wartbarkeit der Anwendungen verbessern.

\subsection{Zod}
\label{subsec:Zod}
Zod ist eine TypeScript-orientierte Validierungsbibliothek, die es ermöglicht, Datenstrukturen präzise zu definieren und zur Laufzeit zu überprüfen. Durch die Nutzung von Zod können Entwickler sicherstellen, dass ihre Anwendungen robust gegenüber fehlerhaften oder unerwarteten Daten sind, was die Zuverlässigkeit und Wartbarkeit des Codes erheblich verbessert.

\subsection{Supabase}
Supabase ist eine umfassende Backend-as-a-Service-Plattform, die Entwicklern eine Alternative zu traditionellen Backend-Lösungen bietet. Sie kombiniert mehrere Funktionen, die eine schnelle und sichere Entwicklung von Webanwendungen ermöglichen.

\subsubsection{Supabase DB}
Die Supabase DB basiert auf PostgreSQL und bietet eine skalierbare und leistungsfähige relationale Datenbanklösung. Sie ermöglicht die Speicherung, Abfrage und Verwaltung von Daten in Echtzeit und ist dabei vollständig in die Supabase-Plattform integriert, um eine reibungslose Datenverwaltung zu gewährleisten.

\subsubsection{Supabase Auth}
Supabase Auth stellt eine vollständige Authentifizierungs- und Autorisierungslösung bereit. Es unterstützt verschiedene Authentifizierungsmethoden, wie E-Mail/Passwort, OAuth und weitere, und ermöglicht so eine einfache Integration sicherer Login-Mechanismen in Webanwendungen.

\subsubsection{Supabase Storage}
Supabase Storage bietet eine skalierbare Lösung zur Speicherung und Verwaltung von Dateien. Es ermöglicht Entwicklern, Medien und Dokumente effizient zu speichern, abzurufen und zu verwalten, wobei der Zugriff auf diese Dateien über ein sicheres und gut integriertes API erfolgt.

\subsection{Vercel}
Vercel ist eine Cloud-Plattform, die sich auf die Bereitstellung und das Hosting moderner Webanwendungen spezialisiert hat. Sie bietet optimierte Deployment-Prozesse und eine hervorragende Performance, insbesondere für Frameworks wie Next.js, und stellt sicher, dass Anwendungen weltweit schnell und zuverlässig zugänglich sind.

\section{Gehäuse}  
Im Kontext der Produktentwicklung bezeichnet "Gehäuse" ein physisches Gehäuse oder eine Hülle, in der elektronische oder mechanische Komponenten untergebracht werden. Solche Gehäuse dienen nicht nur dem Schutz der internen Bauteile vor äußeren Einflüssen, sondern sind auch essenziell für das Design und die Ergonomie des Endprodukts. Die Konstruktion eines Gehäuses umfasst dabei Aspekte wie Materialwahl, Wärmeableitung, Montagepunkte und Benutzerzugänglichkeit, um eine funktionale und ästhetisch ansprechende Lösung zu realisieren.

\subsection{Fusion 360}
\label{ref:fusion360_grundlagen}
Fusion 360 ist eine integrierte CAD-, CAM- und CAE-Plattform von Autodesk, die speziell für die Produktentwicklung und mechanische Konstruktion entwickelt wurde. Mit Fusion 360 können Designer und Ingenieure 3D-Modelle erstellen, simulieren und optimieren, um präzise Gehäusedesigns zu entwerfen. Die Software unterstützt den gesamten Entwicklungsprozess – von der Konzeptskizze über die detaillierte Modellierung bis hin zur Fertigungsplanung – und ermöglicht so eine effiziente Umsetzung komplexer Projekte.

\subsection{Slicer}

\subsection{3D Drucker}


Hier sind einige mögliche Erweiterungen für deine Gliederung:  

### **Weitere mögliche Sektionen und Untersektionen**  

#### **Zusätzliche Hauptsektionen:**  
- **Materialwahl**  
  - Eigenschaften verschiedener Kunststoffe (PLA, ABS, PETG, etc.)  
  - Nachhaltige Materialien  
  - Mechanische Belastbarkeit  

- **Designprinzipien für Gehäuse**  
  - Thermische Überlegungen (Lüftungsschlitze, Kühlkörper)  
  - Ergonomische Gestaltung  
  - Befestigungsmöglichkeiten (Schrauben, Clips, Magnete)  

- **Nachbearbeitung & Veredelung**  
  - Schleifen und Polieren  
  - Lackieren und Beschichten  
  - Kleben und Montieren  

- **Praxistipps & Fehlervermeidung**  
  - Häufige Designfehler  
  - Optimierung für den 3D-Druck  
  - Wartung und Pflege des Druckers  

---

#### **Mögliche Untersektionen innerhalb deiner bestehenden Kapitel:**  

##### **Fusion 360**  
- **Grundfunktionen & Skizzieren**  
- **Parametrisches Design**  
- **Erstellung von Gehäuseteilen**  
- **Exportformate für den 3D-Druck**  

##### **Slicer**  
- **Optimale Einstellungen für Gehäuse**  
- **Stützstrukturen und Überhänge**  
- **Layerhöhe & Druckqualität**  
- **Slicing-Software im Vergleich (PrusaSlicer, Cura, Bambu Studio)**  

##### **3D Drucker**  
- **Druckertypen und ihre Vor- und Nachteile**  
  - FDM vs. SLA vs. SLS  
- **Druckparameter für Gehäuse**  
  - Wandstärke & Füllmuster  
  - Temperatur- und Kühlungseinstellungen  
- **Wartung & Kalibrierung**  
  - Nozzle & Hotend  
  - Bett-Leveling  
  - Filamentlagerung  

Das könnte deine Dokumentation strukturierter und informativer machen. Willst du noch eine spezifischere Erweiterung?


\end{inhalt}   
\begin{inhalt}
\renewcommand*\chapterpagestyle{scrheadings}
 \section{Hardware \& Mikrocontroller??????}
\subsection{Altium Designer}
\subsection{I2C}
\subsection{SPI}
\subsection{HTTPS}
\subsection{Raspberry PI Pico extension}




\end{inhalt}  %Grundlagen(LaTeX File) einfügen

%Chapter 3 - (Design & Konzept)
\responsible{Thomas Potzmader, Tobias Geppl} 
\begin{inhalt}
\chapter{Design \& Konzept}
\renewcommand*\chapterpagestyle{scrheadings}

\section{Website}

Für Das Bauen einer Website muss sich zuerst gefragt werden mit was man eigentlich eine Website baut. Dafür gibt es die Verschiedenste Möglichkeiten. In disen Fall wurde NextJS genommen da es eines der Meist benutzen Frameworks \cite{SurveyStackOverflow} für Javascript ist.

\vspace{1cm}

\textbf{Frontend}

Der Großteil des Frontends wird zunächst in Obsidian mithilfe des Excalidraw-Plugins entworfen, um zu veranschaulichen, wie eine Seite aussehen könnte.

\newpage
\subsection{Dashboard}

Im Rahmen dieser Diplomarbeit wurde das Dashboard unter Berücksichtigung moderner Designansätze entwickelt. Es orientiert sich an dem Beispiel auf der Shadcn-Website \cite{ShadCNDashboard}, wobei die Gestaltung den spezifischen Anforderungen unseres Projekts angepasst wurde.

\begin{figure}[!htb] 
\centering 
\includegraphics[width=0.75\textwidth]{files/Thomas/pics/Design-Grundlagen/Frontend/Dashboard/dashboard.png} 
\caption[Bildbezeichnung für Abbildungsverzeichnis]{Shadcn Beispiel Dashboard} 
\label{fig:gehaeuse_internet_bild} 
\end{figure}

\begin{figure}[!htb] 
\centering 
\includegraphics[width=0.7\textwidth]{files/Thomas/pics/Design-Grundlagen/Frontend/Dashboard/Dashboard-Draw.png} 
\caption[Bildbezeichnung für Abbildungsverzeichnis]{Umdesignte Dashboard} 
\label{fig:gehaeuse_internet_bild} 
\end{figure}

\newpage

\subsection{Administrationsseiten}
\label{ref:Administrationsseiten}

Für die Entwicklung der Administrationsseiten wurde ein Konzept erarbeitet, das eine strukturierte und benutzerfreundliche Verwaltung der unterschiedlichen Datensätze ermöglicht. Auf der linken Seite wird eine Tabelle angezeigt, in der alle relevanten Einträge übersichtlich dargestellt werden. Auf der rechten Seite befindet sich eine Komponente zur Erstellung neuer Datensätze, welche die jeweiligen Tabellen ergänzt. Das System umfasst dabei separate Tabellen für Schulen, Abteilungen, Klassen, Sensoren und Benutzer, um eine differenzierte und effiziente Datenverwaltung zu gewährleisten. Zusätzlich erfolgt auf der Schul-Administrationsseite eine Visualisierung der Schulstandorte mittels einer Kartenansicht, um die räumliche Verteilung der Schulen anschaulich darzustellen.


\begin{figure}[!htb] 
\centering 
\includegraphics[width=1\textwidth]{files/Thomas/pics/Design-Grundlagen/Frontend/AdminSeite/AdminSeite-Draw.png} 
\caption[Bildbezeichnung für Abbildungsverzeichnis]{Beispiel einer Admin-Tabelle} 
\label{fig:gehaeuse_internet_bild} 
\end{figure}

\newpage

\subsection{Seitenleiste}

Die Seitenleiste \cite{ShadCNSidebar} wurde aufgrund ihrer optimalen Form ausgewählt, um den Benutzeraccount übersichtlich darzustellen. Zudem ermöglicht sie es Administratoren, den aktuell ausgewählten Schulstandort zu wechseln, wodurch eine bessere Übersicht über die laufenden Prozesse erzielt wird. Unterhalb dieses Bereichs werden sämtliche Abteilungen mitsamt den zugehörigen Klassen und Sensoren angezeigt. Die Sichtbarkeit einzelner Abteilungen richtet sich nach der jeweiligen Benutzerrolle: Während Administratoren sowie Schulleiter (Direktoren) alle Abteilungen mit den entsprechenden Klassen und Geräten einsehen können, erhalten Lehrkräfte ausschließlich Zugriff auf die Abteilung, in der sie tätig sind, und Schülerinnen sowie Schüler sehen lediglich ihre eigene Klasse.


%\begin{figure}[H]
%\centering
%\includegraphics[width=0.3\textwidth]{files/Thomas/pics/Design-Grundlagen/Frontend/Sidebar/sidebar.png}
%\caption[Bildbezeichnung für Abbildungsverzeichnis]{Beispiel einer Admin-Tabelle}
%\label{fig:gehaeuse_internet_bild}
%\end{figure}

\begin{figure}[!htb] 
\centering 
\includegraphics[width=0.75\textwidth]{files/Thomas/pics/Design-Grundlagen/Frontend/Sidebar/sidebar-adminvsnotadmin.png} 
\caption[Bildbezeichnung für Abbildungsverzeichnis]{Seitenleiste Design Vergleich} 
\label{fig:gehaeuse_internet_bild} 
\end{figure}


Wird ein Sensor ausgewählt, gelangt der Benutzer auf das zugehörige Dashboard, welches detaillierte Informationen zu dem jeweiligen Sensor, der Klasse oder der Abteilung bereitstellt. Im unteren Bereich der Seitenleiste befindet sich zudem das Profilbild, ergänzt durch den Benutzernamen und die E-Mail-Adresse. Beim Anklicken dieses Elements öffnet sich eine Auswahlliste mit weiteren Optionen. Diese umfasst zum einen die Möglichkeit, über eine dedizierte Account-Seite den Account zu verwalten – beispielsweise den Namen, das Profilbild und den Benutzernamen zu ändern – zum anderen den Zugriff auf die in Abschnitt \ref{ref:Administrationsseiten} erläuterten Administrationsseiten sowie eine Einstellungsseite. Auf dieser können unter anderem persönliche Schlüssel für das zugeordnete Gerät sowie weitere Funktionen eingesehen werden. Abschließend steht ein Logout-Button zur Verfügung, der den Benutzer aus dem System abmeldet und zur Login-Seite zurückführt.

\begin{figure}[!htb] 
\centering 
\includegraphics[width=0.7\textwidth]{files/Thomas/pics/Design-Grundlagen/Frontend/Sidebar/sidebar-nav-user-adminvsnotadmin.png} 
\caption[Bildbezeichnung für Abbildungsverzeichnis]{Benutzer Navigations Vergleich} 
\label{fig:gehaeuse_internet_bild} 
\end{figure}

\newpage

































\subsection{Backend}
\label{ref:backend_design}

Im Folgenden wird das Design des Backends erläutert. Das System ist in der Lage, verschiedene Messwerte zu verarbeiten, darunter CO\textsubscript{2}-Konzentrationen, Feuchtigkeitswerte, Temperaturwerte, Gaswiderstandswerte sowie einen Zeitstempel. Diese Daten werden im JSON-Format vom Backend empfangen. Zur Sicherstellung einer einheitlichen Datenstruktur wurde das folgende JSON-Schema definiert:

\begin{lstlisting}[style=myjson]
{
    "measured_at": "2024-10-23T12:29:02.379+00:00",
    "co2": 6,
    "hum": 7,
    "temp": 8,
    "gasres": 9
}
\end{lstlisting}

\vspace{3cm}

Zur eindeutigen Identifizierung der einzelnen Sensoren wird beim Anlegen eines Sensors auf der Administrationsseite stets eine Sensor-ID generiert. Diese dient als einzigartiger Identifikator, der es ermöglicht, die Messdaten den entsprechenden Sensoren zuzuordnen. Dementsprechend wird die Sensor-ID in den JSON-Daten mitübertragen:

\begin{lstlisting}[style=myjson]
{
    "token": "d192f90b-a5b8-4767-b5af-59ec40fe03c2",
    "measured_at": "2024-10-23T12:29:02.379+00:00",
    "co2": 6,
    "hum": 7,
    "temp": 8,
    "gasres": 9
}
\end{lstlisting}

\newpage

Ein Problem tritt auf, wenn mehrere Messdaten gleichzeitig versendet werden sollen, da mehr Messpunkte erfasst werden, als in einer einzelnen Übertragung enthalten sein können. Zur Lösung dieses Problems wurde ein erweitertes JSON-Format entwickelt, das die Übermittlung mehrerer Messdatensätze in einer einzigen Nachricht ermöglicht:

\begin{lstlisting}[style=myjson]
{
    "token": "d192f90b-a5b8-4767-b5af-59ec40fe03c2",
    "data": [
        {
            "measured_at": "2024-10-23T12:28:02.379+00:00",
            "co2": 3,
            "hum": 4,
            "temp": 5,
            "gasres": 6
        },
        {
            "measured_at": "2024-10-23T12:29:02.379+00:00",
            "co2": 6,
            "hum": 7,
            "temp": 8,
            "gasres": 9
        },
        {
            "measured_at": "2024-10-23T12:30:02.379+00:00",
            "co2": 9,
            "hum": 10,
            "temp": 11,
            "gasres": 12
        }
    ]
}
\end{lstlisting}

Mit diesem erweiterten Datenformat ist das Backend in der Lage, die empfangenen Messdaten effizient in der Datenbank zu speichern.

\newpage


\section{Datenbank}

\label{ref:profile-table-design}
Bei der Planung der Datenbankstruktur wurde zunächst festgestellt, dass Supabase zwar über die integrierte Authentifizierungstabelle theoretisch benutzerspezifische Daten abspeichern könnte, dies jedoch nicht vorgesehen ist. Aus diesem Grund wurde entschieden, eine eigene Tabelle für Benutzer anzulegen.

\begin{figure}[!htb]
  \centering
  \includegraphics[scale=0.5]{files/Thomas/pics/Datenbank_Design/profiles.png}
  \caption[Benutzertabelle der Datenbank]{Tabelle zur Speicherung der Benutzerdaten}
  \label{fig:profiles_tabelle}
\end{figure}

Im nächsten Schritt musste eine sinnvolle Struktur für die Verwaltung der Sensoren entworfen werden. Da das System nicht ausschließlich für die HTL St. Pölten vorgesehen ist, sondern auch von anderen Bildungseinrichtungen genutzt werden könnte, wurde die Datenbank entsprechend erweitert. Es wurde eine Tabelle \textit{Schools} erstellt, um mehrere Schulen abbilden zu können.

\begin{figure}[!htb]
  \centering
  \includegraphics[scale=0.5]{files/Thomas/pics/Datenbank_Design/school.png}
  \caption[Schultabelle der Datenbank]{Schultabelle}
  \label{fig:schools_tabelle}
\end{figure}

\newpage

Da eine Schule in verschiedene Abteilungen untergliedert ist, wurden zusätzlich die Tabellen \textit{Departments} (Abteilungen) sowie \textit{Classes} (Klassen) angelegt. Diese Struktur ermöglicht eine klare Zuordnung und Verwaltung der Sensoren auf Klassen- bzw. Abteilungsebene innerhalb der jeweiligen Schule.

\begin{figure}[!htb]
  \centering
  \begin{subfigure}[b]{0.45\textwidth}
    \centering
    \includegraphics[scale=0.45]{files/Thomas/pics/Datenbank_Design/departments.png}
    \caption[Tabelle Departments]{Abteilungenstabelle}
    \label{fig:departments_tabelle}
  \end{subfigure}
  \hfill
  \begin{subfigure}[b]{0.45\textwidth}
    \centering
    \includegraphics[scale=0.45]{files/Thomas/pics/Datenbank_Design/classes.png}
    \caption[Tabelle Classes]{Klassentabelle}
    \label{fig:classes_tabelle}
  \end{subfigure}
  \caption[Abteilungs- und Klassentabellen]{Datenbankstruktur von Klassen und Abteilungen}
  \label{fig:departments_classes}
\end{figure}

Für die eigentlichen Sensordaten wurden zwei weitere Tabellen erstellt: \textit{Sensors} und \textit{Sensor\_Readings}. Die Trennung dieser Informationen ist notwendig, da bei einer Speicherung der Messwerte direkt in der \textit{Sensors}-Tabelle lediglich die aktuellen Werte abrufbar wären. Da jedoch geplant ist, die historischen Messwerte in Form von Zeitdiagrammen darzustellen, müssen diese dauerhaft gespeichert werden. Aus diesem Grund enthält die Tabelle \textit{Sensor\_Readings} zeitlich zuordenbare Einzelmessungen, welche sich eindeutig einem Sensor zuordnen lassen.

\begin{figure}[!htb]
  \centering
  \begin{subfigure}[b]{0.45\textwidth}
    \centering
    \includegraphics[scale=0.45]{files/Thomas/pics/Datenbank_Design/sensors.png}
    \caption[Tabelle Sensors]{Sensorentabelle}
    \label{fig:sensors_tabelle}
  \end{subfigure}
  \hfill
  \begin{subfigure}[b]{0.45\textwidth}
    \centering
    \includegraphics[scale=0.45]{files/Thomas/pics/Datenbank_Design/sensor_readings.png}
    \caption[Tabelle Sensor Readings]{Messwerttabelle}
    \label{fig:sensor_readings_tabelle}
  \end{subfigure}
  \caption[Sensor- und Messwerttabellen]{Datenbankstruktur von Sensoren und Messwerten}
  \label{fig:sensors_sensor_readings}
\end{figure}

\newpage

\begin{sidewaysfigure}[!htb]
  \centering
  \includegraphics[scale=0.45]{files/Thomas/pics/output.png}
  \caption[Gesamtübersicht der Datenbankstruktur]{Gesamtdiagramm der Datenbankstruktur mit allen relevanten Tabellen und Verknüpfungen.}
  \label{fig:db_gesamt}
\end{sidewaysfigure}

\clearpage
\newpage

\section{Gehäuse}

Zunächst wurde eine umfassende Recherche im Internet durchgeführt, um bestehende Designvarianten zu identifizieren. Dabei wurde ein Design entdeckt, das den Anforderungen entsprach \cite{TemuGehaeuseURL}(\ref{fig:gehaeuse_internet_bild}).

\vspace{0.15cm}

Um eine praktikable Lösung zu entwickeln, wurde ein alternativer Ansatz verfolgt: Es sollte ein Gehäuse konzipiert werden, das sich leicht auseinandernehmen lässt. Ergänzende Recherchen auf YouTube \cite{GehaeuseYoutubeClipsURL} ergaben, dass der Einsatz von Klammern als Verbindungselemente eine Möglichkeit darstellt, das Gehäuse sicher zusammenzuhalten und gleichzeitig eine einfache Demontage zu gewährleisten.

\vspace{0.15cm}

Diese Methode bietet den Vorteil, dass das Gehäuse nicht nur robust zusammengebaut ist, sondern auch bei Bedarf zügig geöffnet werden kann, um Wartungsarbeiten durchzuführen oder interne Komponenten auszutauschen.

\begin{figure}[!htb]
\centering
\includegraphics[width=0.75\textwidth]{files/Thomas/pics/new/Temu-removebg-preview.png}
\caption[Bildbezeichnung für Abbildungsverzeichnis]{}
\label{fig:gehaeuse_internet_bild}
\end{figure}



\end{inhalt}   
\begin{inhalt}
\renewcommand*\chapterpagestyle{scrheadings}
\section{Hardware}
\subsection{Mikrocontroller}
\subsection{Sensoren}
\section{Display}
\section{Benutzer Interaktion}

\end{inhalt}  %Methoden(LaTeX File) einfügen

%Chapter 4 Ergebnisse
\responsible{Thomas Potzmader, Tobias Geppl} 
\begin{inhalt}
\renewcommand*\chapterpagestyle{scrheadings}
\chapter{Ergebnisse}






\end{inhalt}                                      %Ergebnisse(LaTeX File) einfügen


%Chapter 5 Hardwareentwicklung
\responsible{Tobias Geppl} 
\begin{inhalt}
\renewcommand*\chapterpagestyle{scrheadings}


\chapter{Hardwareentwicklung}

In diesem Kapitel geht es um die Entwicklung sowie den Aufbau der Hardware des Messgeräts. Dazu wurde eine Platine entworfen, um alle Komponenten miteinander zu verbinden und diese in ein dafür 3D-gedrucktes Gehäuse zu integrieren.
Das PCB wurde im 2-Layer-Design entworfen, da es für diese Anwendung ausreichend war.
Auf der Ober- und Unterseite des PCBs wurden jeweils Masseflächen angelegt, die mithilfe von Via-Stitching miteinander verbunden sind, um eine gute Verbindung herzustellen.
Die SMD-Bauteile befinden sich nur auf der Unterseite der Platine, um beidseitiges SMD-Löten zu vermeiden. Die SMD-Widerstände und die SMD-Kondensatoren sind im 1206-Format gewählt. Die Spule wurde im 3232-Format gewählt. Durch diese Größen lassen sich die SMD-Komponenten auch leicht per Hand mit dem Lötkolben auflöten.
\bigskip \\
Als Leiterbahnbreite wurde 0,7mm gewählt. Der maximale Strom einer solchen Leitung mit Außenlage auf dem PCB, entspricht ca. 1,85A, was dem vom Gerät maximal benötigtem Strom (Tab. \ref{tab:Stromverbrauch}) leicht standhält. Berechnet wurde dies mit der offiziellen Formel nach IPC-2221 \cite{IPC_2221}:
\smallskip

\begin{center}

\noindent\textbf{Gegebene Werte: }

\text{Leiterbahnbreite} = 0{,}7\,\text{mm} = 27{,}56\,\text{mil}$ 

 $\text{Kupferdicke} = 35\,\mu\text{m} = 1{,}38\,\text{mil}$ 
 
 $\Delta T = 10\,^\circ\text{C}$

\medskip

\noindent\textbf{Querschnittsfläche: } 

$A = 27{,}56 \cdot 1{,}38 = 38{,}02\,\text{mil}^2$

\medskip

\noindent\textbf{Formel nach IPC-2221 (Außenlage): }

I = 0{,}048 \cdot (10)^{0{,}44} \cdot (38{,}02)^{0{,}725} \approx 1{,}85\,\text{A}
    
\end{center}




\section{Pin Zuordnungen} \label{sec:Pin_Zuordnungen}

\textbf{Raspberry Pico W - Pinout:}

\begin{figure}[!htb]
\centering
\includegraphics[width=0.90\textwidth]{files/Tobias/pics/Pinout/pico2w-pinout.pdf}
\caption[Raspberry Pico W - Pinout]{Raspberry Pico W - Pinout}
\label{fig:PicoW_Pinout}
\end{figure}




Um ein möglichst sauberes Design zu erreichen, wurde die Pin-Zuordnung des Mikrocontrollers folgendermaßen gewählt:

\renewcommand{\arraystretch}{1}

\begin{table}[H]
\centering
\rowcolors{2}{white}{white}
\begin{tabular}{|l|c|}
\hline
\rowcolor{cyan!20}
\textbf{GPIO-Pin} & \textbf{Funktion} \\
\hline
GPIO0 & S1 \\
\hline
GPIO1 & S2 \\
\hline
GPIO4 & SDA \\
\hline
GPIO5 & SCL \\
\hline
GPIO6 & Busy \\
\hline
GPIO7 & RST \\
\hline
GPIO8 & DC \\
\hline
GPIO9 & CS \\
\hline
GPIO10 & CLK \\
\hline
GPIO11 & DIN (MOSI) \\
\hline
\end{tabular}
\caption{GPIO-Zuordnung des Raspberry Pico W}
\label{tab:GPIO_Zuordnung}
\end{table}


In Tabelle \ref{tab:GPIO_Zuordnung} sind GPIO0 und GPIO1 dem Auslesen der Taster S1 und S2 (Kap. \ref{sec:Benutzer_Interaktionen}) zugewiesen. GPIO4 und GPIO5 sind den I2C-Kommunikationsleitungen zugeordnet; auf diesen Pins liegt das „Default-I2C“ des Raspberry Pi Pico W (Abb. \ref{fig:PicoW_Pinout}). GPIO7 bis GPIO11 dienen zur Ansteuerung des Displays mit SPI (Abb. \ref{fig:Display_Pinout}).

\bigskip \\

\textbf{Display - Pinout:}

\begin{figure}[!htb]
\centering
\includegraphics[width=0.7\textwidth]{files/Tobias/pics/Pinout/2inch-LCD-Module-4_960.jpg}
\caption[Display - Pinout]{Display - Pinout}
\label{fig:Display_Pinout}
\end{figure}


\begin{table}[H]
\centering
\begin{tabular}{|l|l|}
\hline
\rowcolor{cyan!20}
\textbf{Pin} & \textbf{Funktion} \\
\hline
VCC & Versorgung (3{,}3V / 5V) \\
\hline
GND & Ground \\
\hline
DIN & SPI-Dateninput \\
\hline
CLK & SPI-Taktinput \\
\hline
CS & Chip-Select, Low-Active \\
\hline
DC & Daten-/Befehlauswahl (High = Daten, Low = Befehl) \\
\hline
RST & Reset, Low-Active \\
\hline
BL & Hintergrundbeleuchtung \\
\hline
\end{tabular}
\caption{Pinbelegung des Displays}
\label{tab:display_pins}
\end{table}

\bigskip \\

\textbf{PASCO2 - Pinout:}

\begin{figure}[!htb]
\centering
\includegraphics[width=0.6\textwidth]{files/Tobias/pics/Pinout/minieval_co2_pinout.pdf}
\caption[Display - Pinout]{Display - Pinout}
\label{fig:PASCO2_Pinout}
\end{figure}

Mithilfe des PSEL-Pins wird dem Sensor mitgeteilt, welche Kommunikationsschnittstelle verwendet wird. In diesem Fall wird PSEL für I2C auf GND gezogen.

\bigskip \\

\textbf{BME688 - Pinout:}

\begin{figure}[!htb]
\centering
\includegraphics[width=0.5\textwidth]{files/Tobias/pics/Pinout/environment-3-click-thickbox_default-2.jpg}
\caption[Display - Pinout]{Display - Pinout}
\label{fig:BME688_Pinout}
\end{figure}

Wie in Abbildung \ref{fig:BME688_Pinout} zu sehen ist, befinden sich verschiedene Auswahlmöglichkeiten auf dem ...Board, die mithilfe von Lötbrücken genutzt werden können. Die Kommunikationsschnittstelle kann zwischen SPI und I2C gewählt werden (in diesem Fall I2C), ebenso wie die Adresse des Sensors (0 = 0x76, 1 = 0x77). 





      \section{USB-C}

      Das Messgerät benutzt USB-C für die Spannungsversorgung. Da nur die Versorgungsleitungen des USB-Busses benutzt werden, wird eine USB4125-Buchse verwendet (Kap. \ref{sec:USB4125_75}). Diese hat nur die Versorgungs-, GND- und CC-Pins, welche ausreichend sind, da keine Daten übertragen werden müssen. Ebenfalls erleichtert diese Buchse das Auflöten auf die Platine.

\begin{figure}[!htb]
\centering
\includegraphics[width=0.75\textwidth]{files/Tobias/pics/Schaltungen/Schematik/USBC_Schematik.PNG}
\caption[USB-C Schematik]{USB-C Schematik}
\label{fig:USB-C_Schematik}
\end{figure}

Wie in Abbildung \ref{fig:USB-C_Schematik} zu sehen ist, sind beide CC-Pins der USB-C-Buchse über 5,1-k$\Omega$ Widerstände mit GND verbunden. Bei USB-C erfolgt über diese Pins der Austausch über die Verbindungskonfigurationen. Die 5,1-k$\Omega$ Widerstände signalisieren, dass es sich bei dem angeschlossenen Gerät um einen Verbraucher handelt, der maximal 15W Leistung aufnehmen darf \cite{USBCPowerDelivery}.

\begin{figure}[H] 
  \centering

  \begin{subfigure}[b]{0.48\textwidth}
    \centering
    \includegraphics[height=7cm]{files/Tobias/pics/Schaltungen/PCB/USB_C_PCB.PNG}
    \caption{USB-C Bottom Layer}
    \label{fig:USB-C_Bottom_layer}
  \end{subfigure}
  \hfill
  \begin{subfigure}[b]{0.48\textwidth}
    \centering
    \includegraphics[height=7cm]{files/Tobias/pics/Schaltungen/PCB/USB_C_PCB_3D.PNG}
    \caption{USB-C 3D Ansicht}
    \label{fig:USB-C_3D_Ansicht}
  \end{subfigure}

  \caption{USB-C PCB Ansicht}
  \label{fig:pcb_layers}
\end{figure}



      \section{3,3V Spannungsregler}

Da der Mikrocontroller, die Sensoren sowie das Display mit 3,3V arbeiten und die USB-C-Versorgung 5V bereitstellt, ist ein Spannungsregler notwendig. Verwendet wird der NJM12856 \cite{NJM12856}, da dieser mit 1000mA dem maximalen Strom des Geräts standhält (Tab. \ref{tab:Stromverbrauch}).

Um den maximalen Stromverbrauch des Gerätes zu bestimmen, wurden folgende Werte aus den Datenblättern der einzelnen Komponenten \cite{Raspberry_Pi_Pico_W}, \cite{PASCO2V01}, \cite{BME688}, \cite{LCDDisplayDatasheet} entnommen: 


\renewcommand{\arraystretch}{1}

\begin{table}[H]
\centering
\rowcolors{2}{white}{white}
\begin{tabular}{|l|c|}
\hline
\rowcolor{cyan!20}
\textbf{Komponente} & \textbf{max. Stromaufnahme} \\
\hline
BME688 & 18\,mA \\
\hline
PAS CO\textsubscript{2} & 160\,mA \\
\hline
Display & 45\,mA \\
\hline
Raspberry Pi Pico W & 300\,mA \\
\hline
\end{tabular}
\caption{max. Stromverbrauch der Hauptkomponenten}
\label{tab:Stromverbrauch}
\end{table}

Da für den Raspberry Pico W keine genaue Angabe über den Stromverbrauch im Datenblatt zu finden ist, wurde der obige Wert durch Internetrecherche \cite{PicoWCurrent} und unter Einbezug der benutzten Bussysteme sowie der WLAN-Verbindung geschätzt.

\begin{figure}[!htb]
\centering
\includegraphics[width=0.75\textwidth]{files/Tobias/pics/Schaltungen/Schematik/3V3_Converter_Schematik.PNG}
\caption[3,3V Spannungsregler Schematik]{3,3V Spannungsregler Schematik}
\label{fig:3,3V Spannungsregler Schematik}
\end{figure}

Wie in Abbildung \ref{fig:3,3V Spannungsregler Schematik} zu sehen ist, wurde am Ein- und Ausgang des NJM12856 jeweils ein 4,7µF-Kondensator hinzugefügt, um stabile Spannungen zu gewährleisten. Zusätzlich wurde ein weiterer 100nF-Kondensator am Ausgang des NJM12856 hinzugefügt, um mögliche Störungen zu minimieren. Der CS-Pin bleibt offen, da kein Softstart benötigt wird \cite{NJM12856}. 2 Pins wurden für einen Schalter inkludiert, um die gesamte Spannungsversorgung vom Gerät zu trennen. 


\begin{figure}[H] 
  \centering

  \begin{subfigure}[b]{0.48\textwidth}
    \centering
    \includegraphics[height=7cm]{files/Tobias/pics/Schaltungen/PCB/3V3_Spannungregler_PCB.PNG}
    \caption{3,3V Spannungsregler Bottom Layer}
    \label{fig:USB-C_Bottom_layer}
  \end{subfigure}
  \hspace{2mm} % <-- Abstand verkleinert
  \begin{subfigure}[b]{0.48\textwidth}
    \centering
    \includegraphics[height=7cm]{files/Tobias/pics/Schaltungen/PCB/3V3_Spannungsregler_PCB_3D.PNG}
    \caption{3,3V Spannungsregler 3D Ansicht}
    \label{fig:USB-C_3D_Ansicht}
  \end{subfigure}

  \caption{3,3V Spannungsregler PCB Ansicht}
  \label{fig:pcb_layers}
\end{figure}








      \section{12V Spannungwandler}
      
Der PASCO2V01-Sensor benötigt neben der 3,3V-Spannungsversorgung ebenfalls eine 12V-Spannungsversorgung. Der Aufwärtswandler TLV61046ADBVR \cite{TLV61046} wird dafür benutzt. Dieser ist in dem Datenblatt für Design-Richtlinien des PASCO2V01 \cite{PASCO2_Design_Guidelines}, inklusive Beschaltung, empfohlen.


\begin{figure}[!htb]
\centering
\includegraphics[width=0.75\textwidth]{files/Tobias/pics/Schaltungen/Schematik/12V_Converter_Schematik.PNG}
\caption[12V Spannungswandler Schematik]{12V Spannungswandler Schematik}
\label{fig:12V Spannungswandler Schematik}
\end{figure}

Die Beschaltung des TLV61046ADBVR (Abb. \ref{fig:12V Spannungswandler Schematik}) wurde entsprechend dem Datenblatt für Design-Richtlinien des PASCO2V01 \cite{PASCO2_Design_Guidelines} entnommen. 


\begin{figure}[H] 
  \centering

  \begin{subfigure}[b]{0.48\textwidth}
    \centering
    \includegraphics[height=7cm]{files/Tobias/pics/Schaltungen/PCB/12V_Spannungswandler_PCB_.PNG}
    \caption{12V Spannungswandler Bottom Layer}
    \label{fig:12V_Bottom_layer}
  \end{subfigure}
  \hfill
  \begin{subfigure}[b]{0.48\textwidth}
    \centering
    \includegraphics[height=7cm]{files/Tobias/pics/Schaltungen/PCB/12V_Spannungswandler_PCB_3D.PNG}
    \caption{12V Spannungswandler 3D Ansicht}
    \label{fig:12V_3D_Ansicht}
  \end{subfigure}

  \caption{12V Spannungswandler PCB Ansicht}
  \label{fig:pcb_layers}
\end{figure}
      


   \section{PCB Version 1}
   \label{ref:PCB_Version_1}


   \begin{figure}[H] 
  \centering
  \begin{subfigure}[b]{0.48\textwidth}
    \centering
    \includegraphics[height=7cm]{files/Tobias/pics/Schaltungen/PCB/Version1_Top.PNG}
    \caption{PCB Version 1 - Top Layer}
    \label{fig:PCB_Version1_Top}
  \end{subfigure}
  \hfill
  \begin{subfigure}[b]{0.48\textwidth}
    \centering
    \includegraphics[height=7cm]{files/Tobias/pics/Schaltungen/PCB/Version1_Bottom.PNG}
    \caption{PCB Version 1 - Bottom Layer}
    \label{fig:PCB_Version1_Bot}
  \end{subfigure}
  \caption{PCB Version 1}
  \label{fig:PCB_Version_1}
\end{figure}

Da das Display im Gehäuse befestigt wird und sich nicht direkt auf der Platine befindet, wurde ein JST XH 2.54mm-Stecker (Female) für dessen Verbindung benutzt. Als Taster wurden 2-1825027-0 Taster verwendet. Diese sind geknickt und haben eine Knopflänge von 9,24mm. Dadurch lassen sich die Taster gut in das Gehäuse integrieren. Für die Befestigung im Gehäuse wurden auf der Platine 2 Löcher für M3-Schrauben vorgesehen. Ein drittes Loch direkt unter dem Mikrocontroller dient nur als Platzhalter.

  
\section{PCB Version 2}


   \begin{figure}[H] 
  \centering
  \begin{subfigure}[b]{0.48\textwidth}
    \centering
    \includegraphics[height=7cm]{files/Tobias/pics/Schaltungen/PCB/Version2_Top.PNG}
    \caption{PCB Version 2 - Top Layer}
    \label{fig:PCB_Version2_Top}
  \end{subfigure}
  \hfill
  \begin{subfigure}[b]{0.48\textwidth}
    \centering
    \includegraphics[height=7cm]{files/Tobias/pics/Schaltungen/PCB/Version2_Bottom.PNG}
    \caption{PCB Version 2 - Bottom Layer}
    \label{fig:PCB_Version2_Bot}
  \end{subfigure}
  \caption{PCB Version 2}
  \label{fig:PCB_Version_2}
\end{figure}

Für eine bessere Benutzerfreundlichkeit wurde die USB-C-Buchse von der ursprünglich seitlichen Position auf die Rückseite des PCBs gesetzt. Ebenfalls vereinfacht dies die Integration in das Gehäuse. Dabei wurde von einer USB4125-Buchse auf eine USB4175-Buchse (Kap. \ref{sec:USB4125_75}) gewechselt. Zusätzlich wurde ein weiteres Loch für eine Schraube zur Befestigung im Gehäuse eingebaut. Die ursprünglichen Positionen der SMD-Bauteile sowie der Sensoren wurden ebenfalls geändert.

   \begin{figure}[H] 
  \centering
  \begin{subfigure}[b]{0.48\textwidth}
    \centering
    \includegraphics[height=7cm]{files/Tobias/pics/Schaltungen/3D/Top_Layer_3D.PNG}
    \caption{PCB Version 2 - Top Layer 3D Ansicht}
    \label{fig:PCB_Version2_Top_3D}
  \end{subfigure}
  \hfill
  \begin{subfigure}[b]{0.48\textwidth}
    \centering
    \includegraphics[height=7cm]{files/Tobias/pics/Schaltungen/3D/Bottom_Layer_3D.PNG}
    \caption{PCB Version 2 - Bottom Layer 3D Ansicht}
    \label{fig:PCB_Version2_Bot_3D}
  \end{subfigure}
  \caption{PCB Version 2 - 3D Ansicht}
  \label{fig:PCB_Version_2_3D}
\end{figure}



\section{Fertigung des PCBs/Prints}

Die zweite PCB-Version wurde bei JCL PCB bestellt und im Anschluss eigenständig bestückt. Da der mitgelieferte Verbindungstecker das Displays sehr lange Kabel hat und diese im Gehäuse zu viel Platz verbrauchen würden, wurden diese von 18cm Länge auf 10cm gekürtzt. Am abgetrennten Ende wurde ein JST XH 2.54mm-Stecker (Male) befestigt.



\section{Messungen}
	\subsection{Spannungen}

    Nach der Bestückung der Platine wurden die Versorgungsspannungen mit einem Multimeter überprüft. Sowohl die 3,3V- als auch die 12V-Versorgung waren mit minimalen Abweichungen korrekt.

\end{inhalt}                                      

%Chapter 6 Mikrocontroller-Programmierung
\responsible{Tobias Geppl} 
\begin{inhalt}
\renewcommand*\chapterpagestyle{scrheadings}
\chapter{Mikrocontroller - Programmierung}

\section{PAS CO2}
\section{BME688}

\section{HTTPS}

\section{Display Ansteuerung}

\section{Display Interface}
\subsection{Screens}
\subsection{Widgets}
\subsection{Buttons}



\end{inhalt}                                      


%Chapter 7 Website
\responsible{Thomas Potzmader} 
\begin{inhalt}
\renewcommand*\chapterpagestyle{scrheadings}
\chapter{Website}


\section{Signup}

Für Das Zuordnen der jeweiligen Benutzer ist es Notwendig das die User Sich einloggen können. Dafür muss ein Anmelde Komponente erstellt werden.

\begin{figure}[!htb]
\centering
\includegraphics[width=1\textwidth]{files/Thomas/pics/Website/Signup/sign-up.png}
\caption[Bildbezeichnung für Abbildungsverzeichnis]{}
\label{fig:gehaeuse_internet_bild}
\end{figure}

Zur Validierung der Eingaben in den Feldern für \texttt{E-Mail} und \texttt{Passwort} wird die Bibliothek \textbf{Zod} verwendet. Dabei überprüft \textbf{Zod}, ob das E-Mail-Feld eine syntaktisch gültige E-Mail-Adresse enthält und ob das Passwortfeld einen String mit einer Mindestlänge von sechs Zeichen beinhaltet.


\begin{lstlisting}[style=mytsx]
const formSchema = z.object({
  email: z.string().email({
    message: "Please enter a valid email.",
  }),
  password: z.string().min(6, {
    message: "Password must be at least 6 characters.",
  }),
})
\end{lstlisting}
\index{Code Snippet 1@\hyperref[code:snippet1]{Snippet 1}}

Für den Registrierungsvorgang wurde eine Funktion implementiert, die die eingegebene E-Mail-Adresse sowie das Passwort entgegennimmt und diese über eine Funktion des Supabase-Auth-Services im System registriert.


\begin{lstlisting}[style=mytsx]
export async function insert_user(email: string, password: string) {
  const supabase = await createClient();
  const { error } = await supabase.auth.signUp({
    email,
    password,
  });
  return error;
}
\end{lstlisting}

Nachdem der Nutzer erfolgreich registriert wurde, erfolgt eine Weiterleitung auf die Seite \texttt{/checkyouremails}, um ihn darüber zu informieren, dass er seine E-Mails überprüfen soll.

\begin{lstlisting}[style=mytsx]
      router.push("checkyouremails");
\end{lstlisting}

\subsection{Check Your Emails}

Auf dieser Seite checkt die Seite ob im Table profiles \ref{}


\section{Login}
Für Das Zuordnen der jeweiligen Benutzer ist es Notwendig das die User Sich einlogen können

\begin{figure}[!htb]
\centering
\includegraphics[width=1\textwidth]{files/Thomas/pics/Website/Login/login-screen.png}
\caption[Bildbezeichnung für Abbildungsverzeichnis]{}
\label{fig:gehaeuse_internet_bild}
\end{figure}

Dafür wurde ein Login Komponente erstellt.

\begin{figure}[!htb]
\centering
\includegraphics[width=1\textwidth]{files/Thomas/pics/Website/Login/login.png}
\caption[Bildbezeichnung für Abbildungsverzeichnis]{}
\label{fig:gehaeuse_internet_bild}
\end{figure}

Für das Login Komponent wurde Zod für die überprüfung von 

\begin{lstlisting}[style=mytsx]
export async function login_user(email: string, password: string) {
  const supabase = await createClient();
  const { error } = await supabase.auth.signInWithPassword({
    email,
    password,
  });
  return error;
}
\end{lstlisting}






























\section{Admin Dashboard}

Das Admin Dashboard ist dafür da um alle User, Geräte, Klassen, Abteilungen und Schulen zu Erstellen oder zu Löschen. 

\subsection{Tables}

\subsubsection{School}

\subsubsection{Departments}

\subsubsection{Class}

\subsubsection{Devices}

\subsubsection{User}

\subsection{Create}

Auf allen Seiten der Administrationsoberfläche kommt ein spezielles \emph{Create}-Component zum Einsatz, das jeweils für die Erstellung einzelner Objekte zuständig ist. Die Namensgebung folgt dabei einer klaren Konvention, beispielsweise \texttt{create-school.tsx} für den Schulbereich. Der Begriff \emph{Create} leitet sich aus dem englischen Begriff „erstellen“ ab, was die primäre Funktion dieser Komponente treffend beschreibt.

Jede dieser Komponenten ist individuell auf die jeweilige Entität zugeschnitten, um eine optimale Handhabung und Validierung der Eingabedaten zu gewährleisten. Nach erfolgreicher Erstellung werden die eingegebenen Daten in der Supabase-Datenbank gespeichert, was eine konsistente und strukturierte Datenhaltung ermöglicht.

\subsubsection{School}






Im Folgenden wird erläutert, wie in diesem Projekt die Autovervollständigung für Adressen mithilfe der Google API implementiert wurde.

\subsubsection{Google API für Autovervollständigung}

Für die Implementierung der Autovervollständigung wurde das npm-Package \texttt{react-google-maps-api} eingesetzt. Diese Lösung bietet eine einfache Möglichkeit, Google-basierte Adressvorschläge in das Projekt zu integrieren. Hierfür musste im Google-Dashboard die neue Places API aktiviert werden, da seit dem 17.03.2025 keine Legacy-Version mehr verfügbar ist.

Das verwendete \texttt{AddressAutocomplete}-Component stellt ein Beispiel für die Umsetzung dar:

\begin{lstlisting}[style=mytsx]
<AddressAutocomplete
  value={field.value}
  onChange={field.onChange}
  onBlur={field.onBlur}
/>
\end{lstlisting}

Mit diesem Component kann der Benutzer eine Adresse oder einen Ort suchen. Google liefert daraufhin entsprechende Vorschläge, die anschließend in einem JSON-Format gespeichert werden. Ein exemplarischer Aufbau des JSON-Objekts sieht folgendermaßen aus:

\begin{lstlisting}[style=myjson]
{
  location: {
    address: "",
    location: {
      latitude: 0,
      longitude: 0,
    },
    address_id: null,
  },
}
\end{lstlisting}

Die Validierung und Verwaltung der erhaltenen Daten erfolgt mittels \texttt{Zod} (siehe Abschnitt \ref{subsec:Zod}). Dies gewährleistet, dass die eingegebene Adresse den Mindestanforderungen entspricht und die Geodaten korrekt formatiert sind. Im folgenden Listing wird die Validierung mittels \texttt{Zod} veranschaulicht:

\begin{lstlisting}[style=myjson]
{
  location: z.object({
    address: z.string().min(2, {
      message: "Please enter a valid address.",
    }),
    location: z.object({
      latitude: z.number(),
      longitude: z.number(),
    }),
    address_id: z.string().nullable(),
  }),
}
\end{lstlisting}

Die Entscheidung, die Autovervollständigung über das genannte Package zu realisieren, ergab sich aus der besseren Integration und Einfachheit im Vergleich zur direkten Nutzung der Places API von Google. Zwar stellt Google ein eigenes Component zur Verfügung, jedoch entsprach dessen optische Gestaltung nicht den Vorgaben des ShadCN-Designs, welches in der Website Anwendung findet. Die gewählte Lösung ermöglicht somit eine einheitliche Benutzeroberfläche und erleichtert die Implementierung der gewünschten Funktionalität.


\subsubsection{Departments}

\subsubsection{Class}

\subsubsection{Devices}

\subsubsection{User}

\subsection{Schools-Map}

\subsubsubsection{Google API Maps}

In


\section{Dashboard}

\section{Backend}



API route






\end{inhalt}     

%Chapter 8 Datenbank
\responsible{Thomas Potzmader} 
\begin{inhalt}
\renewcommand*\chapterpagestyle{scrheadings}
\chapter{Datenbank}

In diesem Abschnitt werden die Struktur, verwendete Tabellen sowie die wichtigen SQL-Befehle, Trigger und Funktionen der Datenbank detailliert erläutert.

\section{Tabellen}

Die Datenbank besteht aus den folgenden Tabellen:

\subsection{Schule (Schools)}
Die Tabelle speichert Informationen zu den Schulen, einschließlich Standort und Webseite. Der Primärschlüssel (\texttt{id}) wird automatisch generiert.

\begin{lstlisting}[style=mysql]
CREATE TABLE schools (
    id UUID PRIMARY KEY DEFAULT gen_random_uuid(),
    created_at TIMESTAMPTZ DEFAULT now(),
    name TEXT,
    location JSONB,
    website_url TEXT
);
\end{lstlisting}

\subsection{Abteilungen (Departments)}
Die Tabelle verwaltet die verschiedenen Abteilungen innerhalb einer Schule. Jede Abteilung ist eindeutig einer Schule zugeordnet.

\begin{lstlisting}[style=mysql]
CREATE TABLE departments (
    id UUID PRIMARY KEY DEFAULT gen_random_uuid(),
    created_at TIMESTAMPTZ DEFAULT now(),
    name TEXT,
    school_id UUID REFERENCES schools(id)
);
\end{lstlisting}

\subsection{Klassen (Classes)}
Diese Tabelle speichert Informationen zu den Klassen, die spezifischen Abteilungen angehören.

\begin{lstlisting}[style=mysql]
CREATE TABLE classes (
    id UUID PRIMARY KEY DEFAULT gen_random_uuid(),
    created_at TIMESTAMPTZ DEFAULT now(),
    name TEXT,
    department_id UUID REFERENCES departments(id)
);
\end{lstlisting}

\subsection{Geräte (Sensors)}
Die Tabelle enthält Informationen zu den einzelnen Sensor-Geräten, welche jeweils einer Klasse  ist.

\begin{lstlisting}[style=mysql]
CREATE TABLE sensors (
    id UUID PRIMARY KEY DEFAULT gen_random_uuid(),
    name TEXT,
    created_at TIMESTAMPTZ DEFAULT now()
);
\end{lstlisting}

\subsection{Benutzerprofile (Profiles)}
In dieser Tabelle werden Benutzerdaten, Rollen und deren Zugehörigkeit zu Klassen verwaltet. Zudem sind hier Statusinformationen wie \texttt{approved} und \texttt{email\_confirmed\_at} gespeichert.

\begin{lstlisting}[style=mysql]
CREATE TABLE profiles (
    id UUID PRIMARY KEY DEFAULT gen_random_uuid(),
    created_at TIMESTAMPTZ DEFAULT now(),
    first_name TEXT,
    surname TEXT,
    username TEXT,
    role INTEGER,
    class_id UUID REFERENCES classes(id),
    email_confirmed_at TIMESTAMPTZ,
    approved BOOLEAN
);
\end{lstlisting}

\subsection{Trigger und Funktionen}

\textbf{handle\_new\_user}


Diese Funktion erstellt automatisch ein Benutzerprofil, sobald ein neuer Benutzer angelegt wird. Dies erleichtert die Verwaltung von Nutzerdaten.

\begin{lstlisting}[style=mysql]
CREATE OR REPLACE FUNCTION public.handle_new_user()
 RETURNS trigger
 LANGUAGE plpgsql
 SECURITY DEFINER
 SET search_path TO ''
AS $function$
begin
  insert into public.profiles (id)
  values (new.id);
  return new;
end;
$function$;
\end{lstlisting}

\vspace{0.75cm}

\textbf{handle\_updated\_email\_confirmed\_at}
Diese Funktion aktualisiert das Feld \texttt{email\_confirmed\_at} im Profil eines Benutzers, wenn die E-Mail-Adresse bestätigt wird. Dies gewährleistet die Synchronität zwischen Authentifizierungs- und Profiltabellen.

\begin{lstlisting}[style=mysql]
CREATE OR REPLACE FUNCTION public.handle_updated_email_confirmed_at()
 RETURNS trigger
 LANGUAGE plpgsql
 SECURITY DEFINER
 SET search_path TO ''
AS $function$
BEGIN
  IF NEW.email_confirmed_at IS DISTINCT FROM OLD.email_confirmed_at THEN
    UPDATE public.profiles
    SET email_confirmed_at = NEW.email_confirmed_at
    WHERE id = NEW.id;
  END IF;
  RETURN NEW;
END;
$function$;
\end{lstlisting}

Die Implementierung dieser Funktion erfolgt ebenfalls durch einen Trigger, welcher Änderungen in der Tabelle \texttt{auth.users} überwacht.

\newpage

\section{Storage}

Für die Avatar-Bilder wurde die \texttt{Storage}-Funktion von Supabase verwendet, um Bilder einfach abrufen und hochladen zu können.  
Hierfür wurde ein Bucket (oberster Ordner bzw. Container) mit dem Namen \texttt{Avatar} erstellt.

\begin{figure}[!htb]
\centering
\includegraphics[width=0.55\textwidth]{files/Thomas/pics/Datenbank/storage/supabase_bucket_create.png}
\caption[Erstellen eines Buckets in Supabase]{Erstellen eines Buckets in Supabase}
\label{fig:supabase_bucket_create}
\end{figure}

Wie in Abbildung \ref{fig:supabase_bucket_create} zu sehen ist, wurde der Bucket so konfiguriert, dass nur Bilder mit einer maximalen Dateigröße von 5\,MB und ausschließlich Dateien im \texttt{image}- oder \texttt{gif}-Format hochgeladen werden dürfen.

Die Dateistruktur des Buckets ist wie folgt aufgebaut:

\begin{lstlisting}[style=mysql]
avatar/
 └── [user-id]/
      └── avatar.png
\end{lstlisting}

Somit kann jedem Benutzer ein individuelles Profilbild zugewiesen werden.

\newpage

\section{Auth}

Für die Authentifizierung wurde festgelegt, dass sich Benutzer mittels E-Mail und Passwort anmelden müssen.  
Bevor sich ein Benutzer jedoch anmelden darf, muss er zunächst seine E-Mail-Adresse verifizieren.  
Diese Funktion kann in den Supabase-Einstellungen aktiviert werden.  
Sobald ein Benutzer sich registriert, sendet Supabase automatisch eine Verifizierungs-E-Mail.

\begin{figure}[!htb]
\centering
\includegraphics[width=0.75\textwidth]{files/Thomas/pics/Datenbank/auth/signup-email.png}
\caption[Supabase E-Mail-Verifizierung]{E-Mail-Verifizierung bei Supabase}
\label{fig:supabase_signup_email}
\end{figure}

Der Benutzer muss anschließend nur noch den Verifizierungslink in der E-Mail anklicken, um seine E-Mail-Adresse erfolgreich zu bestätigen.










\end{inhalt}

%Chapter 9 Gehäuse
\responsible{Thomas Potzmader} 
\begin{inhalt}
\renewcommand*\chapterpagestyle{scrheadings}
\chapter{Gehäuse}

Beim Gehäuse wurde wie in \ref{ref:fusion360_grundlagen} erwähnt Fussion 360 benutzt um ein Gehäuse zu Designen. Um besser ein Gehäuse zu Designen muss man ein paar Sachen vorbereiten. In diesen Fall kann man die PCB von Altium als Step datei exportiren und in Fusion hinaufladen um besser ein Gehäuse rund um die Platine zu Designen. Dafür wurde ein Spezieller Artikel benutzt (https://www.pcbway.com/blog/PCB_Design_Tutorial/3D_model_Render_with_copper_from_Altium_to_Fusion_360_.html). 

\begin{figure}[!htb]
\centering
\includegraphics[width=0.75\textwidth]{files/Thomas/pics/geheause/pcb_fusion.png}
\caption[Bildbezeichnung für Abbildungsverzeichnis]{}
\label{fig:gehaeuse_internet_bild}
\end{figure}

Außerdem Wurde Das Display von hier gedownloadund und auch in das project gebracht

\begin{figure}[!htb]
\centering
\includegraphics[width=0.75\textwidth]{files/Thomas/pics/geheause/display_fusion.png}
\caption[Bildbezeichnung für Abbildungsverzeichnis]{}
\label{fig:gehaeuse_internet_bild}
\end{figure}

Damit kann nun gestarted werden.

\subsection{Version 1.0}

\begin{figure}[!htb]
\centering
\includegraphics[width=0.75\textwidth]{files/Thomas/pics/geheause/1.0/gehaeuse_side.png}
\caption[Bildbezeichnung für Abbildungsverzeichnis]{}
\label{fig:gehaeuse_internet_bild}
\end{figure}

Bei dem ersten meilenstein gab es noch viele Probleme. Wie man in der Abbildung sehen kann war diese Version mit der Version 1.0 pcb (\ref{ref:PCB_Version_1}). Bei dieser Muste man auf den USB-C Port auf der Seite aufpassen Dieser auch sehr nahe auf der PCB dies ist schlecht da USB-C Kabel nicht genormt sind und man nie weis wie groß die Hülle ist muss man immer ein großes Loch machen Damit man mit jeden Stecker benutzen kann. Doch dieses Problem sollte bald gelöst sein.

Außer dem Wurden nur zwei clips Benutzt in der Front aber diese Waren zu wenige dieses Problem wurde auch später gelöst.

Für den Richtigen Airflow wurde das Selbe Muster wie von der Internet rechereche benutzt.

\begin{figure}[!htb]
\centering
\includegraphics[width=0.75\textwidth]{files/Thomas/pics/geheause/1.0/gehaeuse_top.png}
\caption[Bildbezeichnung für Abbildungsverzeichnis]{}
\label{fig:gehaeuse_internet_bild}
\end{figure}

Wie man sehen kann ist auch bei der ersten version die Seiten Breite sehr Klein und somit brach diese Teile jedes mal ab. 

Außerdem passte beim Obriggen Teil das Display noch nicht ganz.

\newpage

\subsection{Version 1.1}

\begin{figure}[!htb]
\centering
\includegraphics[width=0.75\textwidth]{files/Thomas/pics/geheause/1.1/gehaeuse_top.png}
\caption[Bildbezeichnung für Abbildungsverzeichnis]{}
\label{fig:gehaeuse_internet_bild}
\end{figure}

Der große Unterschied zwischen Version 1.0 und 1.1 ist das die Seiten vom Gehäuse Dicker geworden sind um nicht so leicht zu brechen

\subsection{Version 1.2}

\begin{figure}[!htb]
\centering
\includegraphics[width=0.75\textwidth]{files/Thomas/pics/geheause/1.2/gehaeuse_side.png}
\caption[Bildbezeichnung für Abbildungsverzeichnis]{}
\label{fig:gehaeuse_internet_bild}
\end{figure}

In dieser Version wurden 2 weitere clips hinzugefügt um das Verschliesen besser zu machen sonst würde Das gehäuse zu leicht augemacht werden

\begin{figure}[!htb]
\centering
\includegraphics[width=0.75\textwidth]{files/Thomas/pics/geheause/1.2/gehaeuse_back.png}
\caption[Bildbezeichnung für Abbildungsverzeichnis]{}
\label{fig:gehaeuse_internet_bild}
\end{figure}

Außerdem wurde nun wenn das gehäuse zu stereng aufgeht ein Schlitz zwischen den Obrigen teil und den untrigen Teil gemacht um zu garantiren das man mit einen Schlitzschraufenziher das gehäuse aufzuamachen können.

Außerdem wurde die ausschneidung für das Display verbessert nun passt es perfekt in das Gehäuse mit Toleranzen und allem drum un dran


\subsection{Version 2.0}

\begin{figure}[!htb]
\centering
\includegraphics[width=0.75\textwidth]{files/Thomas/pics/geheause/2.0/gehaeuse_side.png}
\caption[Bildbezeichnung für Abbildungsverzeichnis]{}
\label{fig:gehaeuse_internet_bild}
\end{figure}

Da nun die PCB Version zwei (\ref{ref:PCB_Version_2}) Fertig war wurde das Gehäuse umdesigned um der PCB wieder zu passen Die wichtigsten Dinge sind da der USB-C nicht mehr auf der Seite ist kann man nun die clips Symetrisch von der Mitte gleich weit Wegplatziren damit wird noch mehr halt zwischen den Teilen versporchen.  

Außerdem wurde für das Montiren in den Kabelkanal in den Schulraume in der HTL St Pölten  kleine Sclitze gemacht die dafür sorgen das man das Geräte in den Adapter hineinstecken kann

\begin{figure}[!htb]
\centering
\includegraphics[width=0.75\textwidth]{files/Thomas/pics/geheause/2.0/gehaeuse_bot.png}
\caption[Bildbezeichnung für Abbildungsverzeichnis]{}
\label{fig:gehaeuse_internet_bild}
\end{figure}

Da nun der USB-C nicht merh auf der Seite ist sondern auf der Unterseite ist Ganze Thema mit USB-C gehäuse größen egal den jetzt passt jedes usb-c kabel. 

\subsection{Version 2.1}

\begin{figure}[!htb]
\centering
\includegraphics[width=0.75\textwidth]{files/Thomas/pics/geheause/2.1/gehaeuse_side.png}
\caption[Bildbezeichnung für Abbildungsverzeichnis]{}
\label{fig:gehaeuse_internet_bild}
\end{figure}

Da Die Clips immer wieder probleme gemacht haben wurden diesen innen Verstärkt und wurden da es probleme mit der Dicke gab verkleinert 









\end{inhalt}   


%%ANHANG
%===================================
\appendix
\responsible{Name 1, Name 2} 

%%Literaturverzeichnis
%=============================%%
%
\begin{literature}

\section*{Internetlinks}


%%  Grundlagen & Methoden


%%TOBIAS GEPPL LINKS

\bibitem[1]{AltiumDesignerWiki}
\textbf{ALtium Designer:} \textit{Was ist ALtium Designer}, 18. September 2024. \url{https://de.wikipedia.org/wiki/Altium_Designer}

\bibitem[2]{I2CKommunkation}
\textbf{I2C Bussystem:} \textit{Was ist I2C}, 18. September 2024. \url{https://de.wikipedia.org/wiki/I%C2%B2C}

\bibitem[3]{SPI_Kommunikation}
\textbf{SPI Bussystem:} \textit{Was ist SPI}, 18. September 2024. \url{https://de.wikipedia.org/wiki/Serial_Peripheral_Interface}

\bibitem[4]{HTTPS_Kommunkation}
\textbf{HTTPS Protokoll:} \textit{Was ist HTTPS}, 18. September 2024. \url{https://de.wikipedia.org/wiki/Hypertext_Transfer_Protocol_Secure}

\bibitem[5]{VisualStudioCode}
\textbf{VS Code Texteditor:} \textit{Was ist HTTPS}, 18. September 2024. \url{https://de.wikipedia.org/wiki/Visual_Studio_Code}

\bibitem[6]{Raspberry_Pi_Pico_Erweiterung}
\textbf{Raspberry Pi Pico Erweiterung:} \textit{Was ist HTTPS}, 18. September 2024. \url{https://de.wikipedia.org/wiki/Hypertext_Transfer_Protocol_Secure}

\bibitem[7]{Raspberry_Pi_Pico_W}
\textbf{Raspberry Pi Pico W:} \textit{Was ist HTTPS}, 18. September 2024. \url{https://datasheets.raspberrypi.com/picow/pico-w-datasheet.pdf}

\bibitem[8]{PASCO2V01}
\textbf{PASCO2V01 Sensor:} \textit{Was ist HTTPS}, 18. September 2024. \url{https://www.infineon.com/dgdl/Infineon-PASCO2V01-DataSheet-v01_70-EN.pdf?fileId=8ac78c8c80027ecd01809278f1af1ba2}

\bibitem[9]{BME688}
\textbf{BME688 Sensor:} \textit{Was ist HTTPS}, 18. September 2024. 
\url{https://www.bosch-sensortec.com/media/boschsensortec/downloads/datasheets/bst-bme688-ds000.pdf}

\bibitem[10]{2Inch LCD Display Wiki}
\textbf{Waveshare 2Inch LCD Display Wiki:} \textit{Was ist HTTPS}, 18. September 2024. \url{https://www.waveshare.com/wiki/2inch_LCD_Module?Amazon}

\bibitem[11]{USB4125}
\textbf{USB4125:} \textit{Was ist HTTPS}, 18. September 2024. \url{https://gct.co/connector/usb4125}

\bibitem[12]{USB4175}
\textbf{USB4175:} \textit{Was ist HTTPS}, 18. September 2024. \url{https://gct.co/connector/usb4175}

\bibitem[13]{USBC_Kommunikation}
\textbf{USB-C Power Delivery:} \textit{Was ist HTTPS}, 18. September 2024. \url{https://resources.altium.com/de/p/add-usb-type-c-power-delivery-your-designs}

\bibitem[14]{NJM12856}
\textbf{NJM12856:} \textit{Was ist HTTPS}, 18. September 2024. \url{https://www.mouser.com/datasheet/2/294/NJM12856_E-1917311.pdf?srsltid=AfmBOopjT0bgSb_HDB46CyuoYJy9D6rqTyAHd5STTh1L03hbkJLPfPKi}

\bibitem[15]{PicoWCurrent}
\textbf{Raspberry Pico W Stromverbrauch:} \textit{Was ist HTTPS}, 18. September 2024. \url{https://peppe8o.com/raspberry-pi-pico-w-power-consumption/}

\bibitem[16]{2Inch LCD Display Datasheet}
\textbf{Waveshare 2Inch LCD Display:} \textit{Was ist HTTPS}, 18. September 2024. \url{https://files.waveshare.com/upload/b/b1/2inch_LCD_Module.pdf}

\bibitem[17]{TLV61046}
\textbf{LV61046ADBVR:} \textit{Was ist HTTPS}, 18. September 2024. \url{https://www.ti.com/lit/ds/symlink/tlv61046a.pdf?ts=1742970576219&ref_url=https%253A%252F%252Fwww.ti.com%252Fproduct%252FTLV61046A}

\bibitem[18]{PASCO2_Design_Guidelines}
\textbf{PASCO2V01 Design Guidelines:} \textit{Was ist HTTPS}, 18. September 2024. \url{https://www.infineon.com/dgdl/Infineon-PAS_CO2_General_Design-In_Guideline.docx.-ApplicationNotes-v01_02-EN.pdf?fileId=5546d4627a0b0c7b017a5174394768a1}

\bibitem[19]{Excalidraw}
\textbf{Excalidraw:} \textit{Was ist HTTPS}, 18. September 2024. \url{https://excalidraw.com/}


%%THOMAS PETER POTZMADER LINKS 

\bibitem[100]{WebEntwicklungWiki}
\textbf{Was ist Web-Entwicklung:} \textit{Was ist Web-Entwicklung}, 18. September 2024. \url{https://de.wikipedia.org/wiki/Webentwicklung#:~:text=Als%20Webentwicklung%20(englisch%20Web%20development,dagegen%20meist%20von%20Webdesignern%20%C3%BCbernommen}

\bibitem[200]{WebEntwicklungFrontendWiki}
\textbf{Was ist Web-Entwicklung:} \textit{Was ist Web-Entwicklung}, 18. September 2024. \url{https://en.wikipedia.org/wiki/Front-end_web_development}

\bibitem[300]{WebEntwicklungFrontendBackendWiki}
\textbf{Was ist Web-Entwicklung:} \textit{Was ist Web-Entwicklung}, 18. September 2024. \url{https://en.wikipedia.org/wiki/Frontend_and_backend}

\bibitem[400]{DatenBankWiki}
\textbf{Was ist Web-Entwicklung:} \textit{Was ist Web-Entwicklung}, 18. September 2024. \url{https://de.wikipedia.org/wiki/Datenbank}

\bibitem[500]{NextJSWiki}
\textbf{Was ist Web-Entwicklung:} \textit{Was ist Web-Entwicklung}, 18. September 2024. \url{https://en.wikipedia.org/wiki/Next.js}

%\bibitem[5]{NextJS}
%\textbf{Was ist Web-Entwicklung:} \textit{Was ist Web-Entwicklung}, 18. September 2024. %\url{https://nextjs.org/}

\bibitem[600]{ReactWiki}
\textbf{Was ist Web-Entwicklung:} \textit{Was ist Web-Entwicklung}, 18. September 2024. \url{https://en.wikipedia.org/wiki/React_(software)}

%\bibitem[6]{ReactWiki}
%\textbf{Was ist Web-Entwicklung:} \textit{Was ist Web-Entwicklung}, 18. September 2024. %\url{https://react.dev/}

\bibitem[700]{TypeScriptWiki}
\textbf{Was ist Web-Entwicklung:} \textit{Was ist Web-Entwicklung}, 18. September 2024. \url{https://en.wikipedia.org/wiki/TypeScript}

\bibitem[800]{TailwindWiki}
\textbf{Was ist Web-Entwicklung:} \textit{Was ist Web-Entwicklung}, 18. September 2024. \url{https://en.wikipedia.org/wiki/Tailwind_CSS}

\bibitem[900]{ShadCN}
\textbf{Was ist Web-Entwicklung:} \textit{Was ist Web-Entwicklung}, 18. September 2024. \url{https://ui.shadcn.com/docs/about}

\bibitem[1000]{Zustand}
\textbf{Was ist Web-Entwicklung:} \textit{Was ist Web-Entwicklung}, 18. September 2024. \url{https://zustand.docs.pmnd.rs/getting-started/introduction}

\bibitem[110]{Zod}
\textbf{Was ist Web-Entwicklung:} \textit{Was ist Web-Entwicklung}, 18. September 2024. \url{https://zod.dev/}

\bibitem[120]{Supabase}
\textbf{Was ist Web-Entwicklung:} \textit{Was ist Web-Entwicklung}, 18. September 2024. \url{https://supabase.com/}

\bibitem[130]{SupabaseDB}
\textbf{Was ist Web-Entwicklung:} \textit{Was ist Web-Entwicklung}, 18. September 2024. \url{https://supabase.com/docs/guides/database/overview}

\bibitem[140]{SupabaseAuth}
\textbf{Was ist Web-Entwicklung:} \textit{Was ist Web-Entwicklung}, 18. September 2024. \url{https://supabase.com/docs/guides/auth}

\bibitem[150]{SupabaseStorage}
\textbf{Was ist Web-Entwicklung:} \textit{Was ist Web-Entwicklung}, 18. September 2024. \url{https://supabase.com/docs/guides/storage}

\bibitem[160]{SupabaseRealtime}
\textbf{Was ist Web-Entwicklung:} \textit{Was ist Web-Entwicklung}, 18. September 2024. \url{https://supabase.com/docs/guides/realtime}

\bibitem[170]{Vercel}
\textbf{Was ist Web-Entwicklung:} \textit{Was ist Web-Entwicklung}, 18. September 2024. \url{https://en.wikipedia.org/wiki/Vercel}

\bibitem[180]{Fusion360Wiki}
\textbf{Was ist Web-Entwicklung:} \textit{Was ist Web-Entwicklung}, 18. September 2024. \url{https://en.wikipedia.org/wiki/Fusion_360}

\bibitem[190]{SlicerWiki}
\textbf{Was ist Web-Entwicklung:} \textit{Was ist Web-Entwicklung}, 18. September 2024. \url{https://en.wikipedia.org/wiki/Slicer_(3D_printing)}

\bibitem[204]{3DDruckerWiki}
\textbf{Was ist Web-Entwicklung:} \textit{Was ist Web-Entwicklung}, 18. September 2024. \url{https://de.wikipedia.org/wiki/3D-Druck}


%% Design & Konzept

%%TOBIAS GEPPL LINKS









%%THOMAS PETER POTZMADER LINKS 

\bibitem[210]{TemuGehaeuseURL}
\textbf{Was ist Web-Entwicklung:} \textit{Was ist Web-Entwicklung}, 18. September 2024. \url{https://www.temu.com/at/kuiper/n9.html?subj=googleshopping-landingpage&_bg_fs=1&_p_rfs=1&_x_ads_channel=google&_x_ads_sub_channel=shopping&_x_login_type=Google&_x_vst_scene=adg&mkt_rec=1&goods_id=601099796472236&sku_id=17593341043996&_x_ns_sku_id=17593341043996&_x_gmc_account=759765814&_x_ads_account=3754959944&_x_ads_set=22287958068&_x_ads_id=172614047741&_x_ads_creative_id=735412192030&_x_ns_source=g&_x_ns_gclid=CjwKCAjwp8--BhBREiwAj7og19_yz0RlMuNXGbBwIhwytKo4-4VMhTFLU35PmC44Q9M8dK-UKoXhExoCcWcQAvD_BwE&_x_ns_placement=&_x_ns_match_type=&_x_ns_ad_position=&_x_ns_product_id=17593341043996&_x_ns_target=&_x_ns_devicemodel=&_x_ns_wbraid=Cj8KCQjwhMq-BhDSARIuAFfrQ2u_rXLoGzqxmfD54jCtOS-2Q7j7dp0ChV0viGcLG9BmZL3WMPp86n2YexoC4WY&_x_ns_gbraid=0AAAAAo4mICFBjRuOi9LzOClrNUMM1XIA1&_x_ns_targetid=pla-2404451734715&gad_source=1&gclid=CjwKCAjwp8--BhBREiwAj7og19_yz0RlMuNXGbBwIhwytKo4-4VMhTFLU35PmC44Q9M8dK-UKoXhExoCcWcQAvD_BwE&adg_ctx=f-9a69b30f}

\bibitem[220]{GehaeuseYoutubeClipsURL}
\textbf{Was ist Web-Entwicklung:} \textit{Was ist Web-Entwicklung}, 18. September 2024. \url{https://www.youtube.com/watch?v=E0NVC8xhf3I}

\bibitem[230]{SurveyStackOverflowFrameworks}
\textbf{Was ist Web-Entwicklung:} \textit{Was ist Web-Entwicklung}, 18. September 2024. \url{https://survey.stackoverflow.co/2024/technology#1-web-frameworks-and-technologies}

\bibitem[240]{ShadCNDashboard}
\textbf{Was ist Web-Entwicklung:} \textit{Was ist Web-Entwicklung}, 18. September 2024. \url{https://ui.shadcn.com/examples/dashboard}

\bibitem[250]{ShadCNSidebar}
\textbf{Was ist Web-Entwicklung:} \textit{Was ist Web-Entwicklung}, 18. September 2024. \url{https://ui.shadcn.com/blocks#sidebar-07}









\end{literature}
                                       %Literaturverzeichnis(LaTeX File) einfügen


%% Abbildungsverzeichnis %===============================%%    + Evnt Tabellenverzeichnis + Abkürzungen
\setcounter{lofdepth}{2}

\dipalistoffigures

%% Codeverzeichnis

\include{files/Anhang/codeverzeichnis}  


%%Zeitnachweise
%===============================%%
%
\include{files/zeitnachweise}                                   %Zeitnachweise(LaTeX File) einfügen

%%USB Verzeichnis
\include{files/USBVerzeichnis.tex}

%%Betreuungsprotokolle
%==========================================
%
\chapter{Betreuungsprotokolle}
%%\includepdf[pages=-]{doc/pdfs/besprechungsprotokoll_1.pdf}          %Besprechunsprotokoll(PDF) einfügen
%%\includepdf[pages=-]{doc/pdfs/besprechungsprotokoll_2.pdf}
%%\includepdf[pages=-]{doc/pdfs/besprechungsprotokoll_3.pdf}
%%usw....


\end{document}