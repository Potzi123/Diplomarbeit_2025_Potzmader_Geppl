%% Dokumentklasse KOMA-Script Report
\documentclass[paper=a4, 12pt]{scrreprt}
%% Encoding UTF8
\usepackage[utf8]{inputenc}
%%Use Source Sans Pro Textstyle
\usepackage[default]{sourcesanspro}
%% 8 Bit Aufloesung der Buchstaben
\usepackage[T1]{fontenc}
%% Seitenraender
%\usepackage[scale=0.72]{geometry}
\usepackage[scale=0.72, twoside, bindingoffset=2mm]{geometry}
%% Spracheinstellungen
\usepackage[english, naustrian]{babel} % your native language must be the last one!!
%% erweiterte Farbenpalette
\usepackage[dvipsnames]{xcolor}
%% Abbildungen
\usepackage{graphicx}
%%Tabelen mit Farbe (cellcolor)
\usepackage{tabulary}
\usepackage{colortbl}
\PassOptionsToPackage{dvipsnames,svgnames,table}{xcolor}
%% Tabellen (erweitert)
\usepackage{tabularx}
%% TikZ + Circuit-TikZ (fuer Schaltungen)
\usepackage[europeanresistors, europeaninductors]{circuitikz}
%% Nuetzliche TikZ Libraries
\usetikzlibrary{arrows, automata, positioning}
%% mathematik
\usepackage{amsmath, amssymb}
%%Formelbeschreibung
\newenvironment{conditions}
  {\par\vspace{\abovedisplayskip}\noindent\begin{tabular}{>{$}l<{$} @{${}...{}$} l}}
  {\end{tabular}\par\vspace{\belowdisplayskip}}
%\usepackage{mathtools}	
%% pdf-einbindung
\usepackage{pdfpages}
%% scource-code einbindung
\usepackage{listings, scrhack} %scrhack vermeidet Umschaltung auf KOMA
% Floats..
\usepackage{courier}
%% euro-symbol
\usepackage{eurosym}
%% landcsape-seiten ermöglichen
\usepackage{lscape}

%% Diplomarbeits-Format
\usepackage{srdpdipa}

%% Abkuerzungsverzeichnis
\usepackage[]{acronym}

%% Todos
\usepackage[]{todonotes}

%% Ganttdiagramme
\usepackage{pgfgantt}

%% Subfigures
\usepackage[lofdepth]{subfig}

%% Für Quellenangaben unter Bildern
\newcommand*{\quelle}[1]{\par\raggedleft\footnotesize Quelle:~#1}

%% Listings (Code)
\usepackage{listings}
\usepackage{color}

\definecolor{dkgreen}{rgb}{0,0.6,0}
\definecolor{gray}{rgb}{0.5,0.5,0.5}
\definecolor{mauve}{rgb}{0.58,0,0.82}

\lstset{frame=none,
  language=Python,
  aboveskip=3mm,
  belowskip=3mm,
  showstringspaces=false,
  columns=flexible,
  basicstyle={\small\ttfamily},
  numbers=left,
  numberstyle=\tiny\color{gray},
  keywordstyle=\color{blue},
  commentstyle=\color{dkgreen},
  stringstyle=\color{mauve},
  breaklines=true,
  breakatwhitespace=true,
  tabsize=3
}

\lstset{literate=%
  {Ö}{{\"O}}1
  {Ä}{{\"A}}1
  {Ü}{{\"U}}1
  {ß}{{\ss}}1
  {ü}{{\"u}}1
  {ä}{{\"a}}1
  {ö}{{\"o}}1
}

%% TIKZ Flowcharts
\usepackage{tikz}
\usetikzlibrary{shapes.geometric, arrows}

\tikzstyle{startstop} = [rectangle, rounded corners, minimum width=3cm, minimum height=1cm,text centered, draw=black, fill=green!32]
\tikzstyle{io} = [trapezium, trapezium left angle=70, trapezium right angle=110, minimum width=3cm, minimum height=1cm, text centered, draw=black, fill=red!35]
\tikzstyle{process} = [rectangle, minimum width=3cm, minimum height=1cm, text centered, draw=black, fill=cyan!32]
\tikzstyle{decision} = [diamond, minimum width=3cm, minimum height=1cm, text centered, draw=black, fill=orange!42]

\tikzstyle{arrow} = [thick,->,>=stealth]


%% definitionen =====================================%%
\dataSchool{HTBLuVA St. Pölten}
\dataDepartment{Höhere Lehranstalt für Elektronik und Technische Informatik}
\dataSubdepartment{Ausbildungsschwerpunkte Embedded- \& Wireless Systems}
\dataSession{2024/25}


\title{AirSens Classroom Data Capturing}
\author{Thomas Potzmader \and Tobias Geppl}
\date{4. April 2025}
\place{St. P\"olten}
\professor{Dipl.-Ing. Wolfgang Uriel Kuran \and DI (FH) Johannes Tomitsch}



%%====================================================%%

% Hyperlinks im Dokument
\usepackage[colorlinks=true,
    linkcolor=black,
    citecolor=black,
    bookmarks=true,
    urlcolor=black,
    bookmarksopen=true]{hyperref}

\begin{document}

\frontmatter

%% titelseite ==========================================%%
\maketitle
%%======================================================%%

%% komplett leere seite ================================%%
\newpage\null\thispagestyle{empty}%\newpage
%%======================================================%%

%% eidesstattliche erklärung ===========================%%
\begin{affidavit}
    \unterschrift{Thomas Potzmader}
    \unterschrift{Tobias Geppl}
\end{affidavit}
%%======================================================%%

%% dokumentation (deutsch/englisch) ====================%%
%%\includepdf[pages=-]{doc/pdfs/dokumentation-de.pdf}                 %Dokumentation(PDF) einfügen
%%\includepdf[pages=-]{doc/pdfs/dokumentation-en.pdf}                 %Dokumentation(PDF) einfügen
%%======================================================%%

\cleardoublepage                                              %Seite freilassen


%% diplomantenvorstellung ==============================%%
%%\includepdf[pages=-]{doc/pdfs/Tobias.pdf}                          %Vorstellung(PDF) einfügen
%%\cleardoublepage                                              %Seite freilassen
%%\includepdf[pages=-]{doc/pdfs/Thomas.pdf}                          %Vorstellung(PDF) einfügen
%%======================================================%%

%%\cleardoublepage                                              %Seite freilassen

%% danksagung ==========================================%%
\begin{acknowledgements}
\subsubsection*{Name 1}
Danksagung Inhalt
\\
Danke Danke Danke Danke Danke Danke Danke Danke Danke Danke Danke Danke Danke Danke Danke Danke 
\\                              %Absatz
\subsubsection*{Name 2}
Danksagung Inhalt
\\
Danke Danke Danke Danke Danke Danke Danke Danke Danke Danke Danke Danke Danke Danke 
\end{acknowledgements}
                             
%%======================================================%%

%%\cleardoublepage                                              %Seite freilassen

%% inhaltsverzeichnis ==================================%%
\renewcommand*\chapterpagestyle{scrheadings}
\tableofcontents
%%======================================================%%

%%\cleardoublepage                                              %Seite freilassen

%% HAUPTTEIL ===========================================%%
\responsible{Name 1, Name 2}
\mainmatter

%Chapter 1 - Einführung/Einleitung
\responsible{Thomas Potzmader, Tobias Geppl} 
\begin{inhalt}
\renewcommand*\chapterpagestyle{scrheadings}
\chapter{Einleitung}

Individuelle Zielsetzung und Aufgabenstellung

\section{Ausgangslage}
\section{Zielsetzung}


\end{inhalt}                                     %Einführung(LaTeX File) einfügen

%Chapter 2 - Grundlagen (& Methoden)
\responsible{Thomas Potzmader, Tobias Geppl} 
\begin{inhalt}
\renewcommand*\chapterpagestyle{scrheadings}
\chapter{Grundlagen}



\end{inhalt}                                      %Grundlagen(LaTeX File) einfügen

%Chapter 3 - (Methoden)
\responsible{Thomas Potzmader, Tobias Geppl} 
\begin{inhalt}
\renewcommand*\chapterpagestyle{scrheadings}
\chapter{Methoden}


\end{inhalt}                                      %Methoden(LaTeX File) einfügen

%Chapter 4 Ergebnisse
\responsible{Thomas Potzmader, Tobias Geppl} 
\begin{inhalt}
\renewcommand*\chapterpagestyle{scrheadings}
\chapter{Ergebnisse}






\end{inhalt}                                      %Ergebnisse(LaTeX File) einfügen


%Chapter 5 Hardwareentwicklung
\responsible{Tobias Geppl} 
\begin{inhalt}
\renewcommand*\chapterpagestyle{scrheadings}


\chapter{Hardwareentwicklung}

In diesem Kapitel geht es um die Entwicklung sowie den Aufbau der Hardware des Messgeräts. Dazu wurde eine Platine entworfen, um alle Komponenten miteinander zu verbinden und diese in ein dafür 3D-gedrucktes Gehäuse zu integrieren.
Das PCB wurde im 2-Layer-Design entworfen, da es für diese Anwendung ausreichend war.
Auf der Ober- und Unterseite des PCBs wurden jeweils Masseflächen angelegt, die mithilfe von Via-Stitching miteinander verbunden sind, um eine gute Verbindung herzustellen.
Die SMD-Bauteile befinden sich nur auf der Unterseite der Platine, um beidseitiges SMD-Löten zu vermeiden. Die SMD-Widerstände und die SMD-Kondensatoren sind im 1206-Format gewählt. Die Spule wurde im 3232-Format gewählt. Durch diese Größen lassen sich die SMD-Komponenten auch leicht per Hand mit dem Lötkolben auflöten.
\bigskip \\
Als Leiterbahnbreite wurde 0,7mm gewählt. Der maximale Strom einer solchen Leitung mit Außenlage auf dem PCB, entspricht ca. 1,85A, was dem vom Gerät maximal benötigtem Strom (Tab. \ref{tab:Stromverbrauch}) leicht standhält. Berechnet wurde dies mit der offiziellen Formel nach IPC-2221 \cite{IPC_2221}:
\smallskip

\begin{center}

\noindent\textbf{Gegebene Werte: }

\text{Leiterbahnbreite} = 0{,}7\,\text{mm} = 27{,}56\,\text{mil}$ 

 $\text{Kupferdicke} = 35\,\mu\text{m} = 1{,}38\,\text{mil}$ 
 
 $\Delta T = 10\,^\circ\text{C}$

\medskip

\noindent\textbf{Querschnittsfläche: } 

$A = 27{,}56 \cdot 1{,}38 = 38{,}02\,\text{mil}^2$

\medskip

\noindent\textbf{Formel nach IPC-2221 (Außenlage): }

I = 0{,}048 \cdot (10)^{0{,}44} \cdot (38{,}02)^{0{,}725} \approx 1{,}85\,\text{A}
    
\end{center}




\section{Pin Zuordnungen} \label{sec:Pin_Zuordnungen}

\textbf{Raspberry Pico W - Pinout:}

\begin{figure}[!htb]
\centering
\includegraphics[width=0.90\textwidth]{files/Tobias/pics/Pinout/pico2w-pinout.pdf}
\caption[Raspberry Pico W - Pinout]{Raspberry Pico W - Pinout}
\label{fig:PicoW_Pinout}
\end{figure}




Um ein möglichst sauberes Design zu erreichen, wurde die Pin-Zuordnung des Mikrocontrollers folgendermaßen gewählt:

\renewcommand{\arraystretch}{1}

\begin{table}[H]
\centering
\rowcolors{2}{white}{white}
\begin{tabular}{|l|c|}
\hline
\rowcolor{cyan!20}
\textbf{GPIO-Pin} & \textbf{Funktion} \\
\hline
GPIO0 & S1 \\
\hline
GPIO1 & S2 \\
\hline
GPIO4 & SDA \\
\hline
GPIO5 & SCL \\
\hline
GPIO6 & Busy \\
\hline
GPIO7 & RST \\
\hline
GPIO8 & DC \\
\hline
GPIO9 & CS \\
\hline
GPIO10 & CLK \\
\hline
GPIO11 & DIN (MOSI) \\
\hline
\end{tabular}
\caption{GPIO-Zuordnung des Raspberry Pico W}
\label{tab:GPIO_Zuordnung}
\end{table}


In Tabelle \ref{tab:GPIO_Zuordnung} sind GPIO0 und GPIO1 dem Auslesen der Taster S1 und S2 (Kap. \ref{sec:Benutzer_Interaktionen}) zugewiesen. GPIO4 und GPIO5 sind den I2C-Kommunikationsleitungen zugeordnet; auf diesen Pins liegt das „Default-I2C“ des Raspberry Pi Pico W (Abb. \ref{fig:PicoW_Pinout}). GPIO7 bis GPIO11 dienen zur Ansteuerung des Displays mit SPI (Abb. \ref{fig:Display_Pinout}).

\bigskip \\

\textbf{Display - Pinout:}

\begin{figure}[!htb]
\centering
\includegraphics[width=0.7\textwidth]{files/Tobias/pics/Pinout/2inch-LCD-Module-4_960.jpg}
\caption[Display - Pinout]{Display - Pinout}
\label{fig:Display_Pinout}
\end{figure}


\begin{table}[H]
\centering
\begin{tabular}{|l|l|}
\hline
\rowcolor{cyan!20}
\textbf{Pin} & \textbf{Funktion} \\
\hline
VCC & Versorgung (3{,}3V / 5V) \\
\hline
GND & Ground \\
\hline
DIN & SPI-Dateninput \\
\hline
CLK & SPI-Taktinput \\
\hline
CS & Chip-Select, Low-Active \\
\hline
DC & Daten-/Befehlauswahl (High = Daten, Low = Befehl) \\
\hline
RST & Reset, Low-Active \\
\hline
BL & Hintergrundbeleuchtung \\
\hline
\end{tabular}
\caption{Pinbelegung des Displays}
\label{tab:display_pins}
\end{table}

\bigskip \\

\textbf{PASCO2 - Pinout:}

\begin{figure}[!htb]
\centering
\includegraphics[width=0.6\textwidth]{files/Tobias/pics/Pinout/minieval_co2_pinout.pdf}
\caption[Display - Pinout]{Display - Pinout}
\label{fig:PASCO2_Pinout}
\end{figure}

Mithilfe des PSEL-Pins wird dem Sensor mitgeteilt, welche Kommunikationsschnittstelle verwendet wird. In diesem Fall wird PSEL für I2C auf GND gezogen.

\bigskip \\

\textbf{BME688 - Pinout:}

\begin{figure}[!htb]
\centering
\includegraphics[width=0.5\textwidth]{files/Tobias/pics/Pinout/environment-3-click-thickbox_default-2.jpg}
\caption[Display - Pinout]{Display - Pinout}
\label{fig:BME688_Pinout}
\end{figure}

Wie in Abbildung \ref{fig:BME688_Pinout} zu sehen ist, befinden sich verschiedene Auswahlmöglichkeiten auf dem ...Board, die mithilfe von Lötbrücken genutzt werden können. Die Kommunikationsschnittstelle kann zwischen SPI und I2C gewählt werden (in diesem Fall I2C), ebenso wie die Adresse des Sensors (0 = 0x76, 1 = 0x77). 





      \section{USB-C}

      Das Messgerät benutzt USB-C für die Spannungsversorgung. Da nur die Versorgungsleitungen des USB-Busses benutzt werden, wird eine USB4125-Buchse verwendet (Kap. \ref{sec:USB4125_75}). Diese hat nur die Versorgungs-, GND- und CC-Pins, welche ausreichend sind, da keine Daten übertragen werden müssen. Ebenfalls erleichtert diese Buchse das Auflöten auf die Platine.

\begin{figure}[!htb]
\centering
\includegraphics[width=0.75\textwidth]{files/Tobias/pics/Schaltungen/Schematik/USBC_Schematik.PNG}
\caption[USB-C Schematik]{USB-C Schematik}
\label{fig:USB-C_Schematik}
\end{figure}

Wie in Abbildung \ref{fig:USB-C_Schematik} zu sehen ist, sind beide CC-Pins der USB-C-Buchse über 5,1-k$\Omega$ Widerstände mit GND verbunden. Bei USB-C erfolgt über diese Pins der Austausch über die Verbindungskonfigurationen. Die 5,1-k$\Omega$ Widerstände signalisieren, dass es sich bei dem angeschlossenen Gerät um einen Verbraucher handelt, der maximal 15W Leistung aufnehmen darf \cite{USBCPowerDelivery}.

\begin{figure}[H] 
  \centering

  \begin{subfigure}[b]{0.48\textwidth}
    \centering
    \includegraphics[height=7cm]{files/Tobias/pics/Schaltungen/PCB/USB_C_PCB.PNG}
    \caption{USB-C Bottom Layer}
    \label{fig:USB-C_Bottom_layer}
  \end{subfigure}
  \hfill
  \begin{subfigure}[b]{0.48\textwidth}
    \centering
    \includegraphics[height=7cm]{files/Tobias/pics/Schaltungen/PCB/USB_C_PCB_3D.PNG}
    \caption{USB-C 3D Ansicht}
    \label{fig:USB-C_3D_Ansicht}
  \end{subfigure}

  \caption{USB-C PCB Ansicht}
  \label{fig:pcb_layers}
\end{figure}



      \section{3,3V Spannungsregler}

Da der Mikrocontroller, die Sensoren sowie das Display mit 3,3V arbeiten und die USB-C-Versorgung 5V bereitstellt, ist ein Spannungsregler notwendig. Verwendet wird der NJM12856 \cite{NJM12856}, da dieser mit 1000mA dem maximalen Strom des Geräts standhält (Tab. \ref{tab:Stromverbrauch}).

Um den maximalen Stromverbrauch des Gerätes zu bestimmen, wurden folgende Werte aus den Datenblättern der einzelnen Komponenten \cite{Raspberry_Pi_Pico_W}, \cite{PASCO2V01}, \cite{BME688}, \cite{LCDDisplayDatasheet} entnommen: 


\renewcommand{\arraystretch}{1}

\begin{table}[H]
\centering
\rowcolors{2}{white}{white}
\begin{tabular}{|l|c|}
\hline
\rowcolor{cyan!20}
\textbf{Komponente} & \textbf{max. Stromaufnahme} \\
\hline
BME688 & 18\,mA \\
\hline
PAS CO\textsubscript{2} & 160\,mA \\
\hline
Display & 45\,mA \\
\hline
Raspberry Pi Pico W & 300\,mA \\
\hline
\end{tabular}
\caption{max. Stromverbrauch der Hauptkomponenten}
\label{tab:Stromverbrauch}
\end{table}

Da für den Raspberry Pico W keine genaue Angabe über den Stromverbrauch im Datenblatt zu finden ist, wurde der obige Wert durch Internetrecherche \cite{PicoWCurrent} und unter Einbezug der benutzten Bussysteme sowie der WLAN-Verbindung geschätzt.

\begin{figure}[!htb]
\centering
\includegraphics[width=0.75\textwidth]{files/Tobias/pics/Schaltungen/Schematik/3V3_Converter_Schematik.PNG}
\caption[3,3V Spannungsregler Schematik]{3,3V Spannungsregler Schematik}
\label{fig:3,3V Spannungsregler Schematik}
\end{figure}

Wie in Abbildung \ref{fig:3,3V Spannungsregler Schematik} zu sehen ist, wurde am Ein- und Ausgang des NJM12856 jeweils ein 4,7µF-Kondensator hinzugefügt, um stabile Spannungen zu gewährleisten. Zusätzlich wurde ein weiterer 100nF-Kondensator am Ausgang des NJM12856 hinzugefügt, um mögliche Störungen zu minimieren. Der CS-Pin bleibt offen, da kein Softstart benötigt wird \cite{NJM12856}. 2 Pins wurden für einen Schalter inkludiert, um die gesamte Spannungsversorgung vom Gerät zu trennen. 


\begin{figure}[H] 
  \centering

  \begin{subfigure}[b]{0.48\textwidth}
    \centering
    \includegraphics[height=7cm]{files/Tobias/pics/Schaltungen/PCB/3V3_Spannungregler_PCB.PNG}
    \caption{3,3V Spannungsregler Bottom Layer}
    \label{fig:USB-C_Bottom_layer}
  \end{subfigure}
  \hspace{2mm} % <-- Abstand verkleinert
  \begin{subfigure}[b]{0.48\textwidth}
    \centering
    \includegraphics[height=7cm]{files/Tobias/pics/Schaltungen/PCB/3V3_Spannungsregler_PCB_3D.PNG}
    \caption{3,3V Spannungsregler 3D Ansicht}
    \label{fig:USB-C_3D_Ansicht}
  \end{subfigure}

  \caption{3,3V Spannungsregler PCB Ansicht}
  \label{fig:pcb_layers}
\end{figure}








      \section{12V Spannungwandler}
      
Der PASCO2V01-Sensor benötigt neben der 3,3V-Spannungsversorgung ebenfalls eine 12V-Spannungsversorgung. Der Aufwärtswandler TLV61046ADBVR \cite{TLV61046} wird dafür benutzt. Dieser ist in dem Datenblatt für Design-Richtlinien des PASCO2V01 \cite{PASCO2_Design_Guidelines}, inklusive Beschaltung, empfohlen.


\begin{figure}[!htb]
\centering
\includegraphics[width=0.75\textwidth]{files/Tobias/pics/Schaltungen/Schematik/12V_Converter_Schematik.PNG}
\caption[12V Spannungswandler Schematik]{12V Spannungswandler Schematik}
\label{fig:12V Spannungswandler Schematik}
\end{figure}

Die Beschaltung des TLV61046ADBVR (Abb. \ref{fig:12V Spannungswandler Schematik}) wurde entsprechend dem Datenblatt für Design-Richtlinien des PASCO2V01 \cite{PASCO2_Design_Guidelines} entnommen. 


\begin{figure}[H] 
  \centering

  \begin{subfigure}[b]{0.48\textwidth}
    \centering
    \includegraphics[height=7cm]{files/Tobias/pics/Schaltungen/PCB/12V_Spannungswandler_PCB_.PNG}
    \caption{12V Spannungswandler Bottom Layer}
    \label{fig:12V_Bottom_layer}
  \end{subfigure}
  \hfill
  \begin{subfigure}[b]{0.48\textwidth}
    \centering
    \includegraphics[height=7cm]{files/Tobias/pics/Schaltungen/PCB/12V_Spannungswandler_PCB_3D.PNG}
    \caption{12V Spannungswandler 3D Ansicht}
    \label{fig:12V_3D_Ansicht}
  \end{subfigure}

  \caption{12V Spannungswandler PCB Ansicht}
  \label{fig:pcb_layers}
\end{figure}
      


   \section{PCB Version 1}
   \label{ref:PCB_Version_1}


   \begin{figure}[H] 
  \centering
  \begin{subfigure}[b]{0.48\textwidth}
    \centering
    \includegraphics[height=7cm]{files/Tobias/pics/Schaltungen/PCB/Version1_Top.PNG}
    \caption{PCB Version 1 - Top Layer}
    \label{fig:PCB_Version1_Top}
  \end{subfigure}
  \hfill
  \begin{subfigure}[b]{0.48\textwidth}
    \centering
    \includegraphics[height=7cm]{files/Tobias/pics/Schaltungen/PCB/Version1_Bottom.PNG}
    \caption{PCB Version 1 - Bottom Layer}
    \label{fig:PCB_Version1_Bot}
  \end{subfigure}
  \caption{PCB Version 1}
  \label{fig:PCB_Version_1}
\end{figure}

Da das Display im Gehäuse befestigt wird und sich nicht direkt auf der Platine befindet, wurde ein JST XH 2.54mm-Stecker (Female) für dessen Verbindung benutzt. Als Taster wurden 2-1825027-0 Taster verwendet. Diese sind geknickt und haben eine Knopflänge von 9,24mm. Dadurch lassen sich die Taster gut in das Gehäuse integrieren. Für die Befestigung im Gehäuse wurden auf der Platine 2 Löcher für M3-Schrauben vorgesehen. Ein drittes Loch direkt unter dem Mikrocontroller dient nur als Platzhalter.

  
\section{PCB Version 2}


   \begin{figure}[H] 
  \centering
  \begin{subfigure}[b]{0.48\textwidth}
    \centering
    \includegraphics[height=7cm]{files/Tobias/pics/Schaltungen/PCB/Version2_Top.PNG}
    \caption{PCB Version 2 - Top Layer}
    \label{fig:PCB_Version2_Top}
  \end{subfigure}
  \hfill
  \begin{subfigure}[b]{0.48\textwidth}
    \centering
    \includegraphics[height=7cm]{files/Tobias/pics/Schaltungen/PCB/Version2_Bottom.PNG}
    \caption{PCB Version 2 - Bottom Layer}
    \label{fig:PCB_Version2_Bot}
  \end{subfigure}
  \caption{PCB Version 2}
  \label{fig:PCB_Version_2}
\end{figure}

Für eine bessere Benutzerfreundlichkeit wurde die USB-C-Buchse von der ursprünglich seitlichen Position auf die Rückseite des PCBs gesetzt. Ebenfalls vereinfacht dies die Integration in das Gehäuse. Dabei wurde von einer USB4125-Buchse auf eine USB4175-Buchse (Kap. \ref{sec:USB4125_75}) gewechselt. Zusätzlich wurde ein weiteres Loch für eine Schraube zur Befestigung im Gehäuse eingebaut. Die ursprünglichen Positionen der SMD-Bauteile sowie der Sensoren wurden ebenfalls geändert.

   \begin{figure}[H] 
  \centering
  \begin{subfigure}[b]{0.48\textwidth}
    \centering
    \includegraphics[height=7cm]{files/Tobias/pics/Schaltungen/3D/Top_Layer_3D.PNG}
    \caption{PCB Version 2 - Top Layer 3D Ansicht}
    \label{fig:PCB_Version2_Top_3D}
  \end{subfigure}
  \hfill
  \begin{subfigure}[b]{0.48\textwidth}
    \centering
    \includegraphics[height=7cm]{files/Tobias/pics/Schaltungen/3D/Bottom_Layer_3D.PNG}
    \caption{PCB Version 2 - Bottom Layer 3D Ansicht}
    \label{fig:PCB_Version2_Bot_3D}
  \end{subfigure}
  \caption{PCB Version 2 - 3D Ansicht}
  \label{fig:PCB_Version_2_3D}
\end{figure}



\section{Fertigung des PCBs/Prints}

Die zweite PCB-Version wurde bei JCL PCB bestellt und im Anschluss eigenständig bestückt. Da der mitgelieferte Verbindungstecker das Displays sehr lange Kabel hat und diese im Gehäuse zu viel Platz verbrauchen würden, wurden diese von 18cm Länge auf 10cm gekürtzt. Am abgetrennten Ende wurde ein JST XH 2.54mm-Stecker (Male) befestigt.



\section{Messungen}
	\subsection{Spannungen}

    Nach der Bestückung der Platine wurden die Versorgungsspannungen mit einem Multimeter überprüft. Sowohl die 3,3V- als auch die 12V-Versorgung waren mit minimalen Abweichungen korrekt.

\end{inhalt}                                      

%Chapter 6 Mikrocontroller-Programmierung
\responsible{Tobias Geppl} 
\begin{inhalt}
\renewcommand*\chapterpagestyle{scrheadings}
\chapter{Mikrocontroller - Programmierung}

\section{PAS CO2}
\section{BME688}

\section{HTTPS}

\section{Display Ansteuerung}

\section{Display Interface}
\subsection{Screens}
\subsection{Widgets}
\subsection{Buttons}



\end{inhalt}                                      


%Chapter 7 Frontend
\responsible{Thomas Potzmader} 
\include{files/Thomas/Database}     

%Chapter 8 Backend
\responsible{Thomas Potzmader} 
\begin{inhalt}
\renewcommand*\chapterpagestyle{scrheadings}
\chapter{Backend}

Einführung in Tobias Teil Kurze Erklärung

\section{Komponentenwahl}

\subsection{Sensoren}
Im ersten Schritt werden grundlegende Datenbankeinstellungen vorgenommen, die als „Sensoren“ fungieren und die Umgebung überwachen. Beispiele hierfür sind Befehle, die Zeitüberschreitungen verhindern und die richtige Zeichencodierung sicherstellen:
\begin{lstlisting}[language=SQL, caption=Grundlegende Datenbankeinstellungen]
SET statement_timeout = 0;
SET client_encoding = 'UTF8';
SET standard_conforming_strings = on;
\end{lstlisting}
Diese Einstellungen legen den Grundstein für einen stabilen Betrieb der Datenbank.

\subsection{Mikrocontroller}
Die „Mikrocontroller“ im Backend entsprechen den installierten Erweiterungen, die verschiedene Funktionen steuern und die Performance der Datenbank verbessern. Beispiele hierfür sind:
\begin{lstlisting}[language=SQL, caption=Installation von Erweiterungen]
CREATE EXTENSION IF NOT EXISTS "pgsodium" WITH SCHEMA "pgsodium";
CREATE EXTENSION IF NOT EXISTS "pg_graphql" WITH SCHEMA "graphql";
CREATE EXTENSION IF NOT EXISTS "pg_stat_statements" WITH SCHEMA "extensions";
CREATE EXTENSION IF NOT EXISTS "pgcrypto" WITH SCHEMA "extensions";
CREATE EXTENSION IF NOT EXISTS "pgjwt" WITH SCHEMA "extensions";
CREATE EXTENSION IF NOT EXISTS "supabase_vault" WITH SCHEMA "vault";
CREATE EXTENSION IF NOT EXISTS "uuid-ossp" WITH SCHEMA "extensions";
\end{lstlisting}
Diese Erweiterungen erweitern die Funktionalität, etwa in der Kryptographie oder bei der Implementierung von GraphQL-Schnittstellen.

\subsection{Bedienelemente}
Zu den „Bedienelementen“ zählt die Erstellung der Datenstrukturen, also Tabellen und zugehörige Constraints, die das Rückgrat der Datenbank bilden. Beispielsweise wird die Tabelle \texttt{classes} folgendermaßen erstellt:
\begin{lstlisting}[language=SQL, caption=Erstellung der Tabelle "classes"]
CREATE TABLE IF NOT EXISTS "public"."classes" (
    "id" uuid DEFAULT gen_random_uuid() NOT NULL,
    "created_at" timestamp with time zone DEFAULT now() NOT NULL,
    "name" text,
    "department_id" uuid DEFAULT gen_random_uuid()
);
\end{lstlisting}
Zusätzlich werden eindeutige Schlüssel (Primary Keys) und Fremdschlüsselbeziehungen definiert, um die Datenintegrität zu gewährleisten.

\section{Versorgungen \& Ansteuerungen}

\subsection{Bussysteme}
Die „Bussysteme“ im Backend spiegeln die Kommunikationswege zwischen den einzelnen Tabellen wider – insbesondere die Beziehungen, die über Fremdschlüssel hergestellt werden. Ein Beispiel:
\begin{lstlisting}[language=SQL, caption=Definition eines Fremdschlüssels]
ALTER TABLE ONLY "public"."classes"
    ADD CONSTRAINT "classes_department_id_fkey" FOREIGN KEY ("department_id")
    REFERENCES "public"."departments"("id") ON UPDATE CASCADE ON DELETE CASCADE;
\end{lstlisting}
Dieser Befehl sorgt dafür, dass Änderungen in den referenzierten Tabellen automatisch übernommen werden.

\subsection{Spannungsversorgungen}
Die „Spannungsversorgungen“ symbolisieren die Sicherheits- und Automatisierungselemente des Backends, beispielsweise Triggerfunktionen und die Vergabe von Zugriffsrechten. Ein Beispiel für eine Triggerfunktion, die beim Anlegen eines neuen Benutzers automatisch ein Profil erstellt, lautet:
\begin{lstlisting}[language=SQL, caption=Triggerfunktion handle_new_user()]
CREATE OR REPLACE FUNCTION public.handle_new_user()
 RETURNS trigger
 LANGUAGE plpgsql
 SECURITY DEFINER
 SET search_path TO ''
AS $function$
begin
  insert into public.profiles (id)
  values (new.id);
  return new;
end;
$function$
;
\end{lstlisting}
Zusätzlich werden mittels GRANT-Befehlen Zugriffsrechte für unterschiedliche Rollen definiert, um einen kontrollierten Zugriff auf die Daten zu gewährleisten.

\section{Pin-Layout \& Schematik}
Das „Pin-Layout \& Schematik“ entspricht der Gesamtstruktur der Datenbank. Hier wird das Layout der Tabellen, deren Beziehungen sowie Sicherheitsmechanismen wie Row Level Security dargestellt:
\begin{lstlisting}[language=SQL, caption=Aktivierung von Row Level Security]
ALTER TABLE "public"."classes" ENABLE ROW LEVEL SECURITY;
ALTER TABLE "public"."departments" ENABLE ROW LEVEL SECURITY;
ALTER TABLE "public"."schools" ENABLE ROW LEVEL SECURITY;
\end{lstlisting}
Diese Maßnahmen stellen sicher, dass nur autorisierte Benutzer auf bestimmte Daten zugreifen können.

\section{PCB-Design}
Im „PCB-Design“ wird das finale Zusammenspiel aller Komponenten des Backends zusammengefasst. Dies umfasst abschließend die Konfiguration der Tabellen, Triggerfunktionen und Zugriffsrechte, welche zusammen ein robustes und sicheres System bilden:
\begin{lstlisting}[language=SQL, caption=Vergabe von Zugriffsrechten]
grant select on table "public"."profiles" to "anon";
grant insert on table "public"."profiles" to "authenticated";
grant update on table "public"."profiles" to "service_role";
\end{lstlisting}
Das Endresultat ist ein Backend, das durch präzise Konfiguration und sorgfältige Planung den Anforderungen moderner Anwendungen gerecht wird.

\end{inhalt}
                                      

%Chapter 9 Frontend
\responsible{Thomas Potzmader} 
\include{files/Thomas/Frontend}      


%%ANHANG
%===================================
\appendix
\responsible{Name 1, Name 2} 

%%Literaturverzeichnis
%=============================%%
%
\begin{literature}

\section*{Internetlinks}


%%  Grundlagen & Methoden


%%TOBIAS GEPPL LINKS

\bibitem[1]{AltiumDesignerWiki}
\textbf{ALtium Designer:} \textit{Was ist ALtium Designer}, 18. September 2024. \url{https://de.wikipedia.org/wiki/Altium_Designer}

\bibitem[2]{I2CKommunkation}
\textbf{I2C Bussystem:} \textit{Was ist I2C}, 18. September 2024. \url{https://de.wikipedia.org/wiki/I%C2%B2C}

\bibitem[3]{SPI_Kommunikation}
\textbf{SPI Bussystem:} \textit{Was ist SPI}, 18. September 2024. \url{https://de.wikipedia.org/wiki/Serial_Peripheral_Interface}

\bibitem[4]{HTTPS_Kommunkation}
\textbf{HTTPS Protokoll:} \textit{Was ist HTTPS}, 18. September 2024. \url{https://de.wikipedia.org/wiki/Hypertext_Transfer_Protocol_Secure}

\bibitem[5]{VisualStudioCode}
\textbf{VS Code Texteditor:} \textit{Was ist HTTPS}, 18. September 2024. \url{https://de.wikipedia.org/wiki/Visual_Studio_Code}

\bibitem[6]{Raspberry_Pi_Pico_Erweiterung}
\textbf{Raspberry Pi Pico Erweiterung:} \textit{Was ist HTTPS}, 18. September 2024. \url{https://de.wikipedia.org/wiki/Hypertext_Transfer_Protocol_Secure}

\bibitem[7]{Raspberry_Pi_Pico_W}
\textbf{Raspberry Pi Pico W:} \textit{Was ist HTTPS}, 18. September 2024. \url{https://datasheets.raspberrypi.com/picow/pico-w-datasheet.pdf}

\bibitem[8]{PASCO2V01}
\textbf{PASCO2V01 Sensor:} \textit{Was ist HTTPS}, 18. September 2024. \url{https://www.infineon.com/dgdl/Infineon-PASCO2V01-DataSheet-v01_70-EN.pdf?fileId=8ac78c8c80027ecd01809278f1af1ba2}

\bibitem[9]{BME688}
\textbf{BME688 Sensor:} \textit{Was ist HTTPS}, 18. September 2024. 
\url{https://www.bosch-sensortec.com/media/boschsensortec/downloads/datasheets/bst-bme688-ds000.pdf}

\bibitem[10]{2Inch LCD Display Wiki}
\textbf{Waveshare 2Inch LCD Display Wiki:} \textit{Was ist HTTPS}, 18. September 2024. \url{https://www.waveshare.com/wiki/2inch_LCD_Module?Amazon}

\bibitem[11]{USB4125}
\textbf{USB4125:} \textit{Was ist HTTPS}, 18. September 2024. \url{https://gct.co/connector/usb4125}

\bibitem[12]{USB4175}
\textbf{USB4175:} \textit{Was ist HTTPS}, 18. September 2024. \url{https://gct.co/connector/usb4175}

\bibitem[13]{USBC_Kommunikation}
\textbf{USB-C Power Delivery:} \textit{Was ist HTTPS}, 18. September 2024. \url{https://resources.altium.com/de/p/add-usb-type-c-power-delivery-your-designs}

\bibitem[14]{NJM12856}
\textbf{NJM12856:} \textit{Was ist HTTPS}, 18. September 2024. \url{https://www.mouser.com/datasheet/2/294/NJM12856_E-1917311.pdf?srsltid=AfmBOopjT0bgSb_HDB46CyuoYJy9D6rqTyAHd5STTh1L03hbkJLPfPKi}

\bibitem[15]{PicoWCurrent}
\textbf{Raspberry Pico W Stromverbrauch:} \textit{Was ist HTTPS}, 18. September 2024. \url{https://peppe8o.com/raspberry-pi-pico-w-power-consumption/}

\bibitem[16]{2Inch LCD Display Datasheet}
\textbf{Waveshare 2Inch LCD Display:} \textit{Was ist HTTPS}, 18. September 2024. \url{https://files.waveshare.com/upload/b/b1/2inch_LCD_Module.pdf}

\bibitem[17]{TLV61046}
\textbf{LV61046ADBVR:} \textit{Was ist HTTPS}, 18. September 2024. \url{https://www.ti.com/lit/ds/symlink/tlv61046a.pdf?ts=1742970576219&ref_url=https%253A%252F%252Fwww.ti.com%252Fproduct%252FTLV61046A}

\bibitem[18]{PASCO2_Design_Guidelines}
\textbf{PASCO2V01 Design Guidelines:} \textit{Was ist HTTPS}, 18. September 2024. \url{https://www.infineon.com/dgdl/Infineon-PAS_CO2_General_Design-In_Guideline.docx.-ApplicationNotes-v01_02-EN.pdf?fileId=5546d4627a0b0c7b017a5174394768a1}

\bibitem[19]{Excalidraw}
\textbf{Excalidraw:} \textit{Was ist HTTPS}, 18. September 2024. \url{https://excalidraw.com/}


%%THOMAS PETER POTZMADER LINKS 

\bibitem[100]{WebEntwicklungWiki}
\textbf{Was ist Web-Entwicklung:} \textit{Was ist Web-Entwicklung}, 18. September 2024. \url{https://de.wikipedia.org/wiki/Webentwicklung#:~:text=Als%20Webentwicklung%20(englisch%20Web%20development,dagegen%20meist%20von%20Webdesignern%20%C3%BCbernommen}

\bibitem[200]{WebEntwicklungFrontendWiki}
\textbf{Was ist Web-Entwicklung:} \textit{Was ist Web-Entwicklung}, 18. September 2024. \url{https://en.wikipedia.org/wiki/Front-end_web_development}

\bibitem[300]{WebEntwicklungFrontendBackendWiki}
\textbf{Was ist Web-Entwicklung:} \textit{Was ist Web-Entwicklung}, 18. September 2024. \url{https://en.wikipedia.org/wiki/Frontend_and_backend}

\bibitem[400]{DatenBankWiki}
\textbf{Was ist Web-Entwicklung:} \textit{Was ist Web-Entwicklung}, 18. September 2024. \url{https://de.wikipedia.org/wiki/Datenbank}

\bibitem[500]{NextJSWiki}
\textbf{Was ist Web-Entwicklung:} \textit{Was ist Web-Entwicklung}, 18. September 2024. \url{https://en.wikipedia.org/wiki/Next.js}

%\bibitem[5]{NextJS}
%\textbf{Was ist Web-Entwicklung:} \textit{Was ist Web-Entwicklung}, 18. September 2024. %\url{https://nextjs.org/}

\bibitem[600]{ReactWiki}
\textbf{Was ist Web-Entwicklung:} \textit{Was ist Web-Entwicklung}, 18. September 2024. \url{https://en.wikipedia.org/wiki/React_(software)}

%\bibitem[6]{ReactWiki}
%\textbf{Was ist Web-Entwicklung:} \textit{Was ist Web-Entwicklung}, 18. September 2024. %\url{https://react.dev/}

\bibitem[700]{TypeScriptWiki}
\textbf{Was ist Web-Entwicklung:} \textit{Was ist Web-Entwicklung}, 18. September 2024. \url{https://en.wikipedia.org/wiki/TypeScript}

\bibitem[800]{TailwindWiki}
\textbf{Was ist Web-Entwicklung:} \textit{Was ist Web-Entwicklung}, 18. September 2024. \url{https://en.wikipedia.org/wiki/Tailwind_CSS}

\bibitem[900]{ShadCN}
\textbf{Was ist Web-Entwicklung:} \textit{Was ist Web-Entwicklung}, 18. September 2024. \url{https://ui.shadcn.com/docs/about}

\bibitem[1000]{Zustand}
\textbf{Was ist Web-Entwicklung:} \textit{Was ist Web-Entwicklung}, 18. September 2024. \url{https://zustand.docs.pmnd.rs/getting-started/introduction}

\bibitem[110]{Zod}
\textbf{Was ist Web-Entwicklung:} \textit{Was ist Web-Entwicklung}, 18. September 2024. \url{https://zod.dev/}

\bibitem[120]{Supabase}
\textbf{Was ist Web-Entwicklung:} \textit{Was ist Web-Entwicklung}, 18. September 2024. \url{https://supabase.com/}

\bibitem[130]{SupabaseDB}
\textbf{Was ist Web-Entwicklung:} \textit{Was ist Web-Entwicklung}, 18. September 2024. \url{https://supabase.com/docs/guides/database/overview}

\bibitem[140]{SupabaseAuth}
\textbf{Was ist Web-Entwicklung:} \textit{Was ist Web-Entwicklung}, 18. September 2024. \url{https://supabase.com/docs/guides/auth}

\bibitem[150]{SupabaseStorage}
\textbf{Was ist Web-Entwicklung:} \textit{Was ist Web-Entwicklung}, 18. September 2024. \url{https://supabase.com/docs/guides/storage}

\bibitem[160]{SupabaseRealtime}
\textbf{Was ist Web-Entwicklung:} \textit{Was ist Web-Entwicklung}, 18. September 2024. \url{https://supabase.com/docs/guides/realtime}

\bibitem[170]{Vercel}
\textbf{Was ist Web-Entwicklung:} \textit{Was ist Web-Entwicklung}, 18. September 2024. \url{https://en.wikipedia.org/wiki/Vercel}

\bibitem[180]{Fusion360Wiki}
\textbf{Was ist Web-Entwicklung:} \textit{Was ist Web-Entwicklung}, 18. September 2024. \url{https://en.wikipedia.org/wiki/Fusion_360}

\bibitem[190]{SlicerWiki}
\textbf{Was ist Web-Entwicklung:} \textit{Was ist Web-Entwicklung}, 18. September 2024. \url{https://en.wikipedia.org/wiki/Slicer_(3D_printing)}

\bibitem[204]{3DDruckerWiki}
\textbf{Was ist Web-Entwicklung:} \textit{Was ist Web-Entwicklung}, 18. September 2024. \url{https://de.wikipedia.org/wiki/3D-Druck}


%% Design & Konzept

%%TOBIAS GEPPL LINKS









%%THOMAS PETER POTZMADER LINKS 

\bibitem[210]{TemuGehaeuseURL}
\textbf{Was ist Web-Entwicklung:} \textit{Was ist Web-Entwicklung}, 18. September 2024. \url{https://www.temu.com/at/kuiper/n9.html?subj=googleshopping-landingpage&_bg_fs=1&_p_rfs=1&_x_ads_channel=google&_x_ads_sub_channel=shopping&_x_login_type=Google&_x_vst_scene=adg&mkt_rec=1&goods_id=601099796472236&sku_id=17593341043996&_x_ns_sku_id=17593341043996&_x_gmc_account=759765814&_x_ads_account=3754959944&_x_ads_set=22287958068&_x_ads_id=172614047741&_x_ads_creative_id=735412192030&_x_ns_source=g&_x_ns_gclid=CjwKCAjwp8--BhBREiwAj7og19_yz0RlMuNXGbBwIhwytKo4-4VMhTFLU35PmC44Q9M8dK-UKoXhExoCcWcQAvD_BwE&_x_ns_placement=&_x_ns_match_type=&_x_ns_ad_position=&_x_ns_product_id=17593341043996&_x_ns_target=&_x_ns_devicemodel=&_x_ns_wbraid=Cj8KCQjwhMq-BhDSARIuAFfrQ2u_rXLoGzqxmfD54jCtOS-2Q7j7dp0ChV0viGcLG9BmZL3WMPp86n2YexoC4WY&_x_ns_gbraid=0AAAAAo4mICFBjRuOi9LzOClrNUMM1XIA1&_x_ns_targetid=pla-2404451734715&gad_source=1&gclid=CjwKCAjwp8--BhBREiwAj7og19_yz0RlMuNXGbBwIhwytKo4-4VMhTFLU35PmC44Q9M8dK-UKoXhExoCcWcQAvD_BwE&adg_ctx=f-9a69b30f}

\bibitem[220]{GehaeuseYoutubeClipsURL}
\textbf{Was ist Web-Entwicklung:} \textit{Was ist Web-Entwicklung}, 18. September 2024. \url{https://www.youtube.com/watch?v=E0NVC8xhf3I}

\bibitem[230]{SurveyStackOverflowFrameworks}
\textbf{Was ist Web-Entwicklung:} \textit{Was ist Web-Entwicklung}, 18. September 2024. \url{https://survey.stackoverflow.co/2024/technology#1-web-frameworks-and-technologies}

\bibitem[240]{ShadCNDashboard}
\textbf{Was ist Web-Entwicklung:} \textit{Was ist Web-Entwicklung}, 18. September 2024. \url{https://ui.shadcn.com/examples/dashboard}

\bibitem[250]{ShadCNSidebar}
\textbf{Was ist Web-Entwicklung:} \textit{Was ist Web-Entwicklung}, 18. September 2024. \url{https://ui.shadcn.com/blocks#sidebar-07}









\end{literature}
                                       %Literaturverzeichnis(LaTeX File) einfügen


%% Abbildungsverzeichnis %===============================%%    + Evnt Tabellenverzeichnis + Abkürzungen
\setcounter{lofdepth}{2}

\dipalistoffigures


%%Zeitnachweise
%===============================%%
%
\include{files/zeitnachweise}                                   %Zeitnachweise(LaTeX File) einfügen

%%USB Verzeichnis
\include{files/USBVerzeichnis.tex}

%%Betreuungsprotokolle
%==========================================
%
\chapter{Betreuungsprotokolle}
%%\includepdf[pages=-]{doc/pdfs/besprechungsprotokoll_1.pdf}          %Besprechunsprotokoll(PDF) einfügen
%%\includepdf[pages=-]{doc/pdfs/besprechungsprotokoll_2.pdf}
%%\includepdf[pages=-]{doc/pdfs/besprechungsprotokoll_3.pdf}
%%usw....


\end{document}